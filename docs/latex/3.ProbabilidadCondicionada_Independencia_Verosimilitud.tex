
% Default to the notebook output style

% Inherit from the specified cell style.
   
\documentclass[11pt]{article}

    \usepackage[T1]{fontenc}
    % Nicer default font (+ math font) than Computer Modern for most use cases
    \usepackage{mathpazo}

    % Basic figure setup, for now with no caption control since it's done
    % automatically by Pandoc (which extracts ![](path) syntax from Markdown).
    \usepackage{graphicx}
    % We will generate all images so they have a width \maxwidth. This means
    % that they will get their normal width if they fit onto the page, but
    % are scaled down if they would overflow the margins.
    \makeatletter
    \def\maxwidth{\ifdim\Gin@nat@width>\linewidth\linewidth
    \else\Gin@nat@width\fi}
    \makeatother
    \let\Oldincludegraphics\includegraphics
    % Set max figure width to be 80% of text width, for now hardcoded.
    \renewcommand{\includegraphics}[1]{\Oldincludegraphics[width=.8\maxwidth]{#1}}
    % Ensure that by default, figures have no caption (until we provide a
    % proper Figure object with a Caption API and a way to capture that
    % in the conversion process - todo).
    \usepackage{caption}
    \DeclareCaptionLabelFormat{nolabel}{}
    \captionsetup{labelformat=nolabel}

    \usepackage{adjustbox} % Used to constrain images to a maximum size 
    \usepackage{xcolor} % Allow colors to be defined
    \usepackage{enumerate} % Needed for markdown enumerations to work
    \usepackage{geometry} % Used to adjust the document margins
    \usepackage{amsmath} % Equations
    \usepackage{amssymb} % Equations
    \usepackage{textcomp} % defines textquotesingle
    % Hack from http://tex.stackexchange.com/a/47451/13684:
    \AtBeginDocument{%
        \def\PYZsq{\textquotesingle}% Upright quotes in Pygmentized code
    }
    \usepackage{upquote} % Upright quotes for verbatim code
    \usepackage{eurosym} % defines \euro
    \usepackage[mathletters]{ucs} % Extended unicode (utf-8) support
    \usepackage[utf8x]{inputenc} % Allow utf-8 characters in the tex document
    \usepackage{fancyvrb} % verbatim replacement that allows latex
    \usepackage{grffile} % extends the file name processing of package graphics 
                         % to support a larger range 
    % The hyperref package gives us a pdf with properly built
    % internal navigation ('pdf bookmarks' for the table of contents,
    % internal cross-reference links, web links for URLs, etc.)
    \usepackage{hyperref}
    \usepackage{longtable} % longtable support required by pandoc >1.10
    \usepackage{booktabs}  % table support for pandoc > 1.12.2
    \usepackage[inline]{enumitem} % IRkernel/repr support (it uses the enumerate* environment)
    \usepackage[normalem]{ulem} % ulem is needed to support strikethroughs (\sout)
                                % normalem makes italics be italics, not underlines
    \usepackage{mathrsfs}
    

    
    
    % Colors for the hyperref package
    \definecolor{urlcolor}{rgb}{0,.145,.698}
    \definecolor{linkcolor}{rgb}{.71,0.21,0.01}
    \definecolor{citecolor}{rgb}{.12,.54,.11}

    % ANSI colors
    \definecolor{ansi-black}{HTML}{3E424D}
    \definecolor{ansi-black-intense}{HTML}{282C36}
    \definecolor{ansi-red}{HTML}{E75C58}
    \definecolor{ansi-red-intense}{HTML}{B22B31}
    \definecolor{ansi-green}{HTML}{00A250}
    \definecolor{ansi-green-intense}{HTML}{007427}
    \definecolor{ansi-yellow}{HTML}{DDB62B}
    \definecolor{ansi-yellow-intense}{HTML}{B27D12}
    \definecolor{ansi-blue}{HTML}{208FFB}
    \definecolor{ansi-blue-intense}{HTML}{0065CA}
    \definecolor{ansi-magenta}{HTML}{D160C4}
    \definecolor{ansi-magenta-intense}{HTML}{A03196}
    \definecolor{ansi-cyan}{HTML}{60C6C8}
    \definecolor{ansi-cyan-intense}{HTML}{258F8F}
    \definecolor{ansi-white}{HTML}{C5C1B4}
    \definecolor{ansi-white-intense}{HTML}{A1A6B2}
    \definecolor{ansi-default-inverse-fg}{HTML}{FFFFFF}
    \definecolor{ansi-default-inverse-bg}{HTML}{000000}

    % commands and environments needed by pandoc snippets
    % extracted from the output of `pandoc -s`
    \providecommand{\tightlist}{%
      \setlength{\itemsep}{0pt}\setlength{\parskip}{0pt}}
    \DefineVerbatimEnvironment{Highlighting}{Verbatim}{commandchars=\\\{\}}
    % Add ',fontsize=\small' for more characters per line
    \newenvironment{Shaded}{}{}
    \newcommand{\KeywordTok}[1]{\textcolor[rgb]{0.00,0.44,0.13}{\textbf{{#1}}}}
    \newcommand{\DataTypeTok}[1]{\textcolor[rgb]{0.56,0.13,0.00}{{#1}}}
    \newcommand{\DecValTok}[1]{\textcolor[rgb]{0.25,0.63,0.44}{{#1}}}
    \newcommand{\BaseNTok}[1]{\textcolor[rgb]{0.25,0.63,0.44}{{#1}}}
    \newcommand{\FloatTok}[1]{\textcolor[rgb]{0.25,0.63,0.44}{{#1}}}
    \newcommand{\CharTok}[1]{\textcolor[rgb]{0.25,0.44,0.63}{{#1}}}
    \newcommand{\StringTok}[1]{\textcolor[rgb]{0.25,0.44,0.63}{{#1}}}
    \newcommand{\CommentTok}[1]{\textcolor[rgb]{0.38,0.63,0.69}{\textit{{#1}}}}
    \newcommand{\OtherTok}[1]{\textcolor[rgb]{0.00,0.44,0.13}{{#1}}}
    \newcommand{\AlertTok}[1]{\textcolor[rgb]{1.00,0.00,0.00}{\textbf{{#1}}}}
    \newcommand{\FunctionTok}[1]{\textcolor[rgb]{0.02,0.16,0.49}{{#1}}}
    \newcommand{\RegionMarkerTok}[1]{{#1}}
    \newcommand{\ErrorTok}[1]{\textcolor[rgb]{1.00,0.00,0.00}{\textbf{{#1}}}}
    \newcommand{\NormalTok}[1]{{#1}}
    
    % Additional commands for more recent versions of Pandoc
    \newcommand{\ConstantTok}[1]{\textcolor[rgb]{0.53,0.00,0.00}{{#1}}}
    \newcommand{\SpecialCharTok}[1]{\textcolor[rgb]{0.25,0.44,0.63}{{#1}}}
    \newcommand{\VerbatimStringTok}[1]{\textcolor[rgb]{0.25,0.44,0.63}{{#1}}}
    \newcommand{\SpecialStringTok}[1]{\textcolor[rgb]{0.73,0.40,0.53}{{#1}}}
    \newcommand{\ImportTok}[1]{{#1}}
    \newcommand{\DocumentationTok}[1]{\textcolor[rgb]{0.73,0.13,0.13}{\textit{{#1}}}}
    \newcommand{\AnnotationTok}[1]{\textcolor[rgb]{0.38,0.63,0.69}{\textbf{\textit{{#1}}}}}
    \newcommand{\CommentVarTok}[1]{\textcolor[rgb]{0.38,0.63,0.69}{\textbf{\textit{{#1}}}}}
    \newcommand{\VariableTok}[1]{\textcolor[rgb]{0.10,0.09,0.49}{{#1}}}
    \newcommand{\ControlFlowTok}[1]{\textcolor[rgb]{0.00,0.44,0.13}{\textbf{{#1}}}}
    \newcommand{\OperatorTok}[1]{\textcolor[rgb]{0.40,0.40,0.40}{{#1}}}
    \newcommand{\BuiltInTok}[1]{{#1}}
    \newcommand{\ExtensionTok}[1]{{#1}}
    \newcommand{\PreprocessorTok}[1]{\textcolor[rgb]{0.74,0.48,0.00}{{#1}}}
    \newcommand{\AttributeTok}[1]{\textcolor[rgb]{0.49,0.56,0.16}{{#1}}}
    \newcommand{\InformationTok}[1]{\textcolor[rgb]{0.38,0.63,0.69}{\textbf{\textit{{#1}}}}}
    \newcommand{\WarningTok}[1]{\textcolor[rgb]{0.38,0.63,0.69}{\textbf{\textit{{#1}}}}}
    
    
    % Define a nice break command that doesn't care if a line doesn't already
    % exist.
    \def\br{\hspace*{\fill} \\* }
    % Math Jax compatibility definitions
    \def\gt{>}
    \def\lt{<}
    \let\Oldtex\TeX
    \let\Oldlatex\LaTeX
    \renewcommand{\TeX}{\textrm{\Oldtex}}
    \renewcommand{\LaTeX}{\textrm{\Oldlatex}}
    % Document parameters
    % Document title
    \title{I.3.Probabilidad Condicionada. Independencia. Verosimilitud}
    
    
    
    
    

    % Pygments definitions
    
\makeatletter
\def\PY@reset{\let\PY@it=\relax \let\PY@bf=\relax%
    \let\PY@ul=\relax \let\PY@tc=\relax%
    \let\PY@bc=\relax \let\PY@ff=\relax}
\def\PY@tok#1{\csname PY@tok@#1\endcsname}
\def\PY@toks#1+{\ifx\relax#1\empty\else%
    \PY@tok{#1}\expandafter\PY@toks\fi}
\def\PY@do#1{\PY@bc{\PY@tc{\PY@ul{%
    \PY@it{\PY@bf{\PY@ff{#1}}}}}}}
\def\PY#1#2{\PY@reset\PY@toks#1+\relax+\PY@do{#2}}

\expandafter\def\csname PY@tok@w\endcsname{\def\PY@tc##1{\textcolor[rgb]{0.73,0.73,0.73}{##1}}}
\expandafter\def\csname PY@tok@c\endcsname{\let\PY@it=\textit\def\PY@tc##1{\textcolor[rgb]{0.25,0.50,0.50}{##1}}}
\expandafter\def\csname PY@tok@cp\endcsname{\def\PY@tc##1{\textcolor[rgb]{0.74,0.48,0.00}{##1}}}
\expandafter\def\csname PY@tok@k\endcsname{\let\PY@bf=\textbf\def\PY@tc##1{\textcolor[rgb]{0.00,0.50,0.00}{##1}}}
\expandafter\def\csname PY@tok@kp\endcsname{\def\PY@tc##1{\textcolor[rgb]{0.00,0.50,0.00}{##1}}}
\expandafter\def\csname PY@tok@kt\endcsname{\def\PY@tc##1{\textcolor[rgb]{0.69,0.00,0.25}{##1}}}
\expandafter\def\csname PY@tok@o\endcsname{\def\PY@tc##1{\textcolor[rgb]{0.40,0.40,0.40}{##1}}}
\expandafter\def\csname PY@tok@ow\endcsname{\let\PY@bf=\textbf\def\PY@tc##1{\textcolor[rgb]{0.67,0.13,1.00}{##1}}}
\expandafter\def\csname PY@tok@nb\endcsname{\def\PY@tc##1{\textcolor[rgb]{0.00,0.50,0.00}{##1}}}
\expandafter\def\csname PY@tok@nf\endcsname{\def\PY@tc##1{\textcolor[rgb]{0.00,0.00,1.00}{##1}}}
\expandafter\def\csname PY@tok@nc\endcsname{\let\PY@bf=\textbf\def\PY@tc##1{\textcolor[rgb]{0.00,0.00,1.00}{##1}}}
\expandafter\def\csname PY@tok@nn\endcsname{\let\PY@bf=\textbf\def\PY@tc##1{\textcolor[rgb]{0.00,0.00,1.00}{##1}}}
\expandafter\def\csname PY@tok@ne\endcsname{\let\PY@bf=\textbf\def\PY@tc##1{\textcolor[rgb]{0.82,0.25,0.23}{##1}}}
\expandafter\def\csname PY@tok@nv\endcsname{\def\PY@tc##1{\textcolor[rgb]{0.10,0.09,0.49}{##1}}}
\expandafter\def\csname PY@tok@no\endcsname{\def\PY@tc##1{\textcolor[rgb]{0.53,0.00,0.00}{##1}}}
\expandafter\def\csname PY@tok@nl\endcsname{\def\PY@tc##1{\textcolor[rgb]{0.63,0.63,0.00}{##1}}}
\expandafter\def\csname PY@tok@ni\endcsname{\let\PY@bf=\textbf\def\PY@tc##1{\textcolor[rgb]{0.60,0.60,0.60}{##1}}}
\expandafter\def\csname PY@tok@na\endcsname{\def\PY@tc##1{\textcolor[rgb]{0.49,0.56,0.16}{##1}}}
\expandafter\def\csname PY@tok@nt\endcsname{\let\PY@bf=\textbf\def\PY@tc##1{\textcolor[rgb]{0.00,0.50,0.00}{##1}}}
\expandafter\def\csname PY@tok@nd\endcsname{\def\PY@tc##1{\textcolor[rgb]{0.67,0.13,1.00}{##1}}}
\expandafter\def\csname PY@tok@s\endcsname{\def\PY@tc##1{\textcolor[rgb]{0.73,0.13,0.13}{##1}}}
\expandafter\def\csname PY@tok@sd\endcsname{\let\PY@it=\textit\def\PY@tc##1{\textcolor[rgb]{0.73,0.13,0.13}{##1}}}
\expandafter\def\csname PY@tok@si\endcsname{\let\PY@bf=\textbf\def\PY@tc##1{\textcolor[rgb]{0.73,0.40,0.53}{##1}}}
\expandafter\def\csname PY@tok@se\endcsname{\let\PY@bf=\textbf\def\PY@tc##1{\textcolor[rgb]{0.73,0.40,0.13}{##1}}}
\expandafter\def\csname PY@tok@sr\endcsname{\def\PY@tc##1{\textcolor[rgb]{0.73,0.40,0.53}{##1}}}
\expandafter\def\csname PY@tok@ss\endcsname{\def\PY@tc##1{\textcolor[rgb]{0.10,0.09,0.49}{##1}}}
\expandafter\def\csname PY@tok@sx\endcsname{\def\PY@tc##1{\textcolor[rgb]{0.00,0.50,0.00}{##1}}}
\expandafter\def\csname PY@tok@m\endcsname{\def\PY@tc##1{\textcolor[rgb]{0.40,0.40,0.40}{##1}}}
\expandafter\def\csname PY@tok@gh\endcsname{\let\PY@bf=\textbf\def\PY@tc##1{\textcolor[rgb]{0.00,0.00,0.50}{##1}}}
\expandafter\def\csname PY@tok@gu\endcsname{\let\PY@bf=\textbf\def\PY@tc##1{\textcolor[rgb]{0.50,0.00,0.50}{##1}}}
\expandafter\def\csname PY@tok@gd\endcsname{\def\PY@tc##1{\textcolor[rgb]{0.63,0.00,0.00}{##1}}}
\expandafter\def\csname PY@tok@gi\endcsname{\def\PY@tc##1{\textcolor[rgb]{0.00,0.63,0.00}{##1}}}
\expandafter\def\csname PY@tok@gr\endcsname{\def\PY@tc##1{\textcolor[rgb]{1.00,0.00,0.00}{##1}}}
\expandafter\def\csname PY@tok@ge\endcsname{\let\PY@it=\textit}
\expandafter\def\csname PY@tok@gs\endcsname{\let\PY@bf=\textbf}
\expandafter\def\csname PY@tok@gp\endcsname{\let\PY@bf=\textbf\def\PY@tc##1{\textcolor[rgb]{0.00,0.00,0.50}{##1}}}
\expandafter\def\csname PY@tok@go\endcsname{\def\PY@tc##1{\textcolor[rgb]{0.53,0.53,0.53}{##1}}}
\expandafter\def\csname PY@tok@gt\endcsname{\def\PY@tc##1{\textcolor[rgb]{0.00,0.27,0.87}{##1}}}
\expandafter\def\csname PY@tok@err\endcsname{\def\PY@bc##1{\setlength{\fboxsep}{0pt}\fcolorbox[rgb]{1.00,0.00,0.00}{1,1,1}{\strut ##1}}}
\expandafter\def\csname PY@tok@kc\endcsname{\let\PY@bf=\textbf\def\PY@tc##1{\textcolor[rgb]{0.00,0.50,0.00}{##1}}}
\expandafter\def\csname PY@tok@kd\endcsname{\let\PY@bf=\textbf\def\PY@tc##1{\textcolor[rgb]{0.00,0.50,0.00}{##1}}}
\expandafter\def\csname PY@tok@kn\endcsname{\let\PY@bf=\textbf\def\PY@tc##1{\textcolor[rgb]{0.00,0.50,0.00}{##1}}}
\expandafter\def\csname PY@tok@kr\endcsname{\let\PY@bf=\textbf\def\PY@tc##1{\textcolor[rgb]{0.00,0.50,0.00}{##1}}}
\expandafter\def\csname PY@tok@bp\endcsname{\def\PY@tc##1{\textcolor[rgb]{0.00,0.50,0.00}{##1}}}
\expandafter\def\csname PY@tok@fm\endcsname{\def\PY@tc##1{\textcolor[rgb]{0.00,0.00,1.00}{##1}}}
\expandafter\def\csname PY@tok@vc\endcsname{\def\PY@tc##1{\textcolor[rgb]{0.10,0.09,0.49}{##1}}}
\expandafter\def\csname PY@tok@vg\endcsname{\def\PY@tc##1{\textcolor[rgb]{0.10,0.09,0.49}{##1}}}
\expandafter\def\csname PY@tok@vi\endcsname{\def\PY@tc##1{\textcolor[rgb]{0.10,0.09,0.49}{##1}}}
\expandafter\def\csname PY@tok@vm\endcsname{\def\PY@tc##1{\textcolor[rgb]{0.10,0.09,0.49}{##1}}}
\expandafter\def\csname PY@tok@sa\endcsname{\def\PY@tc##1{\textcolor[rgb]{0.73,0.13,0.13}{##1}}}
\expandafter\def\csname PY@tok@sb\endcsname{\def\PY@tc##1{\textcolor[rgb]{0.73,0.13,0.13}{##1}}}
\expandafter\def\csname PY@tok@sc\endcsname{\def\PY@tc##1{\textcolor[rgb]{0.73,0.13,0.13}{##1}}}
\expandafter\def\csname PY@tok@dl\endcsname{\def\PY@tc##1{\textcolor[rgb]{0.73,0.13,0.13}{##1}}}
\expandafter\def\csname PY@tok@s2\endcsname{\def\PY@tc##1{\textcolor[rgb]{0.73,0.13,0.13}{##1}}}
\expandafter\def\csname PY@tok@sh\endcsname{\def\PY@tc##1{\textcolor[rgb]{0.73,0.13,0.13}{##1}}}
\expandafter\def\csname PY@tok@s1\endcsname{\def\PY@tc##1{\textcolor[rgb]{0.73,0.13,0.13}{##1}}}
\expandafter\def\csname PY@tok@mb\endcsname{\def\PY@tc##1{\textcolor[rgb]{0.40,0.40,0.40}{##1}}}
\expandafter\def\csname PY@tok@mf\endcsname{\def\PY@tc##1{\textcolor[rgb]{0.40,0.40,0.40}{##1}}}
\expandafter\def\csname PY@tok@mh\endcsname{\def\PY@tc##1{\textcolor[rgb]{0.40,0.40,0.40}{##1}}}
\expandafter\def\csname PY@tok@mi\endcsname{\def\PY@tc##1{\textcolor[rgb]{0.40,0.40,0.40}{##1}}}
\expandafter\def\csname PY@tok@il\endcsname{\def\PY@tc##1{\textcolor[rgb]{0.40,0.40,0.40}{##1}}}
\expandafter\def\csname PY@tok@mo\endcsname{\def\PY@tc##1{\textcolor[rgb]{0.40,0.40,0.40}{##1}}}
\expandafter\def\csname PY@tok@ch\endcsname{\let\PY@it=\textit\def\PY@tc##1{\textcolor[rgb]{0.25,0.50,0.50}{##1}}}
\expandafter\def\csname PY@tok@cm\endcsname{\let\PY@it=\textit\def\PY@tc##1{\textcolor[rgb]{0.25,0.50,0.50}{##1}}}
\expandafter\def\csname PY@tok@cpf\endcsname{\let\PY@it=\textit\def\PY@tc##1{\textcolor[rgb]{0.25,0.50,0.50}{##1}}}
\expandafter\def\csname PY@tok@c1\endcsname{\let\PY@it=\textit\def\PY@tc##1{\textcolor[rgb]{0.25,0.50,0.50}{##1}}}
\expandafter\def\csname PY@tok@cs\endcsname{\let\PY@it=\textit\def\PY@tc##1{\textcolor[rgb]{0.25,0.50,0.50}{##1}}}

\def\PYZbs{\char`\\}
\def\PYZus{\char`\_}
\def\PYZob{\char`\{}
\def\PYZcb{\char`\}}
\def\PYZca{\char`\^}
\def\PYZam{\char`\&}
\def\PYZlt{\char`\<}
\def\PYZgt{\char`\>}
\def\PYZsh{\char`\#}
\def\PYZpc{\char`\%}
\def\PYZdl{\char`\$}
\def\PYZhy{\char`\-}
\def\PYZsq{\char`\'}
\def\PYZdq{\char`\"}
\def\PYZti{\char`\~}
% for compatibility with earlier versions
\def\PYZat{@}
\def\PYZlb{[}
\def\PYZrb{]}
\makeatother


    % Exact colors from NB
    \definecolor{incolor}{rgb}{0.0, 0.0, 0.5}
    \definecolor{outcolor}{rgb}{0.545, 0.0, 0.0}



    
    % Prevent overflowing lines due to hard-to-break entities
    \sloppy 
    % Setup hyperref package
    \hypersetup{
      breaklinks=true,  % so long urls are correctly broken across lines
      colorlinks=true,
      urlcolor=urlcolor,
      linkcolor=linkcolor,
      citecolor=citecolor,
      }
    % Slightly bigger margins than the latex defaults
    
    \geometry{verbose,tmargin=1in,bmargin=1in,lmargin=1in,rmargin=1in}
    
 %%%%%%%%%%%%%%%%%%%%%%%%%%%%%%%%%%%%%%%%%%%%%%%%%%
%%%%%%%%%%%%%%%%%%%%%%%%%%%%%%%%%%%%%%%%%%%%%%%%%%
%%%%%%%%%%%%%%%%%%%%%%%%%%%%%%%%%%%%%%%%%%%%%%%%%%   

    \begin{document}
    
    
    \maketitle
    
    

    
    \section*{I.3 Probabilidad condicionada. Independencia de sucesos.
Verosimilitud}\label{i.3-probabilidad-condicionada.-independencia-de-sucesos.-verosimilitud}

\subsection*{Probabilidad condicionada}\label{probabilidad-condicionada}

Consideremos que tenemos un espacio de probabilidad
\((\Omega, \mathscr{F}, P)\). Recordemos que un \textbf{suceso}
\(A \in \mathscr{F}\), \(A \subset \Omega\) es un subconjunto formado
por algunos resultados posibles del espacio muestral. También que el
suceso acontece si lo hace cualquiera de los resultados que lo componen.

Supongamos ahora que sabemos que acontece otro suceso \(B\) del espacio
de probabilidad. ?`Podemos utilizar este conocimiento para mejorar
nuestro conocimineto de \(A\)? Por ejemplo, si se lanza un dado bueno
sabemos que las probabilidades de todas sus caras son idénticas e
iguales a \(\frac{1}{6}\). Si nos dicen que el resultado que ha salido
es par, esto es el suceso \(\{par\}\), ello nos permite saber que las
probabilidades de todos los resultados impares es nula, y que el de los
pares pasa a ser de \(\frac{1}{3}\) cada una.

%%%%%%%%%%%%%%%%%%%%%%%%%%%%%%%%%%

    Consideremos que estamos interesados en el suceso \(A\), que tiene una
probabilidad \(P(A)\). Supongamos que observamos un suceso relacionado
\(B\). ?`Aumenta de algún modo la certidumbre que tenemos en relación al
suceso \(A\) por haber acontecido otro relacionado \(B\)?

Definimos la \textbf{probabilidad del suceso \(A\) condicionada por el
suceso \(B\)} como

\[P(A/B) = \frac{P(A\cap B)}{P(B)}\]

%%%%%%%%%%%%%%%%%%%%%%%%%%%%%%%%%%%

    Como sabemos que \(B\), el \textbf{suceso condicionante}, acontece,
podemos interpretarlo como el espacio muestral en un nuevo espacio de
probabilidad con una nueva asignación de probabilidades
\(P(\bullet/B)\). Adviértase que: 
\begin{itemize}
\item \(B\) se convierte en el suceso
seguro al saber que ha acontecido, por lo que \(P(B/B)=1\). Dado que
\(B\cap B = B\), la división en la definión por \(P(B)\) nos asegura
que, efectivamente, el nuevo suceso seguro tiene probabilidad uno. 
\item Si
\(A \subset B\) entonces \(P(A/B) = \frac{P(A)}{P(B)}\). 
\item Sin embargo
\(A\), el suceso cuya probabilidad condicionada queremos calcular, puede
tener elementos que no están en \(B\) aunque sí en el espacio muestral
original. \(A\cap B\) excluye los elementos de \(A\) que no están en
\(B\), razón por la que se considera \(P(A\cap B)\).
\end{itemize}

%%%%%%%%%%%%%%%%%%%%%%%%%%%%%%%%%%%%

    Consideremos un dado bueno (sin trucar) con seis caras numeradas.
Supongamos que estamos interesados en conocer la probabilidad de cuatro
sucesos:
\(A =\{2\}, \, B=\{5\}, \, C=A\cup B =\{2, 5\}, \, D=\{3, 4, 6\}\):
\begin{itemize}
 \item
\(P(A) = P(B) = \frac{1}{6}\) 
\item
\(P(C) = P(A\cup B)=P(A)+P(B)=\frac{1}{6}+\frac{1}{6}=\frac{1}{3}\) 
\item
\(P(D) = 3 \times \frac{1}{6} = \frac{1}{2}\)
\end{itemize}

%%%%%%%%%%%%%%%%%%%%%%%%%%%%%%%%%%%%

    Consideremos ahora que sabemos que ha salido \emph{par}, que corresponde
al suceso \(par=\{2,4,6\}\), aunque no sabemos qué número. ?`Podemos
mejorar el cálculo de probabilidades? 

\begin{itemize}
\item
\(P(A/par) = \frac{P(A\cap par)}{P(par)} =\frac{P(A)}{P(par)}=\frac{P(2)}{P(par)}=\frac{1/6}{1/2}=\frac{1}{3}\)
\item
\(P(B/par) = \frac{P(B\cap par)}{P(par)} =\frac{P(\emptyset)}{P(par)}=\frac{0}{1/2}=0\)
\item
\(P(C/par) = \frac{P(C\cap par)}{P(par)} =\frac{P(A)}{P(par)}=\frac{P(2)}{P(par)}=\frac{1/6}{1/2}=\frac{1}{3}\)
\item
\(P(D/par) = \frac{P(D\cap par)}{P(par)} = \frac{P(\{4,6\})}{P(par)} =\frac{2\times 1/6}{1/2}=\frac{2}{3}\)
\end{itemize}

%%%%%%%%%%%%%%%%%%%%%%%%%%%%%%%%%%%%%

    \begin{center}
    \adjustimage{max size={0.9\linewidth}{0.9\paperheight}}{3.ProbabilidadCondicionada_Independencia_Verosimilitud_files/3.ProbabilidadCondicionada_Independencia_Verosimilitud_6_0.png}
    \end{center}
    { \hspace*{\fill} \\}

%%%%%%%%%%%%%%%%%%%%%%%%%%%%%%%%%%%%%%

    \subsection*{Independencia de sucesos}\label{independencia-de-sucesos}

Dos sucesos \(A\) y \(B\) son independientes cuando que uno de ellos
acontezca no tiene ningún efecto en que lo haga o lo deje de hacer el
otro. En tal caso:

\[P(A/B) =\frac{P(A\cap B)}{P(B)}= P(A) \quad \text{y} \quad P(B/A)=\frac{P(B\cap A)}{P(A)}=P(B)\]

y, consecuentemente, \(P(A\cap B) = P(A)P(B)\) que suele tomarse también
como definición de independencia.

Es muy importante \textbf{no confundir sucesos independientes y sucesos
incompatibles}. Recordemos que éstos son sucesos disjuntos, esto es, su
intersección es nula y, por tanto, no pueden acontecer simultáneamente.
Por tanto, si uno sucede sabemos que el otro no puede suceder. Ello es
radicalmente diferente de la independencia, que significa que si un
suceso sucede no sabemos nada de qué va a pasar con el otro.

%%%%%%%%%%%%%%%%%%%%%%%%%%%%%%%%%%%%%%

    Consideremos de nuevo el experimento de probabilidad correspondiente al
lanzamiento de un dado bueno.

Los sucesos \(par\) e \(impar\) son disjuntos
\(\left(A \cap B = \emptyset \right)\) y, por ello, incompatibles. Si el
resutado es par, ciertamente sabemos que no puede haber salido impar.
Igualmente sucede con los sucesos \(div3\) (divisible por 3) y
\(nodiv3\) (no divisible por 3) y, en general, con cualquier conjunto de
sucesos disjuntos.

Pero, supongamos ahora que sabemos que el resultado ha sido par. ¿Afecta
ello a que el resultado haya sido no divisible por 3? Si no lo hace, los
sucesos \(par\) y \(nodiv3\) son independientes. Veámoslo...

%%%%%%%%%%%%%%%%%%%%%%%%%%%%%%%%%%%%%%%

    Es evidente que:
\begin{itemize}
\item \(par \cap nodiv3 = \{2,4\} \implies P(par \cap nodiv3)=\frac{2}{6}=\frac{1}{3}\)
\item \(P(nodiv3)=P(\{1,2,4,5\})=\frac{4}{6}=\frac{2}{3}\)
\end{itemize}

Por tanto:

\begin{itemize}
\tightlist
\item
  \(P(nodiv3 / par) =\frac{P(par \cap nodiv3)}{P(par)}=\frac{\frac{2}{6}}{\frac{3}{6}}=\frac{2}{3}\)
\item
  \(P(nodiv3)P(par)=\frac{2}{3}\frac{1}{2}=\frac{1}{3}\)
\end{itemize}

Vemos que, como dice la intuición si se reflexiona sobre ello,
\(nodiv3\) y \(par\) son independientes, ya que:
\begin{itemize}
\item \(P(nodiv3 / par) = P(nodiv3)=\frac{2}{3}\)
\item \(P(par \cap nodiv3)=P(nodiv3)P(par)=\frac{1}{3}\)
\end{itemize}


%%%%%%%%%%%%%%%%%%%%%%%%%%%%%%%%%%%%%%%%

    \subsection{Verosimilitud}\label{verosimilitud}

La \textbf{verosimilitud} (\emph{likelihood}) corresponde a la
probabilidad condicionada, dejándose variar el suceso condicionante. Si
se satisface el suceso \(A\), la verosimilitud evalúa cuanto varía la
probabilidad condicionada de dicha observación con los sucesos
condicionantes \(B_i\), \(i \in \mathbb{N}\):

\[L(B_i)\equiv P(A/B_i)\qquad i \in \mathbb{N}\]

A la vista del suceso \(A\) el principio de \textbf{"máxima
verosimilitud"} (\emph{maximum likelihood}) o \textbf{ML} considera que
el el suceso condicionante \textbf{más verosímil} para la observacion es
el que hace máxima la verosimilitud \(L(B_i)\):

\[B_i \quad / \quad max(L(A/B_i)) \quad \forall i \in \mathbb{N}\]

    Un sencillo ejemplo muestra que el principio de máxima verosimilitud
corresponde al \emph{sentido común}. Consideremos un dado, dos posibles
sucesos condicionantes, \(div3\) y \(nodiv3\), y un suceso
correspondiente a una observación \(A=\{1,2,4,6\}\).

Si observamos el suceso \(A\), esto es, si sabemos que ha salido \(1\) ó
\(2\) ó \(6\) nos preguntamos si es \textbf{más versosímil} que el
lanzamiento haya dado un resultado divisible por tres (\(div3\)) o no
divisible por tres (\(nodiv3\)).

Adviértase que \textbf{no nos preguntamos si es más probable} que el
lanzamiento haya sido \(div3\) o \(nodiv3\), pues ello supondría conocer
\(P(div3/A)\) y \(P(nodiv3/A)\), \textbf{sino más verosímil}, esto es,
conociendo \(P(A/div3)\) y \(P(A/nodiv3)\).

En estos momentos \textbf{la distinción entre más probable y más
verosímil puede resultar artificiosa}, pues estamos utilizando ejemplos
muy sencillos. Sin embargo, con frecuencia, conocemos la probabilidad
condicionada en un sentido y no en el opuesto. Es más, \textbf{los
sucesos condicionantes los estamos considerando aleatorios. Sin embargo
esto no es estrictamente necesario, con lo que podría ni siquiera tener
sentido la búsqueda de las probabilidades de tales sucesos
condicionantes}.

    Calculemos la verosimilitud, considerando el dado bueno:

\[L(div3)\equiv P(A/div3)=\frac{P(\{1,2,4,6\}\cap\{3,6\})}{P(\{3,6\})}=\frac{^1/_6}{^1/_3}=\frac{1}{2}\]
\[L(nodiv3)\equiv P(A/nodiv3)=\frac{P(\{1,2,4,6\}\cap\{1,2,4,5\})}{P(\{1,2,4,5\})}=\frac{^1/_2}{^2/_3}=\frac{3}{4}\]

Tal y como era fácil intuir, es más verosímil que el resultado no sea
divisible por tres, ya que \(A\) se satisface con tres de cuatro sucesos
elementales (todos menos el \(6\)), mientras que si fuera divisible por
tres, \(A\) tan solo se satisface en uno de dos (el \(6\)).

Adviértase que \textbf{la verosimilitud NO es una probabilidad}, como
queda puesto de manifiesto sin más que ver que
\(P(A/div3)+P(A/nodiv3) > 1\). Casualmente podría sumar uno (prúebese,
por ejemplo, para \(A=\{1,2,6\}\)), pero seguiría sin ser una
probabilidad.


    % Add a bibliography block to the postdoc
    
    
    
    \end{document}
