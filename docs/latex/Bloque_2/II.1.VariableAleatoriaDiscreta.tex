
% Default to the notebook output style

    


% Inherit from the specified cell style.




    
\documentclass[11pt]{article}

    
    
    \usepackage[T1]{fontenc}
    % Nicer default font (+ math font) than Computer Modern for most use cases
    \usepackage{mathpazo}

    % Basic figure setup, for now with no caption control since it's done
    % automatically by Pandoc (which extracts ![](path) syntax from Markdown).
    \usepackage{graphicx}
    % We will generate all images so they have a width \maxwidth. This means
    % that they will get their normal width if they fit onto the page, but
    % are scaled down if they would overflow the margins.
    \makeatletter
    \def\maxwidth{\ifdim\Gin@nat@width>\linewidth\linewidth
    \else\Gin@nat@width\fi}
    \makeatother
    \let\Oldincludegraphics\includegraphics
    % Set max figure width to be 80% of text width, for now hardcoded.
    \renewcommand{\includegraphics}[1]{\Oldincludegraphics[width=.8\maxwidth]{#1}}
    % Ensure that by default, figures have no caption (until we provide a
    % proper Figure object with a Caption API and a way to capture that
    % in the conversion process - todo).
    \usepackage{caption}
    \DeclareCaptionLabelFormat{nolabel}{}
    \captionsetup{labelformat=nolabel}

    \usepackage{adjustbox} % Used to constrain images to a maximum size 
    \usepackage{xcolor} % Allow colors to be defined
    \usepackage{enumerate} % Needed for markdown enumerations to work
    \usepackage{geometry} % Used to adjust the document margins
    \usepackage{amsmath} % Equations
    \usepackage{amssymb} % Equations
    \usepackage{textcomp} % defines textquotesingle
    % Hack from http://tex.stackexchange.com/a/47451/13684:
    \AtBeginDocument{%
        \def\PYZsq{\textquotesingle}% Upright quotes in Pygmentized code
    }
    \usepackage{upquote} % Upright quotes for verbatim code
    \usepackage{eurosym} % defines \euro
    \usepackage[mathletters]{ucs} % Extended unicode (utf-8) support
    \usepackage[utf8x]{inputenc} % Allow utf-8 characters in the tex document
    \usepackage{fancyvrb} % verbatim replacement that allows latex
    \usepackage{grffile} % extends the file name processing of package graphics 
                         % to support a larger range 
    % The hyperref package gives us a pdf with properly built
    % internal navigation ('pdf bookmarks' for the table of contents,
    % internal cross-reference links, web links for URLs, etc.)
    \usepackage{hyperref}
    \usepackage{longtable} % longtable support required by pandoc >1.10
    \usepackage{booktabs}  % table support for pandoc > 1.12.2
    \usepackage[inline]{enumitem} % IRkernel/repr support (it uses the enumerate* environment)
    \usepackage[normalem]{ulem} % ulem is needed to support strikethroughs (\sout)
                                % normalem makes italics be italics, not underlines

    % AÑADIDO POR JUAN
    \usepackage{mathrsfs}   

    % Colors for the hyperref package
    \definecolor{urlcolor}{rgb}{0,.145,.698}
    \definecolor{linkcolor}{rgb}{.71,0.21,0.01}
    \definecolor{citecolor}{rgb}{.12,.54,.11}

    % ANSI colors
    \definecolor{ansi-black}{HTML}{3E424D}
    \definecolor{ansi-black-intense}{HTML}{282C36}
    \definecolor{ansi-red}{HTML}{E75C58}
    \definecolor{ansi-red-intense}{HTML}{B22B31}
    \definecolor{ansi-green}{HTML}{00A250}
    \definecolor{ansi-green-intense}{HTML}{007427}
    \definecolor{ansi-yellow}{HTML}{DDB62B}
    \definecolor{ansi-yellow-intense}{HTML}{B27D12}
    \definecolor{ansi-blue}{HTML}{208FFB}
    \definecolor{ansi-blue-intense}{HTML}{0065CA}
    \definecolor{ansi-magenta}{HTML}{D160C4}
    \definecolor{ansi-magenta-intense}{HTML}{A03196}
    \definecolor{ansi-cyan}{HTML}{60C6C8}
    \definecolor{ansi-cyan-intense}{HTML}{258F8F}
    \definecolor{ansi-white}{HTML}{C5C1B4}
    \definecolor{ansi-white-intense}{HTML}{A1A6B2}

    % commands and environments needed by pandoc snippets
    % extracted from the output of `pandoc -s`
    \providecommand{\tightlist}{%
      \setlength{\itemsep}{0pt}\setlength{\parskip}{0pt}}
    \DefineVerbatimEnvironment{Highlighting}{Verbatim}{commandchars=\\\{\}}
    % Add ',fontsize=\small' for more characters per line
    \newenvironment{Shaded}{}{}
    \newcommand{\KeywordTok}[1]{\textcolor[rgb]{0.00,0.44,0.13}{\textbf{{#1}}}}
    \newcommand{\DataTypeTok}[1]{\textcolor[rgb]{0.56,0.13,0.00}{{#1}}}
    \newcommand{\DecValTok}[1]{\textcolor[rgb]{0.25,0.63,0.44}{{#1}}}
    \newcommand{\BaseNTok}[1]{\textcolor[rgb]{0.25,0.63,0.44}{{#1}}}
    \newcommand{\FloatTok}[1]{\textcolor[rgb]{0.25,0.63,0.44}{{#1}}}
    \newcommand{\CharTok}[1]{\textcolor[rgb]{0.25,0.44,0.63}{{#1}}}
    \newcommand{\StringTok}[1]{\textcolor[rgb]{0.25,0.44,0.63}{{#1}}}
    \newcommand{\CommentTok}[1]{\textcolor[rgb]{0.38,0.63,0.69}{\textit{{#1}}}}
    \newcommand{\OtherTok}[1]{\textcolor[rgb]{0.00,0.44,0.13}{{#1}}}
    \newcommand{\AlertTok}[1]{\textcolor[rgb]{1.00,0.00,0.00}{\textbf{{#1}}}}
    \newcommand{\FunctionTok}[1]{\textcolor[rgb]{0.02,0.16,0.49}{{#1}}}
    \newcommand{\RegionMarkerTok}[1]{{#1}}
    \newcommand{\ErrorTok}[1]{\textcolor[rgb]{1.00,0.00,0.00}{\textbf{{#1}}}}
    \newcommand{\NormalTok}[1]{{#1}}
    
    % Additional commands for more recent versions of Pandoc
    \newcommand{\ConstantTok}[1]{\textcolor[rgb]{0.53,0.00,0.00}{{#1}}}
    \newcommand{\SpecialCharTok}[1]{\textcolor[rgb]{0.25,0.44,0.63}{{#1}}}
    \newcommand{\VerbatimStringTok}[1]{\textcolor[rgb]{0.25,0.44,0.63}{{#1}}}
    \newcommand{\SpecialStringTok}[1]{\textcolor[rgb]{0.73,0.40,0.53}{{#1}}}
    \newcommand{\ImportTok}[1]{{#1}}
    \newcommand{\DocumentationTok}[1]{\textcolor[rgb]{0.73,0.13,0.13}{\textit{{#1}}}}
    \newcommand{\AnnotationTok}[1]{\textcolor[rgb]{0.38,0.63,0.69}{\textbf{\textit{{#1}}}}}
    \newcommand{\CommentVarTok}[1]{\textcolor[rgb]{0.38,0.63,0.69}{\textbf{\textit{{#1}}}}}
    \newcommand{\VariableTok}[1]{\textcolor[rgb]{0.10,0.09,0.49}{{#1}}}
    \newcommand{\ControlFlowTok}[1]{\textcolor[rgb]{0.00,0.44,0.13}{\textbf{{#1}}}}
    \newcommand{\OperatorTok}[1]{\textcolor[rgb]{0.40,0.40,0.40}{{#1}}}
    \newcommand{\BuiltInTok}[1]{{#1}}
    \newcommand{\ExtensionTok}[1]{{#1}}
    \newcommand{\PreprocessorTok}[1]{\textcolor[rgb]{0.74,0.48,0.00}{{#1}}}
    \newcommand{\AttributeTok}[1]{\textcolor[rgb]{0.49,0.56,0.16}{{#1}}}
    \newcommand{\InformationTok}[1]{\textcolor[rgb]{0.38,0.63,0.69}{\textbf{\textit{{#1}}}}}
    \newcommand{\WarningTok}[1]{\textcolor[rgb]{0.38,0.63,0.69}{\textbf{\textit{{#1}}}}}
    
    
    % Define a nice break command that doesn't care if a line doesn't already
    % exist.
    \def\br{\hspace*{\fill} \\* }
    % Math Jax compatability definitions
    \def\gt{>}
    \def\lt{<}
    % Document parameters
    \title{II.1. Variable Aleatoria Discreta}
    
    
    % Pygments definitions
    
\makeatletter
\def\PY@reset{\let\PY@it=\relax \let\PY@bf=\relax%
    \let\PY@ul=\relax \let\PY@tc=\relax%
    \let\PY@bc=\relax \let\PY@ff=\relax}
\def\PY@tok#1{\csname PY@tok@#1\endcsname}
\def\PY@toks#1+{\ifx\relax#1\empty\else%
    \PY@tok{#1}\expandafter\PY@toks\fi}
\def\PY@do#1{\PY@bc{\PY@tc{\PY@ul{%
    \PY@it{\PY@bf{\PY@ff{#1}}}}}}}
\def\PY#1#2{\PY@reset\PY@toks#1+\relax+\PY@do{#2}}

\expandafter\def\csname PY@tok@w\endcsname{\def\PY@tc##1{\textcolor[rgb]{0.73,0.73,0.73}{##1}}}
\expandafter\def\csname PY@tok@c\endcsname{\let\PY@it=\textit\def\PY@tc##1{\textcolor[rgb]{0.25,0.50,0.50}{##1}}}
\expandafter\def\csname PY@tok@cp\endcsname{\def\PY@tc##1{\textcolor[rgb]{0.74,0.48,0.00}{##1}}}
\expandafter\def\csname PY@tok@k\endcsname{\let\PY@bf=\textbf\def\PY@tc##1{\textcolor[rgb]{0.00,0.50,0.00}{##1}}}
\expandafter\def\csname PY@tok@kp\endcsname{\def\PY@tc##1{\textcolor[rgb]{0.00,0.50,0.00}{##1}}}
\expandafter\def\csname PY@tok@kt\endcsname{\def\PY@tc##1{\textcolor[rgb]{0.69,0.00,0.25}{##1}}}
\expandafter\def\csname PY@tok@o\endcsname{\def\PY@tc##1{\textcolor[rgb]{0.40,0.40,0.40}{##1}}}
\expandafter\def\csname PY@tok@ow\endcsname{\let\PY@bf=\textbf\def\PY@tc##1{\textcolor[rgb]{0.67,0.13,1.00}{##1}}}
\expandafter\def\csname PY@tok@nb\endcsname{\def\PY@tc##1{\textcolor[rgb]{0.00,0.50,0.00}{##1}}}
\expandafter\def\csname PY@tok@nf\endcsname{\def\PY@tc##1{\textcolor[rgb]{0.00,0.00,1.00}{##1}}}
\expandafter\def\csname PY@tok@nc\endcsname{\let\PY@bf=\textbf\def\PY@tc##1{\textcolor[rgb]{0.00,0.00,1.00}{##1}}}
\expandafter\def\csname PY@tok@nn\endcsname{\let\PY@bf=\textbf\def\PY@tc##1{\textcolor[rgb]{0.00,0.00,1.00}{##1}}}
\expandafter\def\csname PY@tok@ne\endcsname{\let\PY@bf=\textbf\def\PY@tc##1{\textcolor[rgb]{0.82,0.25,0.23}{##1}}}
\expandafter\def\csname PY@tok@nv\endcsname{\def\PY@tc##1{\textcolor[rgb]{0.10,0.09,0.49}{##1}}}
\expandafter\def\csname PY@tok@no\endcsname{\def\PY@tc##1{\textcolor[rgb]{0.53,0.00,0.00}{##1}}}
\expandafter\def\csname PY@tok@nl\endcsname{\def\PY@tc##1{\textcolor[rgb]{0.63,0.63,0.00}{##1}}}
\expandafter\def\csname PY@tok@ni\endcsname{\let\PY@bf=\textbf\def\PY@tc##1{\textcolor[rgb]{0.60,0.60,0.60}{##1}}}
\expandafter\def\csname PY@tok@na\endcsname{\def\PY@tc##1{\textcolor[rgb]{0.49,0.56,0.16}{##1}}}
\expandafter\def\csname PY@tok@nt\endcsname{\let\PY@bf=\textbf\def\PY@tc##1{\textcolor[rgb]{0.00,0.50,0.00}{##1}}}
\expandafter\def\csname PY@tok@nd\endcsname{\def\PY@tc##1{\textcolor[rgb]{0.67,0.13,1.00}{##1}}}
\expandafter\def\csname PY@tok@s\endcsname{\def\PY@tc##1{\textcolor[rgb]{0.73,0.13,0.13}{##1}}}
\expandafter\def\csname PY@tok@sd\endcsname{\let\PY@it=\textit\def\PY@tc##1{\textcolor[rgb]{0.73,0.13,0.13}{##1}}}
\expandafter\def\csname PY@tok@si\endcsname{\let\PY@bf=\textbf\def\PY@tc##1{\textcolor[rgb]{0.73,0.40,0.53}{##1}}}
\expandafter\def\csname PY@tok@se\endcsname{\let\PY@bf=\textbf\def\PY@tc##1{\textcolor[rgb]{0.73,0.40,0.13}{##1}}}
\expandafter\def\csname PY@tok@sr\endcsname{\def\PY@tc##1{\textcolor[rgb]{0.73,0.40,0.53}{##1}}}
\expandafter\def\csname PY@tok@ss\endcsname{\def\PY@tc##1{\textcolor[rgb]{0.10,0.09,0.49}{##1}}}
\expandafter\def\csname PY@tok@sx\endcsname{\def\PY@tc##1{\textcolor[rgb]{0.00,0.50,0.00}{##1}}}
\expandafter\def\csname PY@tok@m\endcsname{\def\PY@tc##1{\textcolor[rgb]{0.40,0.40,0.40}{##1}}}
\expandafter\def\csname PY@tok@gh\endcsname{\let\PY@bf=\textbf\def\PY@tc##1{\textcolor[rgb]{0.00,0.00,0.50}{##1}}}
\expandafter\def\csname PY@tok@gu\endcsname{\let\PY@bf=\textbf\def\PY@tc##1{\textcolor[rgb]{0.50,0.00,0.50}{##1}}}
\expandafter\def\csname PY@tok@gd\endcsname{\def\PY@tc##1{\textcolor[rgb]{0.63,0.00,0.00}{##1}}}
\expandafter\def\csname PY@tok@gi\endcsname{\def\PY@tc##1{\textcolor[rgb]{0.00,0.63,0.00}{##1}}}
\expandafter\def\csname PY@tok@gr\endcsname{\def\PY@tc##1{\textcolor[rgb]{1.00,0.00,0.00}{##1}}}
\expandafter\def\csname PY@tok@ge\endcsname{\let\PY@it=\textit}
\expandafter\def\csname PY@tok@gs\endcsname{\let\PY@bf=\textbf}
\expandafter\def\csname PY@tok@gp\endcsname{\let\PY@bf=\textbf\def\PY@tc##1{\textcolor[rgb]{0.00,0.00,0.50}{##1}}}
\expandafter\def\csname PY@tok@go\endcsname{\def\PY@tc##1{\textcolor[rgb]{0.53,0.53,0.53}{##1}}}
\expandafter\def\csname PY@tok@gt\endcsname{\def\PY@tc##1{\textcolor[rgb]{0.00,0.27,0.87}{##1}}}
\expandafter\def\csname PY@tok@err\endcsname{\def\PY@bc##1{\setlength{\fboxsep}{0pt}\fcolorbox[rgb]{1.00,0.00,0.00}{1,1,1}{\strut ##1}}}
\expandafter\def\csname PY@tok@kc\endcsname{\let\PY@bf=\textbf\def\PY@tc##1{\textcolor[rgb]{0.00,0.50,0.00}{##1}}}
\expandafter\def\csname PY@tok@kd\endcsname{\let\PY@bf=\textbf\def\PY@tc##1{\textcolor[rgb]{0.00,0.50,0.00}{##1}}}
\expandafter\def\csname PY@tok@kn\endcsname{\let\PY@bf=\textbf\def\PY@tc##1{\textcolor[rgb]{0.00,0.50,0.00}{##1}}}
\expandafter\def\csname PY@tok@kr\endcsname{\let\PY@bf=\textbf\def\PY@tc##1{\textcolor[rgb]{0.00,0.50,0.00}{##1}}}
\expandafter\def\csname PY@tok@bp\endcsname{\def\PY@tc##1{\textcolor[rgb]{0.00,0.50,0.00}{##1}}}
\expandafter\def\csname PY@tok@fm\endcsname{\def\PY@tc##1{\textcolor[rgb]{0.00,0.00,1.00}{##1}}}
\expandafter\def\csname PY@tok@vc\endcsname{\def\PY@tc##1{\textcolor[rgb]{0.10,0.09,0.49}{##1}}}
\expandafter\def\csname PY@tok@vg\endcsname{\def\PY@tc##1{\textcolor[rgb]{0.10,0.09,0.49}{##1}}}
\expandafter\def\csname PY@tok@vi\endcsname{\def\PY@tc##1{\textcolor[rgb]{0.10,0.09,0.49}{##1}}}
\expandafter\def\csname PY@tok@vm\endcsname{\def\PY@tc##1{\textcolor[rgb]{0.10,0.09,0.49}{##1}}}
\expandafter\def\csname PY@tok@sa\endcsname{\def\PY@tc##1{\textcolor[rgb]{0.73,0.13,0.13}{##1}}}
\expandafter\def\csname PY@tok@sb\endcsname{\def\PY@tc##1{\textcolor[rgb]{0.73,0.13,0.13}{##1}}}
\expandafter\def\csname PY@tok@sc\endcsname{\def\PY@tc##1{\textcolor[rgb]{0.73,0.13,0.13}{##1}}}
\expandafter\def\csname PY@tok@dl\endcsname{\def\PY@tc##1{\textcolor[rgb]{0.73,0.13,0.13}{##1}}}
\expandafter\def\csname PY@tok@s2\endcsname{\def\PY@tc##1{\textcolor[rgb]{0.73,0.13,0.13}{##1}}}
\expandafter\def\csname PY@tok@sh\endcsname{\def\PY@tc##1{\textcolor[rgb]{0.73,0.13,0.13}{##1}}}
\expandafter\def\csname PY@tok@s1\endcsname{\def\PY@tc##1{\textcolor[rgb]{0.73,0.13,0.13}{##1}}}
\expandafter\def\csname PY@tok@mb\endcsname{\def\PY@tc##1{\textcolor[rgb]{0.40,0.40,0.40}{##1}}}
\expandafter\def\csname PY@tok@mf\endcsname{\def\PY@tc##1{\textcolor[rgb]{0.40,0.40,0.40}{##1}}}
\expandafter\def\csname PY@tok@mh\endcsname{\def\PY@tc##1{\textcolor[rgb]{0.40,0.40,0.40}{##1}}}
\expandafter\def\csname PY@tok@mi\endcsname{\def\PY@tc##1{\textcolor[rgb]{0.40,0.40,0.40}{##1}}}
\expandafter\def\csname PY@tok@il\endcsname{\def\PY@tc##1{\textcolor[rgb]{0.40,0.40,0.40}{##1}}}
\expandafter\def\csname PY@tok@mo\endcsname{\def\PY@tc##1{\textcolor[rgb]{0.40,0.40,0.40}{##1}}}
\expandafter\def\csname PY@tok@ch\endcsname{\let\PY@it=\textit\def\PY@tc##1{\textcolor[rgb]{0.25,0.50,0.50}{##1}}}
\expandafter\def\csname PY@tok@cm\endcsname{\let\PY@it=\textit\def\PY@tc##1{\textcolor[rgb]{0.25,0.50,0.50}{##1}}}
\expandafter\def\csname PY@tok@cpf\endcsname{\let\PY@it=\textit\def\PY@tc##1{\textcolor[rgb]{0.25,0.50,0.50}{##1}}}
\expandafter\def\csname PY@tok@c1\endcsname{\let\PY@it=\textit\def\PY@tc##1{\textcolor[rgb]{0.25,0.50,0.50}{##1}}}
\expandafter\def\csname PY@tok@cs\endcsname{\let\PY@it=\textit\def\PY@tc##1{\textcolor[rgb]{0.25,0.50,0.50}{##1}}}

\def\PYZbs{\char`\\}
\def\PYZus{\char`\_}
\def\PYZob{\char`\{}
\def\PYZcb{\char`\}}
\def\PYZca{\char`\^}
\def\PYZam{\char`\&}
\def\PYZlt{\char`\<}
\def\PYZgt{\char`\>}
\def\PYZsh{\char`\#}
\def\PYZpc{\char`\%}
\def\PYZdl{\char`\$}
\def\PYZhy{\char`\-}
\def\PYZsq{\char`\'}
\def\PYZdq{\char`\"}
\def\PYZti{\char`\~}
% for compatibility with earlier versions
\def\PYZat{@}
\def\PYZlb{[}
\def\PYZrb{]}
\makeatother


    % Exact colors from NB
    \definecolor{incolor}{rgb}{0.0, 0.0, 0.5}
    \definecolor{outcolor}{rgb}{0.545, 0.0, 0.0}



    
    % Prevent overflowing lines due to hard-to-break entities
    \sloppy 
    % Setup hyperref package
    \hypersetup{
      breaklinks=true,  % so long urls are correctly broken across lines
      colorlinks=true,
      urlcolor=urlcolor,
      linkcolor=linkcolor,
      citecolor=citecolor,
      }
    % Slightly bigger margins than the latex defaults
    
    \geometry{verbose,tmargin=1in,bmargin=1in,lmargin=1in,rmargin=1in}
    
    

    \begin{document}
    
    
    \maketitle
    
    

    
    \hypertarget{ii.1-variable-aleatoria-discreta}{%
\section*{Variable Aleatoria
Discreta}\label{ii.1-variable-aleatoria-discreta}}

Consideremos que la variable aleatoria \(X\) sólo toma valores sobre un
conjunto (finito o infinito) numerable de números. Los números
Naturales, \(\mathbb{N}\), los números Enteros, \(\mathbb{Z}\), y los
números Racionales, \(\mathbb{Q}\), son ejemplos de conjuntos infinitos
numerables. Por su parte, los seis números de las caras de un dado
constituyen un ejemplo de conjunto numerable finito.

Consideremos que \(x_i\) son los valores que puede tomar la variable
aleatoria, donde \(i\in I\) se refiere a un índice numerable que recorre
\(I\) para poder indexar todos los valores. Por tanto \(\{X=x_i\}\) se
refiere a un suceso elemental cualquiera.

    Los sucesos correspondientes a cada valor de una variable discreta
pueden tener probabilidades arbitrarias, que interpretaremos con el
símil mecánico de \emph{masa}. Podemos entonces definir una
\textbf{función de masa de probabilidad (fmp)} que asigna una masa de
probabilidad a cada suceso elemental representado por un valor posible
de la variable aleatoria discreta

\[p_X(x_i)=P(\{X=x_i\})\qquad i \in I\]

Considerando las propiedades de la probabilidad, se cumple:

\[0\leq p_X(x_i)\leq 1 \qquad \forall i \in I\]

\[\sum_{i \in I} p_X(x_i)=1\]

Por coherencia con la definición de variable aleatoria podemos
considerar que \(p_X(x)\) está definida en todo el eje real, tomando
\textbf{valor cero} fuera de los puntos \(x_i\) en los que hay masa de
probabilidad.

    Supongamos un dado trucado, cuya función de masa de probabilidad se
representa en la siguiente tabla:

\[
\begin{array}{c|cccccc|c}
  X & 1 & 2 & 3 & 4 & 5 & 6 &  \\ 
  \hline
  p_X(x_i) & \frac{1}{24} & \frac{5}{24} & \frac{5}{24} & \frac{3}{24} & \frac{4}{24} & \frac{6}{24} & \sum_i p_X(x_i)=1
 \end{array}
\]

Podemos fácilmente obtener las probabilidades de sucesos arbitrarios.
Por ejemplo, la probabilidad de que salga \(3\) ó \(5\) es:

\[
P(X=\{3,5\})=P(\{X=3\}\bigcup \{X=5\}) = p_X(3)+p_X(5)=\frac{9}{24}
\]

    \begin{center}
    \adjustimage{max size={0.9\linewidth}{0.9\paperheight}}{II.1.VariableAleatoriaDiscreta_files/II.1.VariableAleatoriaDiscreta_4_0.png}
    \end{center}
    { \hspace*{\fill} \\}
    
    Supongamos ahora, sin pérdida de generalidad, que el índice que numera
los posibles valores de la variable aleatoria \(X\) es el conjunto de
todos los números enteros \(\mathbb{Z}\), y que la ordenación es tal que
\(i<j \iff x_i < x_j\).

Definimos la \textbf{función de probabilidad acumuada (FPA)},
\(F_X(x_i)\), como sigue:

\[F_X(x_i)=\sum_{j=-\infty}^{i}p_X(x_j)\]

Adviértase que \(F_X(-\infty)=0\), \(F_X(\infty)=1\) y que \(F_X(x_i)\)
es una función \emph{no decreciente} en toda subsucesión de puntos
\(x_i\).

Por su parte, es fácil ver que \(p_X(x_i)=F_X(x_i)-F_X(x_{i-1})\)

    Calculemos la FPA para nuestro dado trucado, representándola en una
tabla junto con la fmp:

\[
\begin{array}{c|cccccccc|c}
  X & 0 & 1 & 2 & 3 & 4 & 5 & 6 & 7 & \\ 
  \hline
  p_X(x_i) & 0 & \frac{1}{24} & \frac{5}{24} & \frac{5}{24} & \frac{3}{24} & \frac{4}{24} & \frac{6}{24} & 0 & \sum_i p_X(x_i)=1 \\
  \hline
  F_X(x_i) & 0 & \frac{1}{24} & \frac{6}{24} & \frac{11}{24} & \frac{14}{24} & \frac{18}{24} & \frac{24}{24} & \frac{24}{24} & \begin{matrix}  
  F_X(x_i\leq 0)=0\\
  F_X(x_i\geq 6)=1 
 \end{matrix}
 \end{array}
\]

Donde, para completar la repesentación hemos añadido los valores \(0\) y
\(7\) que, lógicamente, corresponden a sucesos imposibles (como
cualquier otro valor distinto de los primeros seis números naturales).

    \begin{center}
    \adjustimage{max size={0.9\linewidth}{0.9\paperheight}}{II.1.VariableAleatoriaDiscreta_files/II.1.VariableAleatoriaDiscreta_7_0.png}
    \end{center}
    { \hspace*{\fill} \\}
    
    \hypertarget{probabilidad-condicionada}{%
\subsection*{Probabilidad condicionada}\label{probabilidad-condicionada}}

Considérese un suceso \(M\), de probabilidad \(P(M)\), del espacio
muestral sobre el que se ha definido la variable aleatoria discreta
\(X\). Definimos la función de masa de probabilidad de \(X\)
condicionada por \(M\) como sigue:

\[p_X(x_i/M) = \frac{P(\{X=x_i\}\bigcap M)}{P(M)} \]

Adviértase que \(\sum_i p_X(x_i/M) =1\)

Frecuentemente \(M\) podrá expresarse por medio de la variable aleatoria
\(X\), lo que simplifica el cálculo.

    Consideremos los sucesos \emph{Par} e \emph{Impar} en el dado trucado
que ya hemos visto y calculemos las funciones de masa de probabilidad
condicionadas \(p_X(x_i/Par)\) y \(p_X(x_i/Impar)\)

\[
\begin{array}{c|cccccc|c}
  X & 1 & 2 & 3 & 4 & 5 & 6 &  \\ 
  \hline
  p_X(x_i) & \frac{1}{24} & \frac{5}{24} & \frac{5}{24} & \frac{3}{24} & \frac{4}{24} & \frac{6}{24} & \sum_i p_X(x_i)=1 \\
  \hline
  P(Impar) & \frac{1}{24} & + & \frac{5}{24} & + & \frac{4}{24} &  & P(Impar) = \frac{10}{24} \\
  P(Par)  &  & \frac{5}{24} & + & \frac{3}{24} & + & \frac{6}{24} & P(Par) = \frac{14}{24}\\
  \hline
  p_X(x_i/Impar) & \frac{1}{10} & 0 & \frac{5}{10} & 0 & \frac{4}{10} & 0 & \sum_i p_X(x_i/Impar)=1 \\
  p_X(x_i/Par) & 0 & \frac{5}{14} & 0 & \frac{3}{14} & 0 & \frac{6}{14} & \sum_i p_X(x_i/Par)=1
 \end{array}
\]

    \begin{center}
    \adjustimage{max size={0.9\linewidth}{0.9\paperheight}}{II.1.VariableAleatoriaDiscreta_files/II.1.VariableAleatoriaDiscreta_11_0.png}
    \end{center}
    { \hspace*{\fill} \\}
    
    Podemos también definir la probabilidad de un suceso \(M\) condicionada
por un valor arbitrario \(x_i\) de la variable aleatoria \(X\). Para
ello no hay más que advertir que \(\{X=x_i\}\) es un suceso del espacio
muestral:

\[P(M/X=x_i)= \frac{P(M \bigcap \{X=x_i\})}{P(\{X=x_i\})}=\frac{P(M \bigcap \{X=x_i\})}{p_X(x_i)}\]

Haremos uso frecuente del abuso de notación
\(P(M/x_i)\equiv P(M/X=x_i)\).

Adviértase que \(P(M/x_i) \neq p_X(x_i/M)\), estando ambas expresiones
relacionadas mediante el Teorema de Bayes, como se verá más adelante.

    \hypertarget{teorema-de-la-probabilidad-total}{%
\subsection*{Teorema de la Probabilidad
Total}\label{teorema-de-la-probabilidad-total}}

Consideremos una partición del espacio muestral \[
M_j \in \mathscr{P}(\Omega) \, j=1\ldots N\iff  \begin{matrix}
  \bigcup_{j=1}^N M_j = \Omega &  \\
  M_j \cap M_k = \emptyset & j\neq k  
 \end{matrix}
\]

y que conocemos las funciones de masa de probabilidad de una variable
aleatoria \(X\) definida en el mismo condicionadas por todos los sucesos
de dicha partición \(p_X(x_i/M_j)\), así como las probabilidades de
todos los sucesos condicionantes.

La función de masa de probabilidad total (sin condicionar) de la
variable aleatoria \(X\) es

\[p_X(x_i) = \sum_{j=1}^N p_X(x_i/M_j)P(M_j)\]

    Adviértase que el teorema permite descomponer una función de masa de
probabilidad en \emph{mezcla} (\emph{mixture}) de funciones de masa de
probabilidad condicionadas, algo que resulta con frecuencia útil.

En el caso de tener un único suceso \(M\), consideramos que el espacio
muestral se particiona con él y con su suceso complementario
\(\overline M\), y que \(P(\overline M)=1-P(M)\), resultando:

\[p_X(x_i)=p_X(x_i/M)P(M)+p_X(x_i/\overline M)(1-P(M))\]

En el ejemplo del dado trucado podemos advertir fácilmente que

\[p_X(x_i) = p_X(x_i/Impar)P(Impar) + p_X(x_i/Par)P(Par)\]

    \begin{center}
    \adjustimage{max size={0.9\linewidth}{0.9\paperheight}}{II.1.VariableAleatoriaDiscreta_files/II.1.VariableAleatoriaDiscreta_15_0.png}
    \end{center}
    { \hspace*{\fill} \\}
    
    La probabilidad total también puede aplicarse de forma alternativa, si
lo que se conocen son las probabilidades un suceso arbitrario \(M\)
condicionadas por cada uno de los valores posibles de la variable
aleatoria \(X\), esto es, si se conocen \(P(M/x_i)\equiv P(M/X=x_i)\).

En tal caso, la probabilidad total del suceso \(M\) (sin condicionar) es

\[P(M) = \sum_{i=-\infty}^\infty P(M/x_i) p_X(x_i)\]

    \hypertarget{teorema-de-bayes}{%
\subsection*{Teorema de Bayes}\label{teorema-de-bayes}}

Partiendo de la definición de función de masa de probabilidad
condicionada es inmediato obtener el \textbf{Teorema de Bayes} (recordar
que \(P(M/x_i)\equiv P(M/X=x_i)\)):

\[p_X(x_i/M) = \frac{P(M/x_i)p_X(x_i)}{P(M)} \iff P(M/x_i) = \frac{p_X(x_i/M)P(M)}{p(x_i)}\]

Con frecuencia, los denominadores de las expresiones anteriores se
obtienen mediante el \emph{Teorema de la Probabilidad Total}, pues tan
sólo se conocen las probabilidades condicionadas:

\begin{align*}
p_X(x_i/M) &= \frac{P(M/x_i)p_X(x_i)}{\sum_{i=-\infty}^\infty P(M/x_i) p_X(x_i)} \\ 
P(M/x_i) &= \frac{p_X(x_i/M)P(M)}{p_X(x_i/M)P(M)+p_X(x_i/\overline M)(1-P(M))}
\end{align*}

    Apliquemos el Teorema de Bayes al ejemplo del dado trucado para obtener,
trivialmente, las probabilidades de los sucesos \emph{Par} e
\emph{Impar} condicionadas por los valores de la variable aleatoria:

\[
\begin{array}{c|cccccc|c}
  X & 1 & 2 & 3 & 4 & 5 & 6 &  \\ 
  \hline
  p_X(x_i) & \frac{1}{24} & \frac{5}{24} & \frac{5}{24} & \frac{3}{24} & \frac{4}{24} & \frac{6}{24} & \sum_i p_X(x_i)=1 \\
  \hline
  P(Impar) & \frac{1}{24} & + & \frac{5}{24} & + & \frac{4}{24} &  & P(Impar) = \frac{10}{24} \\
  P(Par)  &  & \frac{5}{24} & + & \frac{3}{24} & + & \frac{6}{24} & P(Par) = \frac{14}{24}\\
  \hline
  p_X(x_i/Impar) & \frac{1}{10} & 0 & \frac{5}{10} & 0 & \frac{4}{10} & 0 & \sum_i p_X(x_i/Impar)=1 \\
  p_X(x_i/Par) & 0 & \frac{5}{14} & 0 & \frac{3}{14} & 0 & \frac{6}{14} & \sum_i p_X(x_i/Par)=1\\
  \hline
  P(Impar/x_i) & 1 & 0 & 1 & 0 & 1 & 0 & \sum_i p_X(Impar/x_i)\neq 1\\
  P(Par/x_i) & 0 & 1 & 0 & 1 & 0 & 1 & \sum_i p_X(Par/x_i)\neq 1
 \end{array}
\]

    Podemos extender fácilmente el \textbf{Teorema de la Probabilidad Total}
y el \textbf{Teorema de Bayes} considerando que los sucesos
condicionantes \(M_j\) pueden modelarse con una variable aleatoria \(M\)
cuya función de masa de probabilidad \(p_M(m_i)\) conocemos. Las
funciones de masa de probabilidad condicionadas son \(p_{X/M}(x_i/m_j)\)
y \(p_{M/X}(m_i/x_j)\). En tal caso:

\begin{align*}
p_X(x_i) &= \sum_j p_{X/M}(x_i/m_j)p_M(m_j)\\
p_{M/X}(m_i/x_j) &= \frac{p_{X/M}(x_j/m_i)p_M(m_i)}{\sum_i p_{X/M}(x_j/m_i)p_M(m_i)}
\end{align*}

    \hypertarget{funciuxf3n-de-variable-aleatoria}{%
\subsection*{Función de variable
aleatoria}\label{funciuxf3n-de-variable-aleatoria}}

Consideremos una variable aleatoria discreta \(X\). Podemos aplicar
sobre la misma una función \(g\) que devolverá valores aleatorios, al
serlo también su variable independiente. Lo representamos como sigue:

\[Y = g(X)\]

Cada valor \(x_i\) de \(X\) se transformará por la función en un nuevo
valor \(y_i\) de \(Y\). Adviértase que valores distintos
\(x_i \neq x_j\) pueden dar lugar a valores iguales \(y_i = y_j\) con
tal que \(g(x_i)=g(x_j)\). Por tanto, definimos la función de masa de
probabilidad de \(Y\) como sigue:

\[p_Y(y_i) = \sum_j p_X(g^{-1}(y_i))
= \sum_{\substack{j \\ g(x_j) = y_i}}p_X(x_j)\]

Donde \(j\) indexa las raíces de la ecuación \(y_i=g(x_j)\).

    Consideremos, por ejemplo, la variable aleatoria \(X\) con la siguiente
fmp:

\[
\begin{array}{c|cccc|c}
X & -2 & -1 & 0 & 1\\
\hline
p_X(x_i) & \frac{1}{8} & \frac{1}{8} & \frac{1}{2} & \frac{1}{4} & \sum_i p_X(x_i) = 1
\end{array}
\]

Veamos cuál es la fmp de la variable aleatoria \(Y = X^2\):

\[
\begin{array}{c|ccc|c}
Y & 0 & 1 & 4\\
\hline
p_Y(y_i) & \frac{1}{2} & \frac{1}{4}+\frac{1}{8} & \frac{1}{8} & \sum_i p_Y(y_i) = 1
\end{array}
\]

Si se desean incluir otros valores de \(Y\) en la representación, por
ejemplo el \(2\) ó el \(3\), tendrán masa de probabilidad nulas.

    \hypertarget{esperanza-matemuxe1tica-y-valor-medio.-momentos}{%
\subsection*{Esperanza Matemática y Valor Medio.
Momentos}\label{esperanza-matemuxe1tica-y-valor-medio.-momentos}}

Se define la esperanza matemática de una función \(g(X)\) de una
variable aleatoria como sigue:

\[E(g(X))=\sum_i g(x_i)p_X(x_i)\]

Es fácil advertir que el operador \(E(·)\) es lineal, esto es que

\[E(a_1g_1(X)+a_2g_2(X))=a_1E(g_1(X)) + a_2E(g_2(X))\]

donde \(a_1\) y \(a_2\) son coeficiente constantes arbitrarios y \(g_1\)
y \(g_2\) son dos funciones. También es evidente que, si \(k\) es una
constante arbitraria

\[E(k) = k\]

    Si \(g\) es la función identidad obtenemos el \textbf{valor medio} de la
variable aleatoria \(X\): \[\eta_X=E(X)=\sum_i x_ip_X(x_i)\]

Es conveniente \textbf{reflexionar sobre la relación con la media
muestral}, advirtiendo que ambas son medidas de tendencia central.

Adviértase que, como una función de variable aleatoria define una nueva
variable aleatoria, \(Y=g(X)\), con su propia función de masa de
probabilidad, tenemos dos maneras de calcular la esperanza de una
función de variable aleatoria:

\[\eta_Y = E(Y)=\sum_i y_ip_Y(y_i) = \sum_i g(x_i)p_X(x_i)\]

    Consideremos que la variable aleatoria \(X\) tiene valor medio
\(\eta_X\). Calculemos cuál sera el valor medio de la variable aleatoria
\(Y = aX+b\), donde \(a\) y \(b\) son coeficientes constantes.

Aplicando linealidad:

\[E(aX+b) = aE(X)+b = a\eta_X + b\]

Son interesantes los casos particulares:

\begin{itemize}
\tightlist
\item
  Sumar constante, \(a=1\): \(E(X+b) = \eta_X + b\)
\item
  Multiplicar por constante \(a\), \(b=0\): \(E(aX) = a\eta_X\)
\end{itemize}

    \hypertarget{momentos-de-segundo-orden-valor-cuadruxe1tico-medio-y-varianza}{%
\subsubsection*{Momentos de Segundo Orden: Valor Cuadrático Medio y
Varianza}\label{momentos-de-segundo-orden-valor-cuadruxe1tico-medio-y-varianza}}

El \textbf{valor cuadrático medio}, o momento no centrado de segundo
orden, de la variable aleatoria \(X\) se define:

\[E(X^2)=\sum_i x_i^2p_X(x_i) \geq 0\]

El valor cuadrático medio es una medida de dispersión respecto del
origen, a la que no afecta el signo positivo o negativo de las
excursiones (al estar elevada al cuadrado) sino solo la amplitud de las
mismas. Por ello siempre es una magnitud positiva, salvo en el caso
trivial de que toda la masa de probabilidad se localice en el origen (el
valor es determinista e igual a cero).

Es conveniente reflexionar sobre su relación con el estadístico muestral
correspondiente.

    Consideremos que la variable aleatoria \(X\) tiene valor medio
\(\eta_X\) y valor cuadrático medio \(E(X^2)\). Calculemos cuál sera el
valor cuadrático medio de la variable aleatoria \(Y = aX+b\), donde
\(a\) y \(b\) son coeficientes constantes.

Desarrollando el cuadrado y aplicando linealidad:

\[E((aX+b)^2) = a^2E(X^2)+(2ab\eta_X+b^2)\]

Es interesante el caso de multiplicar por una constante \(a\), \(b=0\):

\[E((aX)^2) = a^2E(X^2)\]

    La \textbf{varianza} o momento centrado de segundo orden, se define:

\[\sigma_X^2 \equiv Var(X) = E((X-\eta_X)^2) \geq 0\]

La varianza es una medida de dispersión de los valores de la variable
aleatoria en relación a su valor medio. De nuevo, al estar elevada al
cuadrado, no importa el signo de las excursiones, sino sólo su amplitud
y, por ello, siempre es una magnitud positiva, o trivialmente nula si
toda la masa de probabilidad está localizada en el mismo punto (valor
determinista).

La raíz cuadrada de la varianza se denomina \textbf{desviación típica},
\(\sigma_X\).

Desarrollando la definición y aplicando la propiedad de linealidad, se
obtiene:

\[\sigma_X^2 = E(X^2) - \eta_X^2 \implies E(X^2) \geq \sigma_X^2\]

Si la variable aleatoria tiene \textbf{media nula}:
\(E(X^2) = \sigma_X^2\)

    Consideremos ahora que la variable aleatoria \(X\) tiene valor medio
\(\eta_X\), valor cuadrático medio \(E(X^2)\) y varianza
\(\sigma_X^2\equiv Var(X)\). Calculemos cuál sera la varianza de la
variable aleatoria \(Y = aX+b\), donde \(a\) y \(b\) son coeficientes
constantes.

Aplicando la relación entre valor cuadrático medio, varianza y cuadrado
de la media:

\[Var(aX+b) = E((aX+b)^2)-E^2(aX+b)\]

Utilizando las expresiones que ya tenemos:

\begin{align*}
E((aX+b)^2) &= a^2E(X^2)+2ab\eta_X+b^2\\
E^2(aX+b) &= (a\eta_X+b)^2 = a^2\eta_X^2+2ab\eta_X+b^2
\end{align*}

Resultando la varianza:

\begin{align*}
Var(aX+b) &= a^2E(X^2)-a^2\eta_X^2=a^2(E(X^2)-\eta_X^2)\\
&=a^2Var(X)
\end{align*}

    \begin{Verbatim}[commandchars=\\\{\}]
Dado trucado:
	 Media: 3.916666666666667
	 Varianza: 2.576388888888884
	 Valor Cuadrático Medio: 17.916666666666664

    \end{Verbatim}

    \hypertarget{tipificaciuxf3n-de-una-variable-aleatoria}{%
\subsubsection*{Tipificación de una Variable
Aleatoria}\label{tipificaciuxf3n-de-una-variable-aleatoria}}

Consideremos que una variable aleatoria \(X\) tiene valor medio
\(\eta_X\) y varianza \(\sigma_X\). \textbf{Tipificar} la variable
quiere decir buscar otra equivalente, pero con \textbf{media nula} y
\textbf{varianza unidad}.

La variable tipificada \(\hat X\) la obtenemos aplicando la siguiente
función:

\[\hat X = \frac{X-\eta_X}{\sigma_X}\]

De lo visto anteriormente, es inmediato advertir que:

\begin{align*}
\eta_{\hat X}= E(\hat X) &= \frac{1}{\sigma_X}\eta_X-\frac{\eta_X}{\sigma_X}&=0 \\
\sigma_{\hat X}^2 = Var(\hat X) &=\frac{1}{\sigma_X^2}\sigma_X^2 &= 1 
\end{align*}

    \hypertarget{momentos-de-orden-superior}{%
\subsubsection*{Momentos de orden
superior}\label{momentos-de-orden-superior}}

El momento no centrado de orden \(n\) se define como:

\[m_n =E(X^n)=\sum_i x_i^np_X(x_i)\]

El momento centrado de orden \(n\):

\[\mu_n =E((X-\eta_X)^n)=\sum_i (x_i-\eta_X)^np_X(x_i)\]

Y el momento tipificado de orden \(n\):

\[\gamma_n =E(\left(\frac{X-\eta_X}{\sigma_X}\right)^n)=\sum_i \left(\frac{x_i-\eta_X}{\sigma_X}\right)^np_X(x_i)\]

El momento tipificado de orden 3 corresponde al coeficiente de
asimetría, y el de orden 4 al coeficiente de curtosis. Es conveniente
reflexionar sobre su relación con sus equivalentes muestrales.

    \hypertarget{esperanza-matemuxe1tica-condicionada}{%
\subsubsection*{Esperanza Matemática
Condicionada}\label{esperanza-matemuxe1tica-condicionada}}

Podemos definir la esperanza matemática de una función \(g\) de la
variable aleatoria \(X\) condicionada por el suceso \(M\) como sigue:

\[E(g(X)/M)=\sum_i g(x_i)p_X(x_i/M)\]

Podemos extender con la esperanza condicionada las propiedades y
definiciones vistas anteriormente. Por ejemplo:

\begin{align*}
\eta_{X/M}=E(X/M)&=\sum_i x_ip_X(x_i/M)\\
E(X^2/M)&=\sum_i x_i^2p_X(x_i/M)\\
\sigma_{X/M}^2 = Var(X/M) &= \sum_i (x_i-\eta_{X/M})^2p_X(x_i/M) = E(X^2/M)-\eta_{X/M}^2
\end{align*}

    \begin{Verbatim}[commandchars=\\\{\}]
Dado trucado - Estadísticos condicionados por ser Impar:
	 Media condicionada por Impar: 3.6000000000000005
	 Varianza condicionada por Impar: 1.639999999999997
	 Valor Cuadrático Medio condicionado por Impar: 14.600000000000001



Dado trucado - Estadísticos condicionados por ser Par:
	 Media condicionada por Par: 4.142857142857142
	 Varianza condicionada por Par: 3.122448979591841
	 Valor Cuadrático Medio condicionado por Par: 20.285714285714285

    \end{Verbatim}


    % Add a bibliography block to the postdoc
    
    
    
    \end{document}
