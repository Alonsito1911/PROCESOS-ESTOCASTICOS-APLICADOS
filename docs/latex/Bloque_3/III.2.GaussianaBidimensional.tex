


    




    
\documentclass[11pt]{article}

    
    \usepackage[breakable]{tcolorbox}
    \tcbset{nobeforeafter} % prevents tcolorboxes being placing in paragraphs
    \usepackage{float}
    \floatplacement{figure}{H} % forces figures to be placed at the correct location
    
    \usepackage[T1]{fontenc}
    % Nicer default font (+ math font) than Computer Modern for most use cases
    \usepackage{mathpazo}

    % Basic figure setup, for now with no caption control since it's done
    % automatically by Pandoc (which extracts ![](path) syntax from Markdown).
    \usepackage{graphicx}
    % We will generate all images so they have a width \maxwidth. This means
    % that they will get their normal width if they fit onto the page, but
    % are scaled down if they would overflow the margins.
    \makeatletter
    \def\maxwidth{\ifdim\Gin@nat@width>\linewidth\linewidth
    \else\Gin@nat@width\fi}
    \makeatother
    \let\Oldincludegraphics\includegraphics
    % Set max figure width to be 80% of text width, for now hardcoded.
    \renewcommand{\includegraphics}[1]{\Oldincludegraphics[width=.8\maxwidth]{#1}}
    % Ensure that by default, figures have no caption (until we provide a
    % proper Figure object with a Caption API and a way to capture that
    % in the conversion process - todo).
    \usepackage{caption}
    \DeclareCaptionLabelFormat{nolabel}{}
    \captionsetup{labelformat=nolabel}

    \usepackage{adjustbox} % Used to constrain images to a maximum size 
    \usepackage{xcolor} % Allow colors to be defined
    \usepackage{enumerate} % Needed for markdown enumerations to work
    \usepackage{geometry} % Used to adjust the document margins
    \usepackage{amsmath} % Equations
    \usepackage{amssymb} % Equations
    \usepackage{textcomp} % defines textquotesingle
    % Hack from http://tex.stackexchange.com/a/47451/13684:
    \AtBeginDocument{%
        \def\PYZsq{\textquotesingle}% Upright quotes in Pygmentized code
    }
    \usepackage{upquote} % Upright quotes for verbatim code
    \usepackage{eurosym} % defines \euro
    \usepackage[mathletters]{ucs} % Extended unicode (utf-8) support
    \usepackage[utf8x]{inputenc} % Allow utf-8 characters in the tex document
    \usepackage{fancyvrb} % verbatim replacement that allows latex
    \usepackage{grffile} % extends the file name processing of package graphics 
                         % to support a larger range 
    % The hyperref package gives us a pdf with properly built
    % internal navigation ('pdf bookmarks' for the table of contents,
    % internal cross-reference links, web links for URLs, etc.)
    \usepackage{hyperref}
    \usepackage{longtable} % longtable support required by pandoc >1.10
    \usepackage{booktabs}  % table support for pandoc > 1.12.2
    \usepackage[inline]{enumitem} % IRkernel/repr support (it uses the enumerate* environment)
    \usepackage[normalem]{ulem} % ulem is needed to support strikethroughs (\sout)
                                % normalem makes italics be italics, not underlines
    \usepackage{mathrsfs}
    

    
    % Colors for the hyperref package
    \definecolor{urlcolor}{rgb}{0,.145,.698}
    \definecolor{linkcolor}{rgb}{.71,0.21,0.01}
    \definecolor{citecolor}{rgb}{.12,.54,.11}

    % ANSI colors
    \definecolor{ansi-black}{HTML}{3E424D}
    \definecolor{ansi-black-intense}{HTML}{282C36}
    \definecolor{ansi-red}{HTML}{E75C58}
    \definecolor{ansi-red-intense}{HTML}{B22B31}
    \definecolor{ansi-green}{HTML}{00A250}
    \definecolor{ansi-green-intense}{HTML}{007427}
    \definecolor{ansi-yellow}{HTML}{DDB62B}
    \definecolor{ansi-yellow-intense}{HTML}{B27D12}
    \definecolor{ansi-blue}{HTML}{208FFB}
    \definecolor{ansi-blue-intense}{HTML}{0065CA}
    \definecolor{ansi-magenta}{HTML}{D160C4}
    \definecolor{ansi-magenta-intense}{HTML}{A03196}
    \definecolor{ansi-cyan}{HTML}{60C6C8}
    \definecolor{ansi-cyan-intense}{HTML}{258F8F}
    \definecolor{ansi-white}{HTML}{C5C1B4}
    \definecolor{ansi-white-intense}{HTML}{A1A6B2}
    \definecolor{ansi-default-inverse-fg}{HTML}{FFFFFF}
    \definecolor{ansi-default-inverse-bg}{HTML}{000000}

    % commands and environments needed by pandoc snippets
    % extracted from the output of `pandoc -s`
    \providecommand{\tightlist}{%
      \setlength{\itemsep}{0pt}\setlength{\parskip}{0pt}}
    \DefineVerbatimEnvironment{Highlighting}{Verbatim}{commandchars=\\\{\}}
    % Add ',fontsize=\small' for more characters per line
    \newenvironment{Shaded}{}{}
    \newcommand{\KeywordTok}[1]{\textcolor[rgb]{0.00,0.44,0.13}{\textbf{{#1}}}}
    \newcommand{\DataTypeTok}[1]{\textcolor[rgb]{0.56,0.13,0.00}{{#1}}}
    \newcommand{\DecValTok}[1]{\textcolor[rgb]{0.25,0.63,0.44}{{#1}}}
    \newcommand{\BaseNTok}[1]{\textcolor[rgb]{0.25,0.63,0.44}{{#1}}}
    \newcommand{\FloatTok}[1]{\textcolor[rgb]{0.25,0.63,0.44}{{#1}}}
    \newcommand{\CharTok}[1]{\textcolor[rgb]{0.25,0.44,0.63}{{#1}}}
    \newcommand{\StringTok}[1]{\textcolor[rgb]{0.25,0.44,0.63}{{#1}}}
    \newcommand{\CommentTok}[1]{\textcolor[rgb]{0.38,0.63,0.69}{\textit{{#1}}}}
    \newcommand{\OtherTok}[1]{\textcolor[rgb]{0.00,0.44,0.13}{{#1}}}
    \newcommand{\AlertTok}[1]{\textcolor[rgb]{1.00,0.00,0.00}{\textbf{{#1}}}}
    \newcommand{\FunctionTok}[1]{\textcolor[rgb]{0.02,0.16,0.49}{{#1}}}
    \newcommand{\RegionMarkerTok}[1]{{#1}}
    \newcommand{\ErrorTok}[1]{\textcolor[rgb]{1.00,0.00,0.00}{\textbf{{#1}}}}
    \newcommand{\NormalTok}[1]{{#1}}
    
    % Additional commands for more recent versions of Pandoc
    \newcommand{\ConstantTok}[1]{\textcolor[rgb]{0.53,0.00,0.00}{{#1}}}
    \newcommand{\SpecialCharTok}[1]{\textcolor[rgb]{0.25,0.44,0.63}{{#1}}}
    \newcommand{\VerbatimStringTok}[1]{\textcolor[rgb]{0.25,0.44,0.63}{{#1}}}
    \newcommand{\SpecialStringTok}[1]{\textcolor[rgb]{0.73,0.40,0.53}{{#1}}}
    \newcommand{\ImportTok}[1]{{#1}}
    \newcommand{\DocumentationTok}[1]{\textcolor[rgb]{0.73,0.13,0.13}{\textit{{#1}}}}
    \newcommand{\AnnotationTok}[1]{\textcolor[rgb]{0.38,0.63,0.69}{\textbf{\textit{{#1}}}}}
    \newcommand{\CommentVarTok}[1]{\textcolor[rgb]{0.38,0.63,0.69}{\textbf{\textit{{#1}}}}}
    \newcommand{\VariableTok}[1]{\textcolor[rgb]{0.10,0.09,0.49}{{#1}}}
    \newcommand{\ControlFlowTok}[1]{\textcolor[rgb]{0.00,0.44,0.13}{\textbf{{#1}}}}
    \newcommand{\OperatorTok}[1]{\textcolor[rgb]{0.40,0.40,0.40}{{#1}}}
    \newcommand{\BuiltInTok}[1]{{#1}}
    \newcommand{\ExtensionTok}[1]{{#1}}
    \newcommand{\PreprocessorTok}[1]{\textcolor[rgb]{0.74,0.48,0.00}{{#1}}}
    \newcommand{\AttributeTok}[1]{\textcolor[rgb]{0.49,0.56,0.16}{{#1}}}
    \newcommand{\InformationTok}[1]{\textcolor[rgb]{0.38,0.63,0.69}{\textbf{\textit{{#1}}}}}
    \newcommand{\WarningTok}[1]{\textcolor[rgb]{0.38,0.63,0.69}{\textbf{\textit{{#1}}}}}
    
    
    % Define a nice break command that doesn't care if a line doesn't already
    % exist.
    \def\br{\hspace*{\fill} \\* }
    % Math Jax compatibility definitions
    \def\gt{>}
    \def\lt{<}
    \let\Oldtex\TeX
    \let\Oldlatex\LaTeX
    \renewcommand{\TeX}{\textrm{\Oldtex}}
    \renewcommand{\LaTeX}{\textrm{\Oldlatex}}
    % Document parameters
    % Document title
    \title{III.2 Gaussiana Bidimensional}
    
    
    
    
    
% Pygments definitions
\makeatletter
\def\PY@reset{\let\PY@it=\relax \let\PY@bf=\relax%
    \let\PY@ul=\relax \let\PY@tc=\relax%
    \let\PY@bc=\relax \let\PY@ff=\relax}
\def\PY@tok#1{\csname PY@tok@#1\endcsname}
\def\PY@toks#1+{\ifx\relax#1\empty\else%
    \PY@tok{#1}\expandafter\PY@toks\fi}
\def\PY@do#1{\PY@bc{\PY@tc{\PY@ul{%
    \PY@it{\PY@bf{\PY@ff{#1}}}}}}}
\def\PY#1#2{\PY@reset\PY@toks#1+\relax+\PY@do{#2}}

\expandafter\def\csname PY@tok@w\endcsname{\def\PY@tc##1{\textcolor[rgb]{0.73,0.73,0.73}{##1}}}
\expandafter\def\csname PY@tok@c\endcsname{\let\PY@it=\textit\def\PY@tc##1{\textcolor[rgb]{0.25,0.50,0.50}{##1}}}
\expandafter\def\csname PY@tok@cp\endcsname{\def\PY@tc##1{\textcolor[rgb]{0.74,0.48,0.00}{##1}}}
\expandafter\def\csname PY@tok@k\endcsname{\let\PY@bf=\textbf\def\PY@tc##1{\textcolor[rgb]{0.00,0.50,0.00}{##1}}}
\expandafter\def\csname PY@tok@kp\endcsname{\def\PY@tc##1{\textcolor[rgb]{0.00,0.50,0.00}{##1}}}
\expandafter\def\csname PY@tok@kt\endcsname{\def\PY@tc##1{\textcolor[rgb]{0.69,0.00,0.25}{##1}}}
\expandafter\def\csname PY@tok@o\endcsname{\def\PY@tc##1{\textcolor[rgb]{0.40,0.40,0.40}{##1}}}
\expandafter\def\csname PY@tok@ow\endcsname{\let\PY@bf=\textbf\def\PY@tc##1{\textcolor[rgb]{0.67,0.13,1.00}{##1}}}
\expandafter\def\csname PY@tok@nb\endcsname{\def\PY@tc##1{\textcolor[rgb]{0.00,0.50,0.00}{##1}}}
\expandafter\def\csname PY@tok@nf\endcsname{\def\PY@tc##1{\textcolor[rgb]{0.00,0.00,1.00}{##1}}}
\expandafter\def\csname PY@tok@nc\endcsname{\let\PY@bf=\textbf\def\PY@tc##1{\textcolor[rgb]{0.00,0.00,1.00}{##1}}}
\expandafter\def\csname PY@tok@nn\endcsname{\let\PY@bf=\textbf\def\PY@tc##1{\textcolor[rgb]{0.00,0.00,1.00}{##1}}}
\expandafter\def\csname PY@tok@ne\endcsname{\let\PY@bf=\textbf\def\PY@tc##1{\textcolor[rgb]{0.82,0.25,0.23}{##1}}}
\expandafter\def\csname PY@tok@nv\endcsname{\def\PY@tc##1{\textcolor[rgb]{0.10,0.09,0.49}{##1}}}
\expandafter\def\csname PY@tok@no\endcsname{\def\PY@tc##1{\textcolor[rgb]{0.53,0.00,0.00}{##1}}}
\expandafter\def\csname PY@tok@nl\endcsname{\def\PY@tc##1{\textcolor[rgb]{0.63,0.63,0.00}{##1}}}
\expandafter\def\csname PY@tok@ni\endcsname{\let\PY@bf=\textbf\def\PY@tc##1{\textcolor[rgb]{0.60,0.60,0.60}{##1}}}
\expandafter\def\csname PY@tok@na\endcsname{\def\PY@tc##1{\textcolor[rgb]{0.49,0.56,0.16}{##1}}}
\expandafter\def\csname PY@tok@nt\endcsname{\let\PY@bf=\textbf\def\PY@tc##1{\textcolor[rgb]{0.00,0.50,0.00}{##1}}}
\expandafter\def\csname PY@tok@nd\endcsname{\def\PY@tc##1{\textcolor[rgb]{0.67,0.13,1.00}{##1}}}
\expandafter\def\csname PY@tok@s\endcsname{\def\PY@tc##1{\textcolor[rgb]{0.73,0.13,0.13}{##1}}}
\expandafter\def\csname PY@tok@sd\endcsname{\let\PY@it=\textit\def\PY@tc##1{\textcolor[rgb]{0.73,0.13,0.13}{##1}}}
\expandafter\def\csname PY@tok@si\endcsname{\let\PY@bf=\textbf\def\PY@tc##1{\textcolor[rgb]{0.73,0.40,0.53}{##1}}}
\expandafter\def\csname PY@tok@se\endcsname{\let\PY@bf=\textbf\def\PY@tc##1{\textcolor[rgb]{0.73,0.40,0.13}{##1}}}
\expandafter\def\csname PY@tok@sr\endcsname{\def\PY@tc##1{\textcolor[rgb]{0.73,0.40,0.53}{##1}}}
\expandafter\def\csname PY@tok@ss\endcsname{\def\PY@tc##1{\textcolor[rgb]{0.10,0.09,0.49}{##1}}}
\expandafter\def\csname PY@tok@sx\endcsname{\def\PY@tc##1{\textcolor[rgb]{0.00,0.50,0.00}{##1}}}
\expandafter\def\csname PY@tok@m\endcsname{\def\PY@tc##1{\textcolor[rgb]{0.40,0.40,0.40}{##1}}}
\expandafter\def\csname PY@tok@gh\endcsname{\let\PY@bf=\textbf\def\PY@tc##1{\textcolor[rgb]{0.00,0.00,0.50}{##1}}}
\expandafter\def\csname PY@tok@gu\endcsname{\let\PY@bf=\textbf\def\PY@tc##1{\textcolor[rgb]{0.50,0.00,0.50}{##1}}}
\expandafter\def\csname PY@tok@gd\endcsname{\def\PY@tc##1{\textcolor[rgb]{0.63,0.00,0.00}{##1}}}
\expandafter\def\csname PY@tok@gi\endcsname{\def\PY@tc##1{\textcolor[rgb]{0.00,0.63,0.00}{##1}}}
\expandafter\def\csname PY@tok@gr\endcsname{\def\PY@tc##1{\textcolor[rgb]{1.00,0.00,0.00}{##1}}}
\expandafter\def\csname PY@tok@ge\endcsname{\let\PY@it=\textit}
\expandafter\def\csname PY@tok@gs\endcsname{\let\PY@bf=\textbf}
\expandafter\def\csname PY@tok@gp\endcsname{\let\PY@bf=\textbf\def\PY@tc##1{\textcolor[rgb]{0.00,0.00,0.50}{##1}}}
\expandafter\def\csname PY@tok@go\endcsname{\def\PY@tc##1{\textcolor[rgb]{0.53,0.53,0.53}{##1}}}
\expandafter\def\csname PY@tok@gt\endcsname{\def\PY@tc##1{\textcolor[rgb]{0.00,0.27,0.87}{##1}}}
\expandafter\def\csname PY@tok@err\endcsname{\def\PY@bc##1{\setlength{\fboxsep}{0pt}\fcolorbox[rgb]{1.00,0.00,0.00}{1,1,1}{\strut ##1}}}
\expandafter\def\csname PY@tok@kc\endcsname{\let\PY@bf=\textbf\def\PY@tc##1{\textcolor[rgb]{0.00,0.50,0.00}{##1}}}
\expandafter\def\csname PY@tok@kd\endcsname{\let\PY@bf=\textbf\def\PY@tc##1{\textcolor[rgb]{0.00,0.50,0.00}{##1}}}
\expandafter\def\csname PY@tok@kn\endcsname{\let\PY@bf=\textbf\def\PY@tc##1{\textcolor[rgb]{0.00,0.50,0.00}{##1}}}
\expandafter\def\csname PY@tok@kr\endcsname{\let\PY@bf=\textbf\def\PY@tc##1{\textcolor[rgb]{0.00,0.50,0.00}{##1}}}
\expandafter\def\csname PY@tok@bp\endcsname{\def\PY@tc##1{\textcolor[rgb]{0.00,0.50,0.00}{##1}}}
\expandafter\def\csname PY@tok@fm\endcsname{\def\PY@tc##1{\textcolor[rgb]{0.00,0.00,1.00}{##1}}}
\expandafter\def\csname PY@tok@vc\endcsname{\def\PY@tc##1{\textcolor[rgb]{0.10,0.09,0.49}{##1}}}
\expandafter\def\csname PY@tok@vg\endcsname{\def\PY@tc##1{\textcolor[rgb]{0.10,0.09,0.49}{##1}}}
\expandafter\def\csname PY@tok@vi\endcsname{\def\PY@tc##1{\textcolor[rgb]{0.10,0.09,0.49}{##1}}}
\expandafter\def\csname PY@tok@vm\endcsname{\def\PY@tc##1{\textcolor[rgb]{0.10,0.09,0.49}{##1}}}
\expandafter\def\csname PY@tok@sa\endcsname{\def\PY@tc##1{\textcolor[rgb]{0.73,0.13,0.13}{##1}}}
\expandafter\def\csname PY@tok@sb\endcsname{\def\PY@tc##1{\textcolor[rgb]{0.73,0.13,0.13}{##1}}}
\expandafter\def\csname PY@tok@sc\endcsname{\def\PY@tc##1{\textcolor[rgb]{0.73,0.13,0.13}{##1}}}
\expandafter\def\csname PY@tok@dl\endcsname{\def\PY@tc##1{\textcolor[rgb]{0.73,0.13,0.13}{##1}}}
\expandafter\def\csname PY@tok@s2\endcsname{\def\PY@tc##1{\textcolor[rgb]{0.73,0.13,0.13}{##1}}}
\expandafter\def\csname PY@tok@sh\endcsname{\def\PY@tc##1{\textcolor[rgb]{0.73,0.13,0.13}{##1}}}
\expandafter\def\csname PY@tok@s1\endcsname{\def\PY@tc##1{\textcolor[rgb]{0.73,0.13,0.13}{##1}}}
\expandafter\def\csname PY@tok@mb\endcsname{\def\PY@tc##1{\textcolor[rgb]{0.40,0.40,0.40}{##1}}}
\expandafter\def\csname PY@tok@mf\endcsname{\def\PY@tc##1{\textcolor[rgb]{0.40,0.40,0.40}{##1}}}
\expandafter\def\csname PY@tok@mh\endcsname{\def\PY@tc##1{\textcolor[rgb]{0.40,0.40,0.40}{##1}}}
\expandafter\def\csname PY@tok@mi\endcsname{\def\PY@tc##1{\textcolor[rgb]{0.40,0.40,0.40}{##1}}}
\expandafter\def\csname PY@tok@il\endcsname{\def\PY@tc##1{\textcolor[rgb]{0.40,0.40,0.40}{##1}}}
\expandafter\def\csname PY@tok@mo\endcsname{\def\PY@tc##1{\textcolor[rgb]{0.40,0.40,0.40}{##1}}}
\expandafter\def\csname PY@tok@ch\endcsname{\let\PY@it=\textit\def\PY@tc##1{\textcolor[rgb]{0.25,0.50,0.50}{##1}}}
\expandafter\def\csname PY@tok@cm\endcsname{\let\PY@it=\textit\def\PY@tc##1{\textcolor[rgb]{0.25,0.50,0.50}{##1}}}
\expandafter\def\csname PY@tok@cpf\endcsname{\let\PY@it=\textit\def\PY@tc##1{\textcolor[rgb]{0.25,0.50,0.50}{##1}}}
\expandafter\def\csname PY@tok@c1\endcsname{\let\PY@it=\textit\def\PY@tc##1{\textcolor[rgb]{0.25,0.50,0.50}{##1}}}
\expandafter\def\csname PY@tok@cs\endcsname{\let\PY@it=\textit\def\PY@tc##1{\textcolor[rgb]{0.25,0.50,0.50}{##1}}}

\def\PYZbs{\char`\\}
\def\PYZus{\char`\_}
\def\PYZob{\char`\{}
\def\PYZcb{\char`\}}
\def\PYZca{\char`\^}
\def\PYZam{\char`\&}
\def\PYZlt{\char`\<}
\def\PYZgt{\char`\>}
\def\PYZsh{\char`\#}
\def\PYZpc{\char`\%}
\def\PYZdl{\char`\$}
\def\PYZhy{\char`\-}
\def\PYZsq{\char`\'}
\def\PYZdq{\char`\"}
\def\PYZti{\char`\~}
% for compatibility with earlier versions
\def\PYZat{@}
\def\PYZlb{[}
\def\PYZrb{]}
\makeatother


    % For linebreaks inside Verbatim environment from package fancyvrb. 
    \makeatletter
        \newbox\Wrappedcontinuationbox 
        \newbox\Wrappedvisiblespacebox 
        \newcommand*\Wrappedvisiblespace {\textcolor{red}{\textvisiblespace}} 
        \newcommand*\Wrappedcontinuationsymbol {\textcolor{red}{\llap{\tiny$\m@th\hookrightarrow$}}} 
        \newcommand*\Wrappedcontinuationindent {3ex } 
        \newcommand*\Wrappedafterbreak {\kern\Wrappedcontinuationindent\copy\Wrappedcontinuationbox} 
        % Take advantage of the already applied Pygments mark-up to insert 
        % potential linebreaks for TeX processing. 
        %        {, <, #, %, $, ' and ": go to next line. 
        %        _, }, ^, &, >, - and ~: stay at end of broken line. 
        % Use of \textquotesingle for straight quote. 
        \newcommand*\Wrappedbreaksatspecials {% 
            \def\PYGZus{\discretionary{\char`\_}{\Wrappedafterbreak}{\char`\_}}% 
            \def\PYGZob{\discretionary{}{\Wrappedafterbreak\char`\{}{\char`\{}}% 
            \def\PYGZcb{\discretionary{\char`\}}{\Wrappedafterbreak}{\char`\}}}% 
            \def\PYGZca{\discretionary{\char`\^}{\Wrappedafterbreak}{\char`\^}}% 
            \def\PYGZam{\discretionary{\char`\&}{\Wrappedafterbreak}{\char`\&}}% 
            \def\PYGZlt{\discretionary{}{\Wrappedafterbreak\char`\<}{\char`\<}}% 
            \def\PYGZgt{\discretionary{\char`\>}{\Wrappedafterbreak}{\char`\>}}% 
            \def\PYGZsh{\discretionary{}{\Wrappedafterbreak\char`\#}{\char`\#}}% 
            \def\PYGZpc{\discretionary{}{\Wrappedafterbreak\char`\%}{\char`\%}}% 
            \def\PYGZdl{\discretionary{}{\Wrappedafterbreak\char`\$}{\char`\$}}% 
            \def\PYGZhy{\discretionary{\char`\-}{\Wrappedafterbreak}{\char`\-}}% 
            \def\PYGZsq{\discretionary{}{\Wrappedafterbreak\textquotesingle}{\textquotesingle}}% 
            \def\PYGZdq{\discretionary{}{\Wrappedafterbreak\char`\"}{\char`\"}}% 
            \def\PYGZti{\discretionary{\char`\~}{\Wrappedafterbreak}{\char`\~}}% 
        } 
        % Some characters . , ; ? ! / are not pygmentized. 
        % This macro makes them "active" and they will insert potential linebreaks 
        \newcommand*\Wrappedbreaksatpunct {% 
            \lccode`\~`\.\lowercase{\def~}{\discretionary{\hbox{\char`\.}}{\Wrappedafterbreak}{\hbox{\char`\.}}}% 
            \lccode`\~`\,\lowercase{\def~}{\discretionary{\hbox{\char`\,}}{\Wrappedafterbreak}{\hbox{\char`\,}}}% 
            \lccode`\~`\;\lowercase{\def~}{\discretionary{\hbox{\char`\;}}{\Wrappedafterbreak}{\hbox{\char`\;}}}% 
            \lccode`\~`\:\lowercase{\def~}{\discretionary{\hbox{\char`\:}}{\Wrappedafterbreak}{\hbox{\char`\:}}}% 
            \lccode`\~`\?\lowercase{\def~}{\discretionary{\hbox{\char`\?}}{\Wrappedafterbreak}{\hbox{\char`\?}}}% 
            \lccode`\~`\!\lowercase{\def~}{\discretionary{\hbox{\char`\!}}{\Wrappedafterbreak}{\hbox{\char`\!}}}% 
            \lccode`\~`\/\lowercase{\def~}{\discretionary{\hbox{\char`\/}}{\Wrappedafterbreak}{\hbox{\char`\/}}}% 
            \catcode`\.\active
            \catcode`\,\active 
            \catcode`\;\active
            \catcode`\:\active
            \catcode`\?\active
            \catcode`\!\active
            \catcode`\/\active 
            \lccode`\~`\~ 	
        }
    \makeatother

    \let\OriginalVerbatim=\Verbatim
    \makeatletter
    \renewcommand{\Verbatim}[1][1]{%
        %\parskip\z@skip
        \sbox\Wrappedcontinuationbox {\Wrappedcontinuationsymbol}%
        \sbox\Wrappedvisiblespacebox {\FV@SetupFont\Wrappedvisiblespace}%
        \def\FancyVerbFormatLine ##1{\hsize\linewidth
            \vtop{\raggedright\hyphenpenalty\z@\exhyphenpenalty\z@
                \doublehyphendemerits\z@\finalhyphendemerits\z@
                \strut ##1\strut}%
        }%
        % If the linebreak is at a space, the latter will be displayed as visible
        % space at end of first line, and a continuation symbol starts next line.
        % Stretch/shrink are however usually zero for typewriter font.
        \def\FV@Space {%
            \nobreak\hskip\z@ plus\fontdimen3\font minus\fontdimen4\font
            \discretionary{\copy\Wrappedvisiblespacebox}{\Wrappedafterbreak}
            {\kern\fontdimen2\font}%
        }%
        
        % Allow breaks at special characters using \PYG... macros.
        \Wrappedbreaksatspecials
        % Breaks at punctuation characters . , ; ? ! and / need catcode=\active 	
        \OriginalVerbatim[#1,codes*=\Wrappedbreaksatpunct]%
    }
    \makeatother

    % Exact colors from NB
    \definecolor{incolor}{HTML}{303F9F}
    \definecolor{outcolor}{HTML}{D84315}
    \definecolor{cellborder}{HTML}{CFCFCF}
    \definecolor{cellbackground}{HTML}{F7F7F7}
    
    % prompt
    \newcommand{\prompt}[4]{
        \llap{{\color{#2}[#3]: #4}}\vspace{-1.25em}
    }
    

    
    % Prevent overflowing lines due to hard-to-break entities
    \sloppy 
    % Setup hyperref package
    \hypersetup{
      breaklinks=true,  % so long urls are correctly broken across lines
      colorlinks=true,
      urlcolor=urlcolor,
      linkcolor=linkcolor,
      citecolor=citecolor,
      }
    % Slightly bigger margins than the latex defaults
    
    \geometry{verbose,tmargin=1in,bmargin=1in,lmargin=1in,rmargin=1in}
    
    

    \begin{document}
    
    
    \maketitle
    
    

    
    \hypertarget{distribuciuxf3n-gaussiana-bidimensional}{%
\section{Distribución Gaussiana
Bidimensional}\label{distribuciuxf3n-gaussiana-bidimensional}}

Consideremos dos variables aleatorias, \(X\) e \(Y\), agrupadas en un
vector aleatorio \(\mathbf{V} = (X, Y)^\mathrm{T}\). Su \textbf{vector
de medias} es

\[
\boldsymbol{\mu_V}\equiv E(\mathbf{V}) = \left( \begin{array}{c}{E(X)} \\ {E(Y)}\end{array}\right) \equiv \left( \begin{array}{c}{\mu_{X}} \\ {\mu_{Y}}\end{array}\right)
\]

Y su \textbf{matriz de covarianza} es

\[
\boldsymbol{\Sigma_V}\equiv\mathbf{C_{XY}} = 
E\left((\mathbf{V}-\boldsymbol{\mu_V})(\mathbf{V}-\boldsymbol{\mu_V})^\mathrm{T} \right) =
\left[ \begin{array}{cc}{\sigma_{X}^{2}} & {\rho_{XY} \sigma_{X} \sigma_{Y}} \\ {\rho_{XY} \sigma_{X} \sigma_{Y}} & {\sigma_{Y}^{2}}\end{array}\right]
\]

donde:

\[\mathbf{C_{XY}} = 
\begin{pmatrix}
E((X-\mu_X)^2) & E((X-\mu_X)(Y-\mu_Y))\\
E((X-\mu_X)(Y-\mu_Y)) & E((Y-\mu_Y)^2) 
\end{pmatrix} = \begin{pmatrix}
\sigma_X^2 & C_{XY}\\
C_{XY} & \sigma_Y^2
\end{pmatrix}
\]

    \hypertarget{funciuxf3n-de-densidad-conjuntamente-gaussiana}{%
\subsection{Función de densidad conjuntamente
Gaussiana}\label{funciuxf3n-de-densidad-conjuntamente-gaussiana}}

El vector aleatorio \(\mathbf{V}=(X,Y)^\mathrm{T}\) es Normal o
Gaussiano o, alternativamente, las variables aleatorias \(X\) e \(Y\)
son conjuntamente Gaussianas si la función de densidad de probabilidad
conjunta es \((N=2)\):

\[
f_{\mathbf{V}}\left(\mathbf{v}\right)\equiv f_{XY}\left(x,y\right)=\frac{1}{\sqrt{(2 \pi)^N\boldsymbol{|\Sigma_V}|}}\exp \left(-\frac{1}{2}(\mathbf{v}-\boldsymbol{\mu_V})^{\mathrm{T}} \boldsymbol{\Sigma_V}^{-1}(\mathbf{v}-\boldsymbol{\mu_V})\right)
\]

Esta expresión es general para vectores aleatorios con \(N\) componentes
aleatorias. En el caso bidimensional:

\[
|\boldsymbol{\Sigma_V}|=\sigma_{X}^{2} \sigma_{Y}^{2}\left(1-\rho_{XY}^{2}\right)
\]

\[
\boldsymbol{\Sigma_V}^{-1}=\frac{1}{\sigma_{X}^{2} \sigma_{Y}^{2}\left(1-\rho_{XY}^{2}\right)} \left[ \begin{array}{cc}{\sigma_{Y}^{2}} & {-\rho_{XY} \sigma_{X} \sigma_{Y}} \\ {-\rho_{XY} \sigma_{X} \sigma_{Y}} & {\sigma_{X}^{2}}\end{array}\right]
\]

    Sustituyendo y desarrollando, la \textbf{función de densidad de
probabilidad conjunta} es:

\[
f_{XY}(x, y)=\frac{1}{2 \pi \sigma_{X} \sigma_{Y} \sqrt{1-\rho_{XY}^{2}}} \exp \left(-\frac{1}{2\left(1-\rho_{XY}^{2}\right)}\left[\frac{\left(x-\mu_{X}\right)^{2}}{\sigma_{X}^{2}}+\frac{\left(y-\mu_{Y}\right)^{2}}{\sigma_{Y}^{2}}-\frac{2 \rho_{XY}\left(x-\mu_{X}\right)\left(y-\mu_{Y}\right)}{\sigma_{X} \sigma_{Y}}\right]\right)
\]

Las \textbf{funciones de densidad de probabilidad marginales} son:

\[
f_{X}(x)=\frac{1}{\sqrt{2 \pi} \sigma_{X} } e^{-\frac{\left(x-\mu_{X}\right)^{2}} {2\sigma_{X}^{2}}} 
\]

\[
f_{Y}(y)=\frac{1}{\sqrt{2 \pi} \sigma_{Y} } e^{-\frac{\left(y-\mu_{X}\right)^{2}} {2\sigma_{Y}^{2}}} 
\]

    Adviértase que la caracterización marginal de \(X\) e \(Y\) no permite,
en general, alcanzar conclusiones sobre la caracterización conjunta. En
particular, si \(X\) e \(Y\) son marginalmente normales, ello ni
siquiera supone que \((X,Y)\) sean conjuntamente normales.

Sin embargo, si las variables aleatorias \(X\) e \(Y\) son
\textbf{independientes} puede obtenerse la función de densidad conjunta
a partir de las marginales:

\[
f_{XY}(x, y)=f_{X}(x)f_{Y}(y) = \frac{1}{2 \pi \sigma_{X} \sigma_{Y}} \exp \left(-\frac{1}{2}\left[\frac{\left(x-\mu_{X}\right)^{2}}{\sigma_{X}^{2}}+\frac{\left(y-\mu_{Y}\right)^{2}}{\sigma_{Y}^{2}}\right]\right)
\]

Adviértase que si las variables aleatorias \(X\) e \(Y\) están
incorreladas, \(\rho_{XY}=0\) y la función de densidad conjunta es igual
al caso en que las variables aleatorias sean independientes. Por tanto,
\textbf{si las variables aleatorias son conjuntamente normales la
independencia no solo implica incorrelación, sino que ambas propiedades
son equivalentes}.

    Las \emph{isolíneas} de la función de densidad conjunta corresponden a
los puntos del plano \((X,Y)\) en los que dicha función es constante. En
general son elipses y definen la denominada \textbf{distancia de
Mahalanobis}, \(d_M\) del punto \(\mathbf{v}=(x,y)^\mathrm{T}\) al valor
medio \(\boldsymbol{\mu_V}=(\mu_X,\mu_Y)^\mathrm{T}\):

\[
d_M^2(\mathbf{v};\boldsymbol{\mu_V})=(\mathbf{v}-\boldsymbol{\mu_V})^{\mathrm{T}} \boldsymbol{\Sigma_V}^{-1}(\mathbf{v}-\boldsymbol{\mu_V})
\]

Esta expresión es general para \(N\) variables aleatorias. Desarrollando
para el caso de dos:

\[
d_M^2\left((x,y);(\mu_X,\mu_Y)\right) = \frac{\left(x-\mu_{X}\right)^{2}}{\sigma_{X}^{2}}+\frac{\left(y-\mu_{Y}\right)^{2}}{\sigma_{Y}^{2}}-\frac{2 \rho\left(x-\mu_{X}\right)\left(y-\mu_{Y}\right)}{\sigma_{X} \sigma_{Y}}
\]

    \hypertarget{vector-aleatorio-normal-estuxe1ndar}{%
\subsubsection{Vector aleatorio Normal
estándar}\label{vector-aleatorio-normal-estuxe1ndar}}

Es el caso en el que:

\begin{itemize}
\tightlist
\item
  las variables aleatorias son independientes y, por tanto,
  incorreladas,
\item
  las medias son nulas y
\item
  las desviaciones típicas son la unidad.
\end{itemize}

Ello supone que \textbf{la matriz de covarianza es la matriz identidad}:

\[
f_{XY}(x, y)= \frac{1}{2 \pi} e^{-\frac{1}{2}\left(x^2 + y^2 \right)}
\]

En el gráfico que sigue, se muestran las funciones de densidad
marginales escaladas para que las modas coincidan con la moda de la
densidad conjunta.

    \begin{center}
    \adjustimage{max size={0.9\linewidth}{0.9\paperheight}}{III.2.GaussianaBidimensional_files/III.2.GaussianaBidimensional_10_0.png}
    \end{center}
    { \hspace*{\fill} \\}
    
    En este casi las isolíneas son circunferencias, centradas en la media, y
la distancia de Mahalanobis corresponde a la distancia Euclídea
habitual:

    \begin{center}
    \adjustimage{max size={0.9\linewidth}{0.9\paperheight}}{III.2.GaussianaBidimensional_files/III.2.GaussianaBidimensional_12_0.png}
    \end{center}
    { \hspace*{\fill} \\}
    
    Puede hacerse una representación intuitiva muestreando la función de
densidad conjunta y mostrando el diagrama de dispersión resultante,
junto con las densidades marginales:

    \begin{center}
    \adjustimage{max size={0.9\linewidth}{0.9\paperheight}}{III.2.GaussianaBidimensional_files/III.2.GaussianaBidimensional_14_0.png}
    \end{center}
    { \hspace*{\fill} \\}
    
    La \textbf{función de distribución de probabilidad} es:

    \begin{center}
    \adjustimage{max size={0.9\linewidth}{0.9\paperheight}}{III.2.GaussianaBidimensional_files/III.2.GaussianaBidimensional_16_0.png}
    \end{center}
    { \hspace*{\fill} \\}
    
    \hypertarget{independencia-con-desviaciones-tuxedpicas-diferentes}{%
\subsubsection{Independencia con desviaciones típicas
diferentes}\label{independencia-con-desviaciones-tuxedpicas-diferentes}}

Es este caso:

\begin{itemize}
\tightlist
\item
  las variables aleatorias son independientes y, por tanto,
  incorreladas,
\item
  las medias son nulas y
\item
  las desviaciones típicas son diferentes, \(\sigma_X\) y \(\sigma_Y\).
\end{itemize}

Ello supone que \textbf{la matriz de covarianza es diagonal} con matriz
inversa:

\[\boldsymbol{\Sigma_V}\equiv \mathbf{C_{XY}} = 
 \begin{pmatrix}
\sigma_X^2 & 0\\
0 & \sigma_Y^2
\end{pmatrix} \qquad
\boldsymbol{\Sigma_V}^{-1}=\left[ \begin{array}{cc}{\frac{1}{\sigma_{X}^{2}}} & {0} \\ {0} & {\frac{1}{\sigma_{Y}^{2}}}\end{array}\right]
\]

La función de densidad conjunta resulta:

\[
f_{XY}(x, y)= \frac{1}{2 \pi\sigma_X\sigma_Y} e^{-\frac{1}{2}\left[(\frac{x}{\sigma_X})^2 + (\frac{y}{\sigma_Y})^2 \right]}
\]

    \hypertarget{sigma_x-2-sigma_y-12}{%
\paragraph{\texorpdfstring{\(\sigma_X = 2\),
\(\sigma_Y = 1/2\)}{\textbackslash{}sigma\_X = 2, \textbackslash{}sigma\_Y = 1/2}}\label{sigma_x-2-sigma_y-12}}

    \begin{center}
    \adjustimage{max size={0.9\linewidth}{0.9\paperheight}}{III.2.GaussianaBidimensional_files/III.2.GaussianaBidimensional_22_0.png}
    \end{center}
    { \hspace*{\fill} \\}
    
    \begin{center}
    \adjustimage{max size={0.9\linewidth}{0.9\paperheight}}{III.2.GaussianaBidimensional_files/III.2.GaussianaBidimensional_23_0.png}
    \end{center}
    { \hspace*{\fill} \\}
    
    \begin{center}
    \adjustimage{max size={0.9\linewidth}{0.9\paperheight}}{III.2.GaussianaBidimensional_files/III.2.GaussianaBidimensional_24_0.png}
    \end{center}
    { \hspace*{\fill} \\}
    
    \hypertarget{sigma_x-12-sigma_y-2}{%
\paragraph{\texorpdfstring{\(\sigma_X = 1/2\),
\(\sigma_Y = 2\)}{\textbackslash{}sigma\_X = 1/2, \textbackslash{}sigma\_Y = 2}}\label{sigma_x-12-sigma_y-2}}

    \begin{center}
    \adjustimage{max size={0.9\linewidth}{0.9\paperheight}}{III.2.GaussianaBidimensional_files/III.2.GaussianaBidimensional_29_0.png}
    \end{center}
    { \hspace*{\fill} \\}
    
    \begin{center}
    \adjustimage{max size={0.9\linewidth}{0.9\paperheight}}{III.2.GaussianaBidimensional_files/III.2.GaussianaBidimensional_30_0.png}
    \end{center}
    { \hspace*{\fill} \\}
    
    \begin{center}
    \adjustimage{max size={0.9\linewidth}{0.9\paperheight}}{III.2.GaussianaBidimensional_files/III.2.GaussianaBidimensional_31_0.png}
    \end{center}
    { \hspace*{\fill} \\}
    
    \hypertarget{dos-variables-aleatorias-correladas}{%
\subsubsection{Dos variables aleatorias
correladas}\label{dos-variables-aleatorias-correladas}}

Consideremos, a modo de ejemplo, que ambas funciones de densidad
marginal sean normales estándar, pero que tengan una correlación
\(\rho_{XY} = 0.9\).

    \begin{center}
    \adjustimage{max size={0.9\linewidth}{0.9\paperheight}}{III.2.GaussianaBidimensional_files/III.2.GaussianaBidimensional_36_0.png}
    \end{center}
    { \hspace*{\fill} \\}
    
    \begin{center}
    \adjustimage{max size={0.9\linewidth}{0.9\paperheight}}{III.2.GaussianaBidimensional_files/III.2.GaussianaBidimensional_37_0.png}
    \end{center}
    { \hspace*{\fill} \\}
    
    \begin{center}
    \adjustimage{max size={0.9\linewidth}{0.9\paperheight}}{III.2.GaussianaBidimensional_files/III.2.GaussianaBidimensional_38_0.png}
    \end{center}
    { \hspace*{\fill} \\}
    
    \hypertarget{sigma_x-12-sigma_y-2-rho_xy0.9}{%
\paragraph{\texorpdfstring{\(\sigma_X = 1/2\), \(\sigma_Y = 2\),
\(\rho_{XY}=0.9\)}{\textbackslash{}sigma\_X = 1/2, \textbackslash{}sigma\_Y = 2, \textbackslash{}rho\_\{XY\}=0.9}}\label{sigma_x-12-sigma_y-2-rho_xy0.9}}

    \begin{center}
    \adjustimage{max size={0.9\linewidth}{0.9\paperheight}}{III.2.GaussianaBidimensional_files/III.2.GaussianaBidimensional_43_0.png}
    \end{center}
    { \hspace*{\fill} \\}
    
    \begin{center}
    \adjustimage{max size={0.9\linewidth}{0.9\paperheight}}{III.2.GaussianaBidimensional_files/III.2.GaussianaBidimensional_44_0.png}
    \end{center}
    { \hspace*{\fill} \\}
    
    \begin{center}
    \adjustimage{max size={0.9\linewidth}{0.9\paperheight}}{III.2.GaussianaBidimensional_files/III.2.GaussianaBidimensional_45_0.png}
    \end{center}
    { \hspace*{\fill} \\}
    
    \hypertarget{transformaciuxf3n-lineal-de-variables-aleatorias-conjuntamente-gaussianas}{%
\subsubsection{Transformación lineal de variables aleatorias
conjuntamente
Gaussianas}\label{transformaciuxf3n-lineal-de-variables-aleatorias-conjuntamente-gaussianas}}

Si \(\mathbf{V}=(X,Y)^\mathrm{T}\) es un vector conjuntamente Gaussiano
ya sabemos que las distribuciones marginales de sus componentes también
lo son. Además:

La variable aleatoria \(W = aX+bY+c = \mathbf{k}^T\mathbf{V} + c\) es
Gaussiana, donde:

\[E(W)\equiv \mu_W = \mathbf{k}^T\boldsymbol{\mu_V} + c\]

\[Var(W)\equiv \sigma_W^2 = \mathbf{k}^\mathrm{T}E\left((\mathbf{V}-\boldsymbol{\mu_V})(\mathbf{V}-\boldsymbol{\mu_V})^\mathrm{T} \right)\mathbf{k}=\mathbf{k}^\mathrm{T}\boldsymbol{\Sigma_V}\mathbf{k}\]

Adviértase que si \(c=0\) y \(\mathbf{k}\) es ortonormal
\(\left( \| \mathbf{k}\| = 1 \right)\), la transformación representa una
proyección ortogonal de la distribución sobre la dirección definida por
\(\mathbf{k}\). Son casos particulares interesantes:

\begin{itemize}
\tightlist
\item
  \(a=1\), \(b=c=0\), que proporciona la distribución marginal sobre el
  eje \(X\)
\item
  \(a=c=0\), \(b=1\), que proporciona la distribución marginal sobre el
  eje \(Y\)
\end{itemize}

    Las variables aleatorias \(W\), \(Z\), definidas como sigue, son
conjuntamente Gaussianas

\begin{align}
W = aX+bY+c &= \mathbf{k_1}^T\mathbf{V} + c\\   
Z = dX+eY+f &= \mathbf{k_2}^T\mathbf{V} + f
\end{align}

Que puede expresarse matricialmente como sigue:

\[
\left( \begin{array}{c}{W} \\ {Z}\end{array}\right) = 
\left[ \begin{array}{cc}{a} & {b} \\ {c} & {d}\end{array}\right]\left( \begin{array}{c}{X} \\ {Y}\end{array}\right) + \left( \begin{array}{c}{c} \\ {f}\end{array}\right)
\]

O de forma más compacta:

\[
\mathbf{U} = \mathbf{K}^\mathrm{T}\mathbf{V}+\mathbf{p}
\]

donde \(\mathbf{U}=(W, Z)^\mathrm{T}\),
\(\mathbf{p}=(c, f)^\mathrm{T}\),
\(\mathbf{K} = \left[\mathbf{k_1} \ \mathbf{k_2} \right]\).

    El vector transformado \(\mathbf{U}=(W, Z)^\mathrm{T}\) tiene el
siguiente vector de medias y matriz de covarianza:

\[
E(\mathbf{U}) \equiv \boldsymbol{\mu_U} = \mathbf{K}^\mathrm{T}\boldsymbol{\mu_V}+\mathbf{p}
\]

\[
\mathbf{C_{WZ}}\equiv \boldsymbol{\Sigma_U} = \mathbf{K}^\mathrm{T}E\left((\mathbf{V}-\boldsymbol{\mu_V})(\mathbf{V}-\boldsymbol{\mu_V})^\mathrm{T} \right)\mathbf{K}=\mathbf{K}^\mathrm{T}\boldsymbol{\Sigma_V}\mathbf{K}
\]

En general, esta transformación representa una \textbf{transformación
afín} del plano \((X,Y)\). Un caso particular de gran importancia es
cuando se trata de un \textbf{movimiento}, esto es de la combinación de
una \textbf{traslación} y de una \textbf{rotación}, en cuyo caso la
matriz \(\mathbf{K} \equiv \mathbf{R}\) es \textbf{ortogonal}
\(\left(\mathbf{RR}^\mathrm{T} = \mathbf{I} \implies \mathbf{R}^{-1} = \mathbf{R}\right)\)
y su determinante es uno \(\left(| \mathbf{R}| =1 \right)\).

    Como las matrices de covarianza son siempre simétricas semidefinidas
positivas, pueden ser diagonalizadas en una base ortonormal de
autovectores
\(\boldsymbol\Phi = \left[\hat{\mathbf{e_1}} \ \hat{\mathbf{e_2}}\right]\),
\(\boldsymbol{\Phi\Phi}^\mathrm{T}=\mathbf{I}\), siendo los autovalores
correspondientes
\(\boldsymbol{\Lambda}= \left[ \begin{array}{cc}{\lambda_1} & {0} \\ {0} & {\lambda_2}\end{array}\right]\).
Por tanto:

\[\boldsymbol{\Sigma_V}\hat{\mathbf{e_i}}=\lambda_i\hat{\mathbf{e_i}}\implies 
\boldsymbol{\Sigma_V}\boldsymbol{\Phi}=\boldsymbol{\Phi\Lambda}=\boldsymbol{\Phi\Lambda}^{1/2}\boldsymbol{\Lambda}^{1/2}\]

\[
\boldsymbol{\Lambda}^{-1/2}\boldsymbol{\Phi}^\mathrm{T}\boldsymbol{\Sigma_V}\boldsymbol{\Phi}\boldsymbol{\Lambda}^{-1/2}=\mathbf{I}
\]

De ello se deduce que, si
\(\mathbf{K}=\boldsymbol{\Phi}\boldsymbol{\Lambda}^{-1/2}\) la matriz de
covarianza resultante de la transformación lineal es la identidad.
Podemos extender al caso de dos variables aleatorias la tipificación de
una única variable aleatoria como sigue:

\[
\mathbf{U} = \mathbf{K}^\mathrm{T}(\mathbf{V}-\boldsymbol{\mu_V}) =
\left(\boldsymbol{\Phi}\boldsymbol{\Lambda}^{-1/2} \right)^\mathrm{T}(\mathbf{V}-\boldsymbol{\mu_V})
\]

Resultando \(E(\mathbf{U}) \equiv \boldsymbol{\mu_U} = \mathbf{0}\) y
\(\mathbf{C_{WZ}}\equiv \boldsymbol{\Sigma_U} = \mathbf{I}\).

Esta transformación suelle llamarse \textbf{blanqueado} y se extiende
fácilmente al caso de \(N\) variables aleatorias conjuntas.

    \hypertarget{distribuciones-condicionadas}{%
\subsection{Distribuciones
condicionadas}\label{distribuciones-condicionadas}}

Consideremos dos variables aleatorias, \(X\) e \(Y\), conjuntamente
Gaussianas y que conocemos su función de densidad conjunta
\(f_{XY}(x,y)\) o, lo que es equivalente, que conocemos sus valores
medios y su matriz de covarianza.

Ya hemos visto que, en tal caso, tanto las distribuciones marginales son
Gaussianas, así como que cualquier transformación lineal resulta
conjuntamente Gaussiana.

Además, la función de densidad condicionada de una variable aleatoria
por la otra, también resulta Gaussiana:

\[f_{X | Y}(x | y)=\frac{f_{XY}(x,y)}{f_Y(y)} = N(\mu_{X|Y}, \sigma_{X|Y})\]

\[f_{Y | X}(y | x)=\frac{f_{XY}(x,y)}{f_X(x)} = N(\mu_{Y|X}, \sigma_{Y|X})\]

Que quedan totalmente determinadas por el valor medio y la desviación
típica de una variable aleatoria condicionada por la otra.

    Sin más que aplicar la definición y operar:

\[
f_{Y | X}(y | x)=
\frac{1}{\sqrt{2 \pi} \sigma_{Y} \sqrt{1-\rho_{XY}^{2}}} \exp \left[-\frac{1}{2}\left(\frac{y-\left[\mu_{Y}+\rho_{XY} \frac{\sigma_{Y}}{\sigma_{X}}\left(x-\mu_{X}\right)\right]}{\sigma_{Y} \sqrt{1-\rho_{XY}^{2}}}\right)^{2}\right]
\]

donde:

\[
\mu_{Y | X}(x) \equiv E(Y | X) = \mu_{Y}+\rho_{XY} \frac{\sigma_{Y}}{\sigma_{X}}\left(x-\mu_{X}\right)
\]

\[
\sigma_{Y | X}^2 \equiv Var(Y | X) = \sigma_{Y}^2 (1-\rho_{XY}^{2})
\]

\(f_{X | Y}(x | y)\) y sus momentos se obtiene sin más que intecambiar
ambas variables.

    \hypertarget{ejemplo-de-estimaciuxf3n-con-variables-conjuntamente-gaussianas}{%
\subsubsection{Ejemplo de estimación con variables conjuntamente
Gaussianas}\label{ejemplo-de-estimaciuxf3n-con-variables-conjuntamente-gaussianas}}

Consideremos que dos variables aleatorias \((X, Y)\) son conjuntamente
Gaussianas, y que conocemos la función de densidad de las observaciones
\(Y\) dadas las causas \(X\), \(f_{Y | X}(y | x)\) o verosimilitud
\(L(x)\) de la causa una vez hecha la observación. La Gaussianidad
implica que es suficiente con conocer la media y la desviación típica
condicionadas. La media condicionada tiene necesariamente una relación
lineal con las causas \(x\) como se vio en la expresión general:

\[
L(x) \equiv f_{Y | X}(y | x)=\frac{1}{\sqrt{2\pi}\sigma_{Y | X}}e^{-\frac{\left[y-(ax+b)\right]^2}{2\sigma_{Y | X}^2}}
\]

donde \(\mu_{Y | X}(x) = ax + b\). En nuestro ejemplo vamos a considerar

\[
\mu_{Y | X}(x) = 3.6x - 0.8
\]

\[
\sigma_{Y | X}^2 = 0.76
\]

Supongamos para nuestro ejemplo que observamos una medida \(Y=1.5\).
Nuestro objetivo es estimar \(X\) a partir de la observación.

    En ausencia de información adicional hemos de aplicar el
\textbf{principio de máxima verosimilitud}, esto es, una vez observada
\(Y=1.5\), qué valor de \(x\) maximiza \(L(x)\). Como sabemos, la moda
de la Gaussiana coeincide con su valor medio. La simetría del exponente
permite concluir que el estimador que maximiza la verosimilitud se
obtiene cuando \(ax+b = y\). Por tanto:

\[
\hat{X} = \frac{y-b}{a} = \frac{1.5+0.8}{3.6} \approx 0.639
\]

El estimador puede mejorarse si conocemos la \textbf{distribución
\emph{a priori}} de las causas \(X\). De nuevo, la Gaussianidad implica
que es suficiente con conocer la media y la desviación típica a priori.
Consideremos pra el ejemplo que la media es \(\mu_X = 0.5\) y que la
desviación típica es \(\sigma_Y=0.5\).

El Teorema de Bayes junto con el de la probabilidad total nos permiten
obtener la \textbf{función de densidad \emph{a posteriori}}. Sin
embargo, la Gaussianidad conjunta proporciona expresiones cerradas tanto
para la media como para la desviación típica a posteriori, lo que
simplifica enormemente el problema.

    Recapitulemos. Conocemos:

\begin{itemize}
\tightlist
\item
  Verosimilitud:

  \begin{itemize}
  \tightlist
  \item
    \(\mu_{Y | X}(x) = ax + b = 3.6x - 0.8\)
  \item
    \(\sigma_{Y | X}^2 = 0.76\)
  \end{itemize}
\item
  Distribución \emph{a priori}:

  \begin{itemize}
  \tightlist
  \item
    \(\mu_X = 0.5\)
  \item
    \(\sigma_X = 0.5\)
  \end{itemize}
\end{itemize}

Necesitamos obtener \(\mu_Y\), \(\sigma_Y\) y \(\rho{XY}\) y, con ello,
tenemos inmediatamente tanto la caracterización conjunta como la
caracterización \emph{a posteriori}

    \[
ax+b = \mu_{Y}+\rho_{XY} \frac{\sigma_{Y}}{\sigma_{X}}\left(x-\mu_{X}\right)
\]

Identificando coeficientes, obtenemos dos ecuaciones:

\begin{align}
a &= \rho_{XY} \frac{\sigma_{Y}}{\sigma_{X}} \implies \rho_{XY}\sigma_{Y} = a\sigma_{X} = 3.6\frac{1}{2}=1.8\\
b &= \mu_{Y}-\rho_{XY} \frac{\sigma_{Y}}{\sigma_{X}}\mu_{X} \implies \mu_{Y} = b + a\mu_X = -0.8 + 3.6\frac{1}{2}=1.0
\end{align}

A la que añadimos la siguiente, para obtener tres ecuaciones para tres
incógnitas

\[
\sigma_{Y | X}^2 = \sigma_{Y}^2 (1-\rho_{XY}^{2}) \implies \sigma_{Y}^2 = \sigma_{Y | X}^2 + \sigma_{Y}^2 \rho_{XY}^{2} = \sigma_{Y | X}^2 + (a\sigma_{X})^2 = 0.76 + (1.8)^2 = 4.0
\]

Por tanto: \(\mu_Y = 1.0\), \(\sigma_Y = 2\), \(\rho_{XY}=0.9\)
(adviértase que \(\rho_{XY}\) lleva el signo de \(a\)).

    Ya tenemos toda la información para obtener la \textbf{función de
densidad \emph{a posteriori}}, \(f_{X | Y}(x | y)\), que queda
complematmente especificada por la media y la desviación típica \emph{a
posteriori} gracias a la Gaussianidad:

\[
\mu_{X | Y}(y) \equiv E(X | Y) = \mu_{X}+\rho_{XY} \frac{\sigma_{X}}{\sigma_{Y}}\left(y-\mu_{Y}\right) = 0.5 + 0.225 (y -1)
\]

\[
\sigma_{X | Y}^2 \equiv Var(X | Y) = \sigma_{X}^2 (1-\rho_{XY}^{2}) =0.0475 \implies \sigma_{X | Y} \approx 0.218 < \sigma_X
\]

Pueden obtenerse expresiones cerradas en función de los momentos de la
verosimilitud y de la densidad \emph{a priori}.

Podemos ahora utilizar el \textbf{principio Máximo \emph{a Posteriori}}
que hace corresponder el estimador \(\hat{X}\) a la media condicionada
\emph{a posteriori}:

\[
\hat{X} = \mu_{X | Y}(y) \equiv E(X | Y) = 0.5 + 0.9 \frac{0.5}{2}(1.5-1) = 0.6125
\]

Puede demostrarse que el estimador \textbf{Máximo \emph{a Posteriori}}
es \textbf{óptimo} en el sentido de minimizar el error cuadrático medio
\(E((\hat{X}-X)^2)\).


    % Add a bibliography block to the postdoc
    
    
    
    \end{document}
