
% Default to the notebook output style

    


% Inherit from the specified cell style.




    
\documentclass[11pt]{article}

    
    
    \usepackage[T1]{fontenc}
    % Nicer default font (+ math font) than Computer Modern for most use cases
    \usepackage{mathpazo}

    % Basic figure setup, for now with no caption control since it's done
    % automatically by Pandoc (which extracts ![](path) syntax from Markdown).
    \usepackage{graphicx}
    % We will generate all images so they have a width \maxwidth. This means
    % that they will get their normal width if they fit onto the page, but
    % are scaled down if they would overflow the margins.
    \makeatletter
    \def\maxwidth{\ifdim\Gin@nat@width>\linewidth\linewidth
    \else\Gin@nat@width\fi}
    \makeatother
    \let\Oldincludegraphics\includegraphics
    % Set max figure width to be 80% of text width, for now hardcoded.
    \renewcommand{\includegraphics}[1]{\Oldincludegraphics[width=.8\maxwidth]{#1}}
    % Ensure that by default, figures have no caption (until we provide a
    % proper Figure object with a Caption API and a way to capture that
    % in the conversion process - todo).
    \usepackage{caption}
    \DeclareCaptionLabelFormat{nolabel}{}
    \captionsetup{labelformat=nolabel}

    \usepackage{adjustbox} % Used to constrain images to a maximum size 
    \usepackage{xcolor} % Allow colors to be defined
    \usepackage{enumerate} % Needed for markdown enumerations to work
    \usepackage{geometry} % Used to adjust the document margins
    \usepackage{amsmath} % Equations
    \usepackage{amssymb} % Equations
    \usepackage{textcomp} % defines textquotesingle
    % Hack from http://tex.stackexchange.com/a/47451/13684:
    \AtBeginDocument{%
        \def\PYZsq{\textquotesingle}% Upright quotes in Pygmentized code
    }
    \usepackage{upquote} % Upright quotes for verbatim code
    \usepackage{eurosym} % defines \euro
    \usepackage[mathletters]{ucs} % Extended unicode (utf-8) support
    \usepackage[utf8x]{inputenc} % Allow utf-8 characters in the tex document
    \usepackage{fancyvrb} % verbatim replacement that allows latex
    \usepackage{grffile} % extends the file name processing of package graphics 
                         % to support a larger range 
    % The hyperref package gives us a pdf with properly built
    % internal navigation ('pdf bookmarks' for the table of contents,
    % internal cross-reference links, web links for URLs, etc.)
    \usepackage{hyperref}
    \usepackage{longtable} % longtable support required by pandoc >1.10
    \usepackage{booktabs}  % table support for pandoc > 1.12.2
    \usepackage[inline]{enumitem} % IRkernel/repr support (it uses the enumerate* environment)
    \usepackage[normalem]{ulem} % ulem is needed to support strikethroughs (\sout)
                                % normalem makes italics be italics, not underlines
    \usepackage{mathrsfs}
    

    
    
    % Colors for the hyperref package
    \definecolor{urlcolor}{rgb}{0,.145,.698}
    \definecolor{linkcolor}{rgb}{.71,0.21,0.01}
    \definecolor{citecolor}{rgb}{.12,.54,.11}

    % ANSI colors
    \definecolor{ansi-black}{HTML}{3E424D}
    \definecolor{ansi-black-intense}{HTML}{282C36}
    \definecolor{ansi-red}{HTML}{E75C58}
    \definecolor{ansi-red-intense}{HTML}{B22B31}
    \definecolor{ansi-green}{HTML}{00A250}
    \definecolor{ansi-green-intense}{HTML}{007427}
    \definecolor{ansi-yellow}{HTML}{DDB62B}
    \definecolor{ansi-yellow-intense}{HTML}{B27D12}
    \definecolor{ansi-blue}{HTML}{208FFB}
    \definecolor{ansi-blue-intense}{HTML}{0065CA}
    \definecolor{ansi-magenta}{HTML}{D160C4}
    \definecolor{ansi-magenta-intense}{HTML}{A03196}
    \definecolor{ansi-cyan}{HTML}{60C6C8}
    \definecolor{ansi-cyan-intense}{HTML}{258F8F}
    \definecolor{ansi-white}{HTML}{C5C1B4}
    \definecolor{ansi-white-intense}{HTML}{A1A6B2}
    \definecolor{ansi-default-inverse-fg}{HTML}{FFFFFF}
    \definecolor{ansi-default-inverse-bg}{HTML}{000000}

    % commands and environments needed by pandoc snippets
    % extracted from the output of `pandoc -s`
    \providecommand{\tightlist}{%
      \setlength{\itemsep}{0pt}\setlength{\parskip}{0pt}}
    \DefineVerbatimEnvironment{Highlighting}{Verbatim}{commandchars=\\\{\}}
    % Add ',fontsize=\small' for more characters per line
    \newenvironment{Shaded}{}{}
    \newcommand{\KeywordTok}[1]{\textcolor[rgb]{0.00,0.44,0.13}{\textbf{{#1}}}}
    \newcommand{\DataTypeTok}[1]{\textcolor[rgb]{0.56,0.13,0.00}{{#1}}}
    \newcommand{\DecValTok}[1]{\textcolor[rgb]{0.25,0.63,0.44}{{#1}}}
    \newcommand{\BaseNTok}[1]{\textcolor[rgb]{0.25,0.63,0.44}{{#1}}}
    \newcommand{\FloatTok}[1]{\textcolor[rgb]{0.25,0.63,0.44}{{#1}}}
    \newcommand{\CharTok}[1]{\textcolor[rgb]{0.25,0.44,0.63}{{#1}}}
    \newcommand{\StringTok}[1]{\textcolor[rgb]{0.25,0.44,0.63}{{#1}}}
    \newcommand{\CommentTok}[1]{\textcolor[rgb]{0.38,0.63,0.69}{\textit{{#1}}}}
    \newcommand{\OtherTok}[1]{\textcolor[rgb]{0.00,0.44,0.13}{{#1}}}
    \newcommand{\AlertTok}[1]{\textcolor[rgb]{1.00,0.00,0.00}{\textbf{{#1}}}}
    \newcommand{\FunctionTok}[1]{\textcolor[rgb]{0.02,0.16,0.49}{{#1}}}
    \newcommand{\RegionMarkerTok}[1]{{#1}}
    \newcommand{\ErrorTok}[1]{\textcolor[rgb]{1.00,0.00,0.00}{\textbf{{#1}}}}
    \newcommand{\NormalTok}[1]{{#1}}
    
    % Additional commands for more recent versions of Pandoc
    \newcommand{\ConstantTok}[1]{\textcolor[rgb]{0.53,0.00,0.00}{{#1}}}
    \newcommand{\SpecialCharTok}[1]{\textcolor[rgb]{0.25,0.44,0.63}{{#1}}}
    \newcommand{\VerbatimStringTok}[1]{\textcolor[rgb]{0.25,0.44,0.63}{{#1}}}
    \newcommand{\SpecialStringTok}[1]{\textcolor[rgb]{0.73,0.40,0.53}{{#1}}}
    \newcommand{\ImportTok}[1]{{#1}}
    \newcommand{\DocumentationTok}[1]{\textcolor[rgb]{0.73,0.13,0.13}{\textit{{#1}}}}
    \newcommand{\AnnotationTok}[1]{\textcolor[rgb]{0.38,0.63,0.69}{\textbf{\textit{{#1}}}}}
    \newcommand{\CommentVarTok}[1]{\textcolor[rgb]{0.38,0.63,0.69}{\textbf{\textit{{#1}}}}}
    \newcommand{\VariableTok}[1]{\textcolor[rgb]{0.10,0.09,0.49}{{#1}}}
    \newcommand{\ControlFlowTok}[1]{\textcolor[rgb]{0.00,0.44,0.13}{\textbf{{#1}}}}
    \newcommand{\OperatorTok}[1]{\textcolor[rgb]{0.40,0.40,0.40}{{#1}}}
    \newcommand{\BuiltInTok}[1]{{#1}}
    \newcommand{\ExtensionTok}[1]{{#1}}
    \newcommand{\PreprocessorTok}[1]{\textcolor[rgb]{0.74,0.48,0.00}{{#1}}}
    \newcommand{\AttributeTok}[1]{\textcolor[rgb]{0.49,0.56,0.16}{{#1}}}
    \newcommand{\InformationTok}[1]{\textcolor[rgb]{0.38,0.63,0.69}{\textbf{\textit{{#1}}}}}
    \newcommand{\WarningTok}[1]{\textcolor[rgb]{0.38,0.63,0.69}{\textbf{\textit{{#1}}}}}
    
    
    % Define a nice break command that doesn't care if a line doesn't already
    % exist.
    \def\br{\hspace*{\fill} \\* }
    % Math Jax compatibility definitions
    \def\gt{>}
    \def\lt{<}
    \let\Oldtex\TeX
    \let\Oldlatex\LaTeX
    \renewcommand{\TeX}{\textrm{\Oldtex}}
    \renewcommand{\LaTeX}{\textrm{\Oldlatex}}
    % Document parameters
    % Document title
    \title{2. Probabilidad y Álgebra de Sucesos}
    
    
    
    
    

    % Pygments definitions
    
\makeatletter
\def\PY@reset{\let\PY@it=\relax \let\PY@bf=\relax%
    \let\PY@ul=\relax \let\PY@tc=\relax%
    \let\PY@bc=\relax \let\PY@ff=\relax}
\def\PY@tok#1{\csname PY@tok@#1\endcsname}
\def\PY@toks#1+{\ifx\relax#1\empty\else%
    \PY@tok{#1}\expandafter\PY@toks\fi}
\def\PY@do#1{\PY@bc{\PY@tc{\PY@ul{%
    \PY@it{\PY@bf{\PY@ff{#1}}}}}}}
\def\PY#1#2{\PY@reset\PY@toks#1+\relax+\PY@do{#2}}

\expandafter\def\csname PY@tok@w\endcsname{\def\PY@tc##1{\textcolor[rgb]{0.73,0.73,0.73}{##1}}}
\expandafter\def\csname PY@tok@c\endcsname{\let\PY@it=\textit\def\PY@tc##1{\textcolor[rgb]{0.25,0.50,0.50}{##1}}}
\expandafter\def\csname PY@tok@cp\endcsname{\def\PY@tc##1{\textcolor[rgb]{0.74,0.48,0.00}{##1}}}
\expandafter\def\csname PY@tok@k\endcsname{\let\PY@bf=\textbf\def\PY@tc##1{\textcolor[rgb]{0.00,0.50,0.00}{##1}}}
\expandafter\def\csname PY@tok@kp\endcsname{\def\PY@tc##1{\textcolor[rgb]{0.00,0.50,0.00}{##1}}}
\expandafter\def\csname PY@tok@kt\endcsname{\def\PY@tc##1{\textcolor[rgb]{0.69,0.00,0.25}{##1}}}
\expandafter\def\csname PY@tok@o\endcsname{\def\PY@tc##1{\textcolor[rgb]{0.40,0.40,0.40}{##1}}}
\expandafter\def\csname PY@tok@ow\endcsname{\let\PY@bf=\textbf\def\PY@tc##1{\textcolor[rgb]{0.67,0.13,1.00}{##1}}}
\expandafter\def\csname PY@tok@nb\endcsname{\def\PY@tc##1{\textcolor[rgb]{0.00,0.50,0.00}{##1}}}
\expandafter\def\csname PY@tok@nf\endcsname{\def\PY@tc##1{\textcolor[rgb]{0.00,0.00,1.00}{##1}}}
\expandafter\def\csname PY@tok@nc\endcsname{\let\PY@bf=\textbf\def\PY@tc##1{\textcolor[rgb]{0.00,0.00,1.00}{##1}}}
\expandafter\def\csname PY@tok@nn\endcsname{\let\PY@bf=\textbf\def\PY@tc##1{\textcolor[rgb]{0.00,0.00,1.00}{##1}}}
\expandafter\def\csname PY@tok@ne\endcsname{\let\PY@bf=\textbf\def\PY@tc##1{\textcolor[rgb]{0.82,0.25,0.23}{##1}}}
\expandafter\def\csname PY@tok@nv\endcsname{\def\PY@tc##1{\textcolor[rgb]{0.10,0.09,0.49}{##1}}}
\expandafter\def\csname PY@tok@no\endcsname{\def\PY@tc##1{\textcolor[rgb]{0.53,0.00,0.00}{##1}}}
\expandafter\def\csname PY@tok@nl\endcsname{\def\PY@tc##1{\textcolor[rgb]{0.63,0.63,0.00}{##1}}}
\expandafter\def\csname PY@tok@ni\endcsname{\let\PY@bf=\textbf\def\PY@tc##1{\textcolor[rgb]{0.60,0.60,0.60}{##1}}}
\expandafter\def\csname PY@tok@na\endcsname{\def\PY@tc##1{\textcolor[rgb]{0.49,0.56,0.16}{##1}}}
\expandafter\def\csname PY@tok@nt\endcsname{\let\PY@bf=\textbf\def\PY@tc##1{\textcolor[rgb]{0.00,0.50,0.00}{##1}}}
\expandafter\def\csname PY@tok@nd\endcsname{\def\PY@tc##1{\textcolor[rgb]{0.67,0.13,1.00}{##1}}}
\expandafter\def\csname PY@tok@s\endcsname{\def\PY@tc##1{\textcolor[rgb]{0.73,0.13,0.13}{##1}}}
\expandafter\def\csname PY@tok@sd\endcsname{\let\PY@it=\textit\def\PY@tc##1{\textcolor[rgb]{0.73,0.13,0.13}{##1}}}
\expandafter\def\csname PY@tok@si\endcsname{\let\PY@bf=\textbf\def\PY@tc##1{\textcolor[rgb]{0.73,0.40,0.53}{##1}}}
\expandafter\def\csname PY@tok@se\endcsname{\let\PY@bf=\textbf\def\PY@tc##1{\textcolor[rgb]{0.73,0.40,0.13}{##1}}}
\expandafter\def\csname PY@tok@sr\endcsname{\def\PY@tc##1{\textcolor[rgb]{0.73,0.40,0.53}{##1}}}
\expandafter\def\csname PY@tok@ss\endcsname{\def\PY@tc##1{\textcolor[rgb]{0.10,0.09,0.49}{##1}}}
\expandafter\def\csname PY@tok@sx\endcsname{\def\PY@tc##1{\textcolor[rgb]{0.00,0.50,0.00}{##1}}}
\expandafter\def\csname PY@tok@m\endcsname{\def\PY@tc##1{\textcolor[rgb]{0.40,0.40,0.40}{##1}}}
\expandafter\def\csname PY@tok@gh\endcsname{\let\PY@bf=\textbf\def\PY@tc##1{\textcolor[rgb]{0.00,0.00,0.50}{##1}}}
\expandafter\def\csname PY@tok@gu\endcsname{\let\PY@bf=\textbf\def\PY@tc##1{\textcolor[rgb]{0.50,0.00,0.50}{##1}}}
\expandafter\def\csname PY@tok@gd\endcsname{\def\PY@tc##1{\textcolor[rgb]{0.63,0.00,0.00}{##1}}}
\expandafter\def\csname PY@tok@gi\endcsname{\def\PY@tc##1{\textcolor[rgb]{0.00,0.63,0.00}{##1}}}
\expandafter\def\csname PY@tok@gr\endcsname{\def\PY@tc##1{\textcolor[rgb]{1.00,0.00,0.00}{##1}}}
\expandafter\def\csname PY@tok@ge\endcsname{\let\PY@it=\textit}
\expandafter\def\csname PY@tok@gs\endcsname{\let\PY@bf=\textbf}
\expandafter\def\csname PY@tok@gp\endcsname{\let\PY@bf=\textbf\def\PY@tc##1{\textcolor[rgb]{0.00,0.00,0.50}{##1}}}
\expandafter\def\csname PY@tok@go\endcsname{\def\PY@tc##1{\textcolor[rgb]{0.53,0.53,0.53}{##1}}}
\expandafter\def\csname PY@tok@gt\endcsname{\def\PY@tc##1{\textcolor[rgb]{0.00,0.27,0.87}{##1}}}
\expandafter\def\csname PY@tok@err\endcsname{\def\PY@bc##1{\setlength{\fboxsep}{0pt}\fcolorbox[rgb]{1.00,0.00,0.00}{1,1,1}{\strut ##1}}}
\expandafter\def\csname PY@tok@kc\endcsname{\let\PY@bf=\textbf\def\PY@tc##1{\textcolor[rgb]{0.00,0.50,0.00}{##1}}}
\expandafter\def\csname PY@tok@kd\endcsname{\let\PY@bf=\textbf\def\PY@tc##1{\textcolor[rgb]{0.00,0.50,0.00}{##1}}}
\expandafter\def\csname PY@tok@kn\endcsname{\let\PY@bf=\textbf\def\PY@tc##1{\textcolor[rgb]{0.00,0.50,0.00}{##1}}}
\expandafter\def\csname PY@tok@kr\endcsname{\let\PY@bf=\textbf\def\PY@tc##1{\textcolor[rgb]{0.00,0.50,0.00}{##1}}}
\expandafter\def\csname PY@tok@bp\endcsname{\def\PY@tc##1{\textcolor[rgb]{0.00,0.50,0.00}{##1}}}
\expandafter\def\csname PY@tok@fm\endcsname{\def\PY@tc##1{\textcolor[rgb]{0.00,0.00,1.00}{##1}}}
\expandafter\def\csname PY@tok@vc\endcsname{\def\PY@tc##1{\textcolor[rgb]{0.10,0.09,0.49}{##1}}}
\expandafter\def\csname PY@tok@vg\endcsname{\def\PY@tc##1{\textcolor[rgb]{0.10,0.09,0.49}{##1}}}
\expandafter\def\csname PY@tok@vi\endcsname{\def\PY@tc##1{\textcolor[rgb]{0.10,0.09,0.49}{##1}}}
\expandafter\def\csname PY@tok@vm\endcsname{\def\PY@tc##1{\textcolor[rgb]{0.10,0.09,0.49}{##1}}}
\expandafter\def\csname PY@tok@sa\endcsname{\def\PY@tc##1{\textcolor[rgb]{0.73,0.13,0.13}{##1}}}
\expandafter\def\csname PY@tok@sb\endcsname{\def\PY@tc##1{\textcolor[rgb]{0.73,0.13,0.13}{##1}}}
\expandafter\def\csname PY@tok@sc\endcsname{\def\PY@tc##1{\textcolor[rgb]{0.73,0.13,0.13}{##1}}}
\expandafter\def\csname PY@tok@dl\endcsname{\def\PY@tc##1{\textcolor[rgb]{0.73,0.13,0.13}{##1}}}
\expandafter\def\csname PY@tok@s2\endcsname{\def\PY@tc##1{\textcolor[rgb]{0.73,0.13,0.13}{##1}}}
\expandafter\def\csname PY@tok@sh\endcsname{\def\PY@tc##1{\textcolor[rgb]{0.73,0.13,0.13}{##1}}}
\expandafter\def\csname PY@tok@s1\endcsname{\def\PY@tc##1{\textcolor[rgb]{0.73,0.13,0.13}{##1}}}
\expandafter\def\csname PY@tok@mb\endcsname{\def\PY@tc##1{\textcolor[rgb]{0.40,0.40,0.40}{##1}}}
\expandafter\def\csname PY@tok@mf\endcsname{\def\PY@tc##1{\textcolor[rgb]{0.40,0.40,0.40}{##1}}}
\expandafter\def\csname PY@tok@mh\endcsname{\def\PY@tc##1{\textcolor[rgb]{0.40,0.40,0.40}{##1}}}
\expandafter\def\csname PY@tok@mi\endcsname{\def\PY@tc##1{\textcolor[rgb]{0.40,0.40,0.40}{##1}}}
\expandafter\def\csname PY@tok@il\endcsname{\def\PY@tc##1{\textcolor[rgb]{0.40,0.40,0.40}{##1}}}
\expandafter\def\csname PY@tok@mo\endcsname{\def\PY@tc##1{\textcolor[rgb]{0.40,0.40,0.40}{##1}}}
\expandafter\def\csname PY@tok@ch\endcsname{\let\PY@it=\textit\def\PY@tc##1{\textcolor[rgb]{0.25,0.50,0.50}{##1}}}
\expandafter\def\csname PY@tok@cm\endcsname{\let\PY@it=\textit\def\PY@tc##1{\textcolor[rgb]{0.25,0.50,0.50}{##1}}}
\expandafter\def\csname PY@tok@cpf\endcsname{\let\PY@it=\textit\def\PY@tc##1{\textcolor[rgb]{0.25,0.50,0.50}{##1}}}
\expandafter\def\csname PY@tok@c1\endcsname{\let\PY@it=\textit\def\PY@tc##1{\textcolor[rgb]{0.25,0.50,0.50}{##1}}}
\expandafter\def\csname PY@tok@cs\endcsname{\let\PY@it=\textit\def\PY@tc##1{\textcolor[rgb]{0.25,0.50,0.50}{##1}}}

\def\PYZbs{\char`\\}
\def\PYZus{\char`\_}
\def\PYZob{\char`\{}
\def\PYZcb{\char`\}}
\def\PYZca{\char`\^}
\def\PYZam{\char`\&}
\def\PYZlt{\char`\<}
\def\PYZgt{\char`\>}
\def\PYZsh{\char`\#}
\def\PYZpc{\char`\%}
\def\PYZdl{\char`\$}
\def\PYZhy{\char`\-}
\def\PYZsq{\char`\'}
\def\PYZdq{\char`\"}
\def\PYZti{\char`\~}
% for compatibility with earlier versions
\def\PYZat{@}
\def\PYZlb{[}
\def\PYZrb{]}
\makeatother


    % Exact colors from NB
    \definecolor{incolor}{rgb}{0.0, 0.0, 0.5}
    \definecolor{outcolor}{rgb}{0.545, 0.0, 0.0}



    
    % Prevent overflowing lines due to hard-to-break entities
    \sloppy 
    % Setup hyperref package
    \hypersetup{
      breaklinks=true,  % so long urls are correctly broken across lines
      colorlinks=true,
      urlcolor=urlcolor,
      linkcolor=linkcolor,
      citecolor=citecolor,
      }
    % Slightly bigger margins than the latex defaults
    
    \geometry{verbose,tmargin=1in,bmargin=1in,lmargin=1in,rmargin=1in}
    
    

    \begin{document}
    
    
    \maketitle
    
    

    
    \section*{I.2 Probabilidad y Álgebra de
Sucesos}\label{i.2-probabilidad-y-uxe1lgebra-de-sucesos}

\subsection*{Reflexiones sobre el concepto de
probabilidad}\label{reflexiones-sobre-el-concepto-de-probabilidad}

Llamamos \textbf{experimento aleatorio} a \emph{todo fenómeno del que
tenemos una incertidumbre en su resultado}, aun cuando conozcamos las
condiciones bajo las que acontece. Por el contrario, en un
\textbf{experimento determinista} podemos conocer el resultado de
antemano, si tenemos suficiente información sobre tales condiciones.

El fenómeno en cuestión puede ser \textbf{natural, sin posibilidad} de
ser realizado experimentalmente, por ejemplo, el lugar de impacto en la
superficie de la Tierra del próximo meteorito que no se desintegre en la
atmósfera (aleatorio) o la hora de nacimiento del sol en Las Palmas de
Gran Canaria el próximo 1 de enero (determinista), o un \textbf{ensayo
que podemos repetir}, por ejemplo la cara ganadora en el lanzamiento de
un dado (aleatorio) o el tiempo de caida de un peso en un tubo vacío
vertical de diez metros de altura (determinista).

    La Historia de la Ciencia ha mostrado una controversia en relación a la
existencia de experimentos aleatorios. Con suficiente información de las
condiciones experimentales y computadores potentes, ¿podríamos
determinar de forma determinista el resultado de cualquier experimento?
La \emph{Mecánica Cuántica} nos dice que hay una aleatoriedad
consustancial a la Naturaleza.

Sin embargo, para nuestros propósitos, añadiremos también experimentos
cuya completa determinación no resulta práctica tales como, por ejemplo,
el lanzamiento de un dado. Por tanto, también admitimos como causa de la
aleatoriedad un insuficiente conocimiento de las condiciones del
fenómeno.

    Cada vez que realizamos un experimento aleatorio obtenemos una
\textbf{observación} de la \textbf{variable estadística} o
\textbf{muestral} de interés. Como ya sabemos, denominamos
\textbf{espacio muestral} al \emph{conjunto de resultados posibles de un
experimento aleatorio}.

Que el resultado de un experimento aleatorio sea incierto, no quiere
decir que todos los resultados posibles del espacio muestral sean total
e igualmente inciertos. Por ejemplo, es más verosímil esperar lluvia en
Gran Canaria en un día de invierno que en uno de verano.

Aunque no podamos determinar por adelantado los resultados de un
experimento aleatorio, podemos saber mucho de ellos. Por ejemplo, si un
dado de seis caras no está trucado, sabemos que los resultados posibles
al lanzarlo son sus seis caras y que no tenemos razón para pensar que
saldrá una más que otra.

    \textbf{¿Cómo podemos obtener información de los resultados de un
experimento aleatorio?} Básicamente tenemos dos posibilidades para ello:

\begin{itemize}
\tightlist
\item
  La primera posibilidad es \textbf{aprender de la experiencia}: en
  verano nunca nieva en las playas canarias, como vemos año tras año.
  Sin embargo, vemos que en invierno nieva varias veces en las cumbres
  más elevadas de las islas, y, aunque no es frecuente, a veces lo hace
  incluso en otoño y en primavera.
\item
  Una posibilidad relacionada con la anterior es crear artificalmente la
  experiencia mediante la \textbf{experimentación}. Por ejemplo, podemos
  lanzar nosotros un dado para observar qué recurrrencias tenemos.
\item
  La segunda posibilidad es disponer de un \textbf{modelo teórico} que
  nos indique qué resultados son posibles y con qué frecuencia sucede
  cada uno. Ciertamente, si un dado no está trucado, podemos esperar que
  cada cara salga una sexta parte de las veces.
\end{itemize}

En la práctica, ambas posibilidades están relacionadas. El aprendizaje a
partir de la experiencia es la fase inductiva a partir de la cual
establecemos una hipótesis. Una vez contrastada mediante la
experimentación, tendremos una teoría que podremos utilizar de forma
deductiva.

    La \textbf{probabilidad} cuantifica el grado de certidumbre de cada
resultado posible del experimento. Se trata, por tanto, de una medida
asociada a cada resultado del espacio muestral: 
\begin{itemize}
\item Cuando un
\textbf{resultado es imposible}, esto es, se tiene certidumbre absoluta
de que no sucederá, se le asigna la \textbf{probabilidad 0}. 
\item Cuando un
\textbf{resultado es seguro}, esto es, se tiene certidumbre absoluta de
que sucederá, se le asigna la \textbf{probabilidad 1}. 
\item Entre estos dos
situaciones extremas, se asignan \textbf{probabilidades entre 0 y 1} a
todos los resultados del espacio muestral, dependiendo de la menor o
mayor certidumbre que se tenga sobre los mismos.
\end{itemize}

\textbf{La medida de la certidumbre que proporciona la probabilidad
corresponde al modelo teórico a partir del cual podemos hacer
deducciones relativas a cómo se comportarán los resultados de un
experimento aleatorio. La construcción de tal modelo mediante la
observación corresponde a la estadística}.

    \paragraph{Relación entre población estadística y
probabilidad}\label{relaciuxf3n-entre-poblaciuxf3n-estaduxedstica-y-probabilidad}

En la fase inductiva, de aprendizaje mediante la experiencia de un
modelo, contamos con observaciones consistentes en una
\textbf{subpoblación muestral} de la población estadística que estamos
interesados en conocer: los resultados que obtendríamos si repitiéramos
infinitas veces el experimento, recorriendo al hacerlo todas las
posibilidades experimentales sin favorecer unas en relación a otras.
Lógicamente, esperamos que la subpoblación estadística sea
representativa del conjunto de la población.

Realmente tal población estadística es una abstracción y proporciona el
vínculo entre la estadística y la probabilidad. Aprender las
características de la población estadística significa inducir una
hipótesis sobre un modelo matemático (probabilístico) que representa el
comportamiento de las observaciones si se repitiera infinitamente el
experimento.

Una vez disponemos un modelo probabilístico adecuadamente contrastado, es
decir, que representa el comportamiento de los experimentos aleatorios,
no necesitamos continuar realizando experimentos para hacer predicciones
sobre nuevos resultados experimentales. Tan solo es necesario realizar
deducciones a partir de la teoría que hemos construido.

    \subsection*{Álgebra de sucesos}\label{uxe1lgebra-de-sucesos}

Un \textbf{conjunto} \(A\) es una reunión de objetos distintos
\(\{a_i\}\) llamados \textbf{elementos}. El número de elementos puede
ser infinito. 
\begin{itemize}
\item El conjunto sin elementos se llama \textbf{vacío}:
\(\emptyset\) 
\item El conjunto de todos los elementos de interés, sin dejar
ninguno fuera se llama \textbf{universal}: \(\Omega\). 
\item Si \(a\) es un
elemento de \(A\) decimos que \(a\) \textbf{pertenece} a \(A\):
\(a \in A\). En caso contrario diremos que \textbf{no pertenece}:
\(a \notin A\). 
\item \(B\) es un \textbf{subconjunto} de \(A\), o \(B\)
está \textbf{incluido} en \(A\), si todos los elementos de \(B\) también
lo son de \(A\): \(A\subset B\). En caso contrario, diremos que
\textbf{no está incluido}. 
\item Dos conjuntos \(A\) y \(B\) son iguales,
\(A = B\), si \(A\subset B\) y \(B \subset A\).
\end{itemize}

    \textbf{El espacio muestral \(\Omega\) es el conjunto cuyos elementos
son todos los resultados posibles de un experimento aleatorio}. Por
tanto, si \(x\) es un resultado posible del experimento aleatorio:
\(x \in S\).

Llamamos \textbf{suceso} a cualquier conjunto \(A\) de resultados
posibles de un experimento aleatorio. 
\begin{itemize}
\item \textbf{Un suceso acontece
siempre que se satisfaga uno de sus elementos}. 
\item Todo suceso está
incluido en el espacio muestral: \(A\subset \Omega\). 
\item Un suceso con un
único resultado se llama \textbf{suceso elemental}. 
\item El espacio
muestral se llama \textbf{suceso seguro} pues acontece siempre. 
\item El
suceso vacío \(\emptyset\) no acontece nunca y se llama, por ello,
\textbf{suceso imposible}.
\end{itemize}

    Consideremos el experimento aleatorio \emph{lanzar un dado de seis
caras} identificadas por las letras 'a' hasta 'f': 
\begin{itemize}
\item El espacio muestral
es \(\Omega = \{a, b, c, d, e, f\}\). Es el \textbf{suceso seguro}, pues
siempre saldrá uno de sus elementos. 
\item El suceso \(A = \{b, c, e\}\)
corresponde a que salga 'b', 'c' o 'e': \(b, c, e \in A\),
\(A \subset \Omega\) 
\item El suceso \(B = \{c, d\}\) corresponde a que
salga 'c' o 'd': \(c, d \in B\), \(B \subset \Omega\)
\end{itemize}

    \begin{center}
    \adjustimage{max size={0.9\linewidth}{0.9\paperheight}}{2.Probabilidad_Sucesos_new_files/2.Probabilidad_Sucesos_new_10_0.png}
    \end{center}
    { \hspace*{\fill} \\}
    
    \textbf{Operaciones} entre los subconjuntos del conjunto universal (o
entre los sucesos del espacio muestral): 
\begin{itemize}
\item \textbf{Complemento},
\(\overline A\): subconjunto de \(\Omega\) que no tiene elementos en
\(A\). Se cumple: 
\begin{itemize}
\item \(\overline \emptyset = \Omega\), 
\item
\(\overline \Omega = \emptyset\). 
\item \textbf{Diferencia}, \(A-B\):
conjunto formado por los elementos de \(A\) que no pertenecen a \(B\). 
\item
Adviértase que \(\overline A = \Omega - A\).
\end{itemize}

\tightlist
\item
  \textbf{Unión}, \(A \cup B\): conjunto formado por elementos que
  pertenecen a \(A\) \textbf{o} a \(B\). Se cumple:
 \begin{itemize}
\item
  Propiedad conmutativa, \(A \cup B = B \cup A\)
\item
  Propiedad asociativa, \((A \cup B) \cup C = A \cup (B\cup C)\)
\item
  \(A\cup\emptyset = A\)
\item
  \(A\cup\Omega=\Omega\)
\end{itemize}

\item
  \textbf{Intersección}, \(A \cap B\): conjunto formado por elementos
  que simultáneamente pertenecen a \(A\) \textbf{y} a \(B\). Se cumple:
\begin{itemize}
\item
  Propiedad conmutativa, \(A \cap B = B \cap A\)
\item
  Propiedad asociativa, \((A \cap B) \cap C = A \cap (B\cap C)\)
\item
  \(A\cap\emptyset = \emptyset\)
\item
  \(A\cap\Omega=A\)
\item
  Si \(A\cap B = \emptyset\) los \textbf{sucesos son incompatibles}.
\end{itemize}
\end{itemize}
    
    Otras propiedades de las operaciones conjuntistas:

\begin{itemize}
\tightlist
\item
  \textbf{Propiedades distributivas}
\begin{itemize}
\item
  \(A\cup(B\cap C) = (A\cup B)\cap(A\cup C)\)
\item
  \(A\cap(B\cup C) = (A\cap B)\cup(A\cap C)\)
\end{itemize}
\item
  \textbf{Leyes de Morgan}
\begin{itemize}
\item
  \(\overline{A\cup B} = \overline{A}\cap \overline{B}\)
\item
  \(\overline{A\cap B} = \overline{A}\cup \overline{B}\)
\end{itemize}
\item
  \textbf{Principio de dualidad}: si en una igualdad conjuntista se
  reemplazan la uniones por intersecciones, las intersecciones por
  uniones, \(\Omega\) por \(\emptyset\) y \(\emptyset\) por \(\Omega\),
  la igualdad se mantiene.
\end{itemize}

    \begin{center}
    \adjustimage{max size={0.9\linewidth}{0.9\paperheight}}{2.Probabilidad_Sucesos_new_files/2.Probabilidad_Sucesos_new_14_0.png}
    \end{center}
    { \hspace*{\fill} \\}
    
    \begin{center}
    \adjustimage{max size={0.9\linewidth}{0.9\paperheight}}{2.Probabilidad_Sucesos_new_files/2.Probabilidad_Sucesos_new_15_0.png}
    \end{center}
    { \hspace*{\fill} \\}
    
    Una \textbf{partición} \(\mathscr{P}\) del conjunto universal (en
nuestro caso, el suceso seguro) lo reparte en varios subconjuntos
disjuntos (sucesos incompatibles) que, al unirse, reconstruye el
conjunto universal.

\[A_i \in \mathscr{P}(\Omega) \, i=1\ldots K \iff  \begin{matrix}
  \bigcup_{i=1}^K A_i = \Omega &  \\
  A_i \cap A_j = \emptyset & i\neq j  
 \end{matrix}\]

    \begin{center}
    \adjustimage{max size={0.9\linewidth}{0.9\paperheight}}{2.Probabilidad_Sucesos_new_files/2.Probabilidad_Sucesos_new_17_0.png}
    \end{center}
    { \hspace*{\fill} \\}
    
    \subsection*{Definiciones de
probabilidad}\label{definiciones-de-probabilidad}

Dependiendo de la estrategia que se utilice para medición de la
incertidumbre, tendremos diferentes \textbf{interpretaciones de
probabilidad} que dan lugar a distintas definiciones:

\begin{itemize}
\tightlist
\item
  \textbf{Definición frecuencial de probabilidad}
\item
  \textbf{Definición clásica de probabilidad}
\item
  \textbf{Definición axiomática de probabilidad}
\end{itemize}

    \subsubsection*{Definición frecuencial de
probabilidad}\label{definiciuxf3n-frecuencial-de-probabilidad}

La \textbf{interpretación frecuencial de la probabilidad} nos da la
primera definición, que se fundamenta en conceptos estadísticos. La
probabilidad de un suceso se estima experimentalmente, repitiendo muchas
veces el mismo experimento y calculando sus \textbf{frecuencias
relativas}. La definición matemática de \textbf{probabilidad muestral}
requiere repetir infinitas veces el experimento, asegurándonos de
muestrear el espacio muestral sin beneficiar a unos resultados en
relación a otros:

\[P(A) = \lim_{N \to \infty}\frac{N_A}{N}\; ,\]

donde \(N_A\) es el número de veces que acontece el suceso \(A\) y \(N\)
es el número de veces que se repite el experimento.

    En la práctica no podemos pasar al límite, y nos debemos conformar con
una \textbf{estimación de la probabilidad muestral} de cada resultado
posible, esto es, de cada elemento del espacio muestral. Tal estimación
será mejor cuanto mayor sea el número de veces que repetimos el
experimento.

En particular, estamos interesados en conocer las frecuencias relativas,
y, en el límite, las probabilidades, de todos y cada uno de los sucesos
elementales, esto es, de todos y cada uno de los resultados que componen
del espacio muestral. Las frecuencias relativas y, con ello, las
probabilidades de todos sucesos se obtienen inmediatamente a partir de
las de los sucesos elementales del espacio muestral.

    Algunas \textbf{propiedades} obtenidas de la definición y del álgebra de
sucesos:
\begin{itemize}
 \item Dado que \(0\leq N_A \leq N\), siempre se cumple que
\(0\leq P(A)\leq 1\). 
\item El suceso imposible \(\emptyset\) nunca
acontece, por lo que \(N_\emptyset = 0\) y \(P(\emptyset)=0\). 
\item El
suceso seguro \(\Omega\) siempre acontece, por lo que \(N_\Omega =N\) y
\(P(\Omega)=1\). 
\item Si \(A\) acontece \(N_A\) veces su complementario,
\(\overline A\) acontece \(N_{\overline A}=N-N_A\) veces, por lo que
\(P(\overline A)=1-P(A)\).
\end{itemize}

    \begin{itemize}
\tightlist
\item
  Si dos sucesos, \(A\) y \(B\), no tienen resultados en común,
  \(A\cap B =\emptyset\), son \textbf{incompatibles} y no pueden
  verificarse al mismo tiempo, por lo que \(P(A\cap B)=0\). En este
  caso, el suceso \(A\cup B\) acontece cuando lo hace uno sólo de los
  dos, por lo que \(N_{A\cup B}=N_A + N_B\) y \(P(A\cup B)=P(A)+P(B)\).
  Esto es:
  \[P(A\cap B)=0 \implies P(A\cup B) = \lim_{N\to \infty}\frac{N_A+N_B}{N}=P(A)+P(B)\]
\end{itemize}

    \begin{itemize}
\tightlist
\item
  Si dos sucesos, \(A\) y \(B\), tienen resultados en común,
  \(A\cap B \neq\emptyset\), pueden verificarse al mismo tiempo, por lo
  que \(N_{A\cup B}=N_A + N_B - N_{A\cap B}\), pues debemos evitar
  contar dos veces los resultados de la intersección, y
  \(P(A\cup B)=P(A)+P(B)-P(A\cap B)\). Esto es:
  \[P(A\cap B)\neq 0 \implies P(A\cup B) = \lim_{N\to \infty}\frac{N_A+N_B-N_{A\cap B}}{N}=P(A)+P(B)-P(A\cap B)\]
\end{itemize}

    \paragraph{Experimento: estimación de probabilidades a partir de
probabilidades
muestrales}\label{experimento-estimaciuxf3n-de-probabilidades-a-partir-de-probabilidades-muestrales}

Lanzamos un \textbf{dado bueno} y otro \textbf{trucado}, éste con la
característica de que pesan más las caras 5 y 6 y que, por lo tanto,
saldrán más veces. ¿Cuáles serán las probabilidades de cada cara para
cada dado? 
\begin{itemize}
\item Dado bueno: \(P(1)=P(2)=P(3)=P(4)=P(5)=P(6)=\frac{1}{6}\) 
\item Dado trucado: 
\begin{itemize}
\item \(P(1)=P(2)=P(3)=P(4)=\frac{1}{12}\) 
\item \(P(5)=P(6)=\frac{1}{3}\)
\end{itemize} 
\item En ambos casos:
\(P(1)+P(2)+P(3)+P(4)+P(5)+P(6)=1\)
\end{itemize}

    \begin{Verbatim}[commandchars=\\\{\}]
Probabilidades caras dado bueno:
 [0.16666667 0.16666667 0.16666667 0.16666667 0.16666667 0.16666667]
Suma de probabilidades: 1.0
Probabilidades caras dado trucado:
 [0.08333333 0.08333333 0.08333333 0.08333333 0.33333333 0.33333333]
Suma de probabilidades: 1.0

    \end{Verbatim}

    \begin{itemize}
\tightlist
\item
  \textbf{Un lanzamiento}:
\end{itemize}

    \begin{Verbatim}[commandchars=\\\{\}]
Dado bueno (1 lanzamiento):
 [3]
Dado trucado (1 lanzamiento):
 [6]

    \end{Verbatim}

    \begin{center}
    \adjustimage{max size={0.9\linewidth}{0.9\paperheight}}{2.Probabilidad_Sucesos_new_files/2.Probabilidad_Sucesos_new_29_0.png}
    \end{center}
    { \hspace*{\fill} \\}
    
    Un único lanzamiento no nos permite extraer ninguna conclusión ya que la
probabilidad muestral resulta 1 para el resultado obtenido, y cero para
todos los demás. Debemos repetir el ensayo varias veces...

    Veamos qué pasa con \(100\), \(1000\) y \(1000000\) de repeticiones,
\textbf{reflexionando sobre los resultados y cómo las probabilidades
muestrales obtenidas estiman las probabilidades teóricas}. Cada sucesión
de lanzamientos se modela mediante una \textbf{secuencia discreta}. Como
veremos, \textbf{según aumente el número de repeticiones del ensayo
mejora la estimación las probabilidades muestrales obtenidas}.
Teóricamente, necesitamos infinitas repeticiones para tener certeza en
la estimación de las probabilidades.

\begin{itemize}
\tightlist
\item
  \textbf{Cien lanzamientos}:
\end{itemize}

    \begin{Verbatim}[commandchars=\\\{\}]
Dado bueno (100 lanzamientos):
 [1 6 4 4 6 1 5 3 5 2 1 6 2 3 4 2 6 3 6 2 3 4 2 6 1 3 4 4 5 5 2 2 4 2 2 3 4
 1 1 4 3 6 3 2 3 6 1 2 2 1 6 2 5 2 6 4 2 5 4 3 2 3 1 3 5 4 4 5 6 2 4 3 4 6
 5 2 1 6 3 1 2 5 1 2 2 2 5 6 6 6 2 1 2 6 6 3 1 5 6 1]
Dado trucado (100 lanzamientos):
 [5 6 5 2 5 6 5 6 5 1 5 3 2 3 6 6 5 5 5 4 6 6 4 2 2 5 2 6 5 4 1 4 5 5 5 5 5
 1 4 6 4 6 6 5 5 5 5 6 5 6 5 1 5 6 4 6 6 5 5 2 1 2 5 5 5 6 3 5 6 3 1 4 3 6
 5 5 5 5 3 5 5 5 4 1 5 2 3 5 5 1 4 5 4 6 6 6 6 5 6 6]

    \end{Verbatim}

    \begin{center}
    \adjustimage{max size={0.9\linewidth}{0.9\paperheight}}{2.Probabilidad_Sucesos_new_files/2.Probabilidad_Sucesos_new_33_0.png}
    \end{center}
    { \hspace*{\fill} \\}
    
    \begin{center}
    \adjustimage{max size={0.9\linewidth}{0.9\paperheight}}{2.Probabilidad_Sucesos_new_files/2.Probabilidad_Sucesos_new_34_0.png}
    \end{center}
    { \hspace*{\fill} \\}
    
    \begin{Verbatim}[commandchars=\\\{\}]
Probabilidades muestrales dado bueno: 
	 [0.15 0.24 0.15 0.15 0.12 0.19]
Probabilidades muestrales dado trucado:
	 [0.08 0.08 0.07 0.11 0.41 0.25]

    \end{Verbatim}

    \begin{itemize}
\tightlist
\item
  \textbf{\emph{Mil lanzamientos}}
\end{itemize}

    \begin{center}
    \adjustimage{max size={0.9\linewidth}{0.9\paperheight}}{2.Probabilidad_Sucesos_new_files/2.Probabilidad_Sucesos_new_37_0.png}
    \end{center}
    { \hspace*{\fill} \\}
    
    \begin{Verbatim}[commandchars=\\\{\}]
Probabilidades muestrales dado bueno: 
	 [0.167 0.178 0.179 0.152 0.175 0.149]
Probabilidades muestrales dado trucado:
	 [0.08  0.086 0.086 0.077 0.352 0.319]

    \end{Verbatim}

    \begin{itemize}
\tightlist
\item
  \textbf{Un millón de lanzamientos}
\end{itemize}

    \begin{center}
    \adjustimage{max size={0.9\linewidth}{0.9\paperheight}}{2.Probabilidad_Sucesos_new_files/2.Probabilidad_Sucesos_new_40_0.png}
    \end{center}
    { \hspace*{\fill} \\}
    
    \begin{Verbatim}[commandchars=\\\{\}]
Probabilidades muestrales dado bueno: 
	 [0.16663  0.166231 0.167139 0.166776 0.166667 0.166557]
Probabilidades muestrales dado trucado:
	 [0.083198 0.083626 0.08349  0.083316 0.333425 0.332945]

    \end{Verbatim}

    \subsubsection*{Definición clásica de
probabilidad}\label{definiciuxf3n-cluxe1sica-probabilidad}

La definición frecuencial de probabilidad se ajusta a nuestra
experiencia, y proporciona un vínculo entre la estadística y la
probabilidad mediante las frecuencias relativas de los resultados
experimentales. Sin embargo, no siempre podemos repetir los experimentos
suficientes veces para poder hacer una estimación aceptable de las
probabilidades muestrales. Además, si el número de resultados posibles
es infinito, no podremos recorrer todo el espacio muestral en nuestros
experimentos repetidos y tendremos que conformarnos con estimaciones de
probabilidad muestral a partir de intervalos.

Sin embargo, nuestro conocimiento del problema, en algunos casos, puede
permitirnos establecer un modelo teórico para las probabilidades, sin
necesidad de experimentación.

Por ejemplo, al lanzar un dado bueno de seis caras hay seis resultados
posibles, y la probabilidad de que salga cada una de ellas es de 1/6. No
es necesario recurrir a la experimentación para saberlo.

    La definición clásica de probabilidad aborda parcialmente los problemas
mencionados, calculando analíticamente (sin experimentación) las
probabilidades de cada resultado. En particular presupone 
\begin{itemize}
\item Un número
finito de resultados experimentales en el espacio muestral \(N\). 
\item Equiprobabilidad de todos los sucesos elementales:
\(P_1 = P_2 \ldots = P_N = \frac{1}{N}\).
\end{itemize}

La probabilidad de un suceso es el cociente de las posibilidades
teóricas de que acontezca, \(N_A\) y de todos los resultados posibles
\(N\):

\[P(A) = \frac{N_A}{N}\]

    Por ejemplo, la probabilidad de que resulte par el lanzamiento de un
dado es 1/2, pues de seis resultados posibles (1 a 6) tan sólo tres (2,
4, 6) nos valen: \(P(par) = \frac{3}{6}=\frac{1}{2}\). \textbf{Esta
definición de probabilidad satisface las propiedades derivadas del
álgebra de sucesos, ya vistas para la probabilidad muestral}.

Las \emph{limitaciones de la definición clásica} son muchas, pues los
resultados elementales no suelen ser equiprobables y el espacio muestral
puede ser infinito. No obstante, la idea de modelar teóricamente la
probabilidad a partir del conocimiento del problema se extiende a
situaciones muy complejas, más allá de la definición clásica.

    \subsubsection*{Definición axiomática de
probabilidad}\label{definiciuxf3n-axiomuxe1tica-de-probabilidad}

En este caso se introduce una estructura matemática, el \textbf{espacio
de probabilidad}, que formaliza teóricamente el experimento aleatorio en
el que estamos interesados. Llamamos \textbf{espacio de probabilidad} a
la terna \((\Omega, \mathscr{F}, P)\) donde:

\begin{itemize}
\tightlist
\item
  \(\Omega\) es el \textbf{espacio muestral} o conjunto de todos los
  resultados posibles del experimento aleatorio.
\item
  \(\mathscr{F}\) es el \textbf{conjunto de todos los sucesos} de
  interés. Es un \emph{conjunto de conjuntos}, pues todo suceso es un
  conjunto de resultados contenidos en \(\Omega\).
\item
  \(\mathscr{F}\) debe contener a \(\emptyset\) y \(\Omega\).
\item
  \(\mathscr{F}\) debe contener el resultado de cualquier operación
  conjuntista de sus elementos. Por ejemplo, si
  \(A, B \subset \mathscr{F} \implies A \cup B \subset \mathscr{F}\)
\item
  \textbf{P es una ley que asigna probabilidades a los sucesos} de
  \(\mathscr{F}\). Si \(A, B \subset \mathscr{F}\), \(P(A), P(B)\) son
  las probabilidades de \(A\) y de \(B\).
\end{itemize}

    Se establecen unos \textbf{axiomas} tales que el cálculo de
probabilidades sobre sucesos que se combinan mediante operaciones
conjuntistas satisface las mismas propiedades que con la definición
frecuencial y clásica: 
\begin{enumerate}
\item \(P(A), P(B)\geq 0\) 
\item \(P(\Omega)=1\) 
\item \(P(A \cup B) = P(A)+P(B)\) si \(A \cap B = \emptyset\)
\end{enumerate}

De los axiomas podemos derivar otras \textbf{propiedades} de la
probabilidad utilizando operaciones conjuntistas: 
\begin{itemize}
\item
\(P(\overline A) = 1 - P(A)\). 
\item \(P(\emptyset) = 0\). 
\item Si
\(A\cap B \neq 0\implies P(A\cup B) =P(A)+P(B)-P(A\cap B)\). 
\item Si
\(A\subset B \implies P(A) \leq P(B)\).
\end{itemize}

    \subsection*{Experimento aleatorio: lanzar un
dado}\label{experimento-aleatorio-lanzar-un-dado}

Aplicaremos ahora algunos de los conceptos vistos a dos casos sencillos
pero muy ilustrativos: los lanzamientos de un dado normal y el de un
dado trucado. El dado trucado es el que ya hemos visto anteriormente

\begin{itemize}
\tightlist
\item
  \textbf{Espacio muestral}: 1, 2, 3, 4, 5, 6
\item
  \textbf{Probabilidades elementales}:
\item
  Normal: P(1) = P(2) = P(3) = P(4) = P(5) = P(6) = 1/6
\item
  Trucado: P(1) = P(2) = P(3) = P(4) = 1/12 ; P(5) = P(6) = 1/3
\item
  \textbf{Sucesos}: son todos los subconjuntos que se puedan formar con
  los elementos del espacio muestral. A partir de la probabilidades
  elementales se asignan probabilidades a cada suceso conforme a las
  leyes conjuntistas de la probabilidad. El suceso acontece si se
  verifica cualquiera de sus elementos.
\end{itemize}

Siendo el espacio muestral finito de dimensión N=6, hay 2N = 26 = 64
sucesos posibles, que codificamos con palabras binarias de 6 bits. El
bit más significativo corresponde al 1 y el menos al 6. Por ejemplo, el
suceso "ha salido un número par, esto es, 2, 4 ó 6, se codifica como
{[}0, 1, 0, 1, 0, 1{]}.

    \begin{Verbatim}[commandchars=\\\{\}]
Probabilidades elementales del espacio muestral:

Probabilidad dado bueno( [1 0 0 0 0 0] uno ) -> 0.16666666666666666
Probabilidad dado trucado( [1 0 0 0 0 0] uno ) -> 0.0833333333 

Probabilidad dado bueno( [0 1 0 0 0 0] dos ) -> 0.16666666666666666
Probabilidad dado trucado( [0 1 0 0 0 0] dos ) -> 0.0833333333 

Probabilidad dado bueno( [0 0 1 0 0 0] tres ) -> 0.16666666666666666
Probabilidad dado trucado( [0 0 1 0 0 0] tres ) -> 0.0833333333 

Probabilidad dado bueno( [0 0 0 1 0 0] cuatro ) -> 0.16666666666666666
Probabilidad dado trucado( [0 0 0 1 0 0] cuatro ) -> 0.0833333333 

Probabilidad dado bueno( [0 0 0 0 1 0] cinco ) -> 0.16666666666666666
Probabilidad dado trucado( [0 0 0 0 1 0] cinco ) -> 0.3333333333 

Probabilidad dado bueno( [0 0 0 0 0 1] seis ) -> 0.16666666666666666
Probabilidad dado trucado( [0 0 0 0 0 1] seis ) -> 0.33333333349999994 


    \end{Verbatim}

    \subsubsection{Aplicación de leyes conjuntistas de la
probabilidad}\label{aplicaciuxf3n-de-leyes-conjuntistas-de-la-probabilidad}

    Veamos algunos ejemplos sencillos:

\begin{itemize}
\tightlist
\item
  Probabilidad suceso nulo: \(P(\emptyset)=0\)
\item
  Probailidad seceso seguro: \(P(\Omega)=1\)
\item
  Probabilidad de que no salga 1:
\end{itemize}

\[P(\overline{\{1\}})=
\begin{cases}
    P\left(\{2,3,4,5,6\}\right)=P(\{2\})+P(\{3\})+P(\{4\})+P(\{5\})+P(\{6\})\\
    1-P(\{1\})
  \end{cases}
\]

    \begin{itemize}
\tightlist
\item
  Probabilidad de que salga par:
  \(P(par)=P\left(\{2,4,6\}\right)=P(\{2\})+P(\{4\})+P(\{6\})\)
\item
  Probabilidad de que salga impar:
  \(P(impar)=P\left(\{1,3,5\}\right)=P(\{1\})+P(\{3\})+P(\{5\})\)
\item
  \(par \cap impar = \emptyset \qquad par \cup impar = \Omega \implies par = \overline{impar} \iff P(par) = 1 - P(impar)\)
\item
  Probabilidad de que salga divisible por 3:
  \(P(div3)=P\left(\{,3,6\}\right)=P(\{3\})+P(\{6\})\)
\item
  Probabilidad de que salga no divisible por 3:
  \(P(nodiv3)=P\left(\{1,2,4,5\}\right)=P(\{1\})+P(\{2\})+P(\{4\})+P(\{5\})\)
\item
  \(div3 \cap nodiv3 = \emptyset \qquad div3 \cup nodiv3 = \Omega \implies div3 = \overline{nodiv3} \iff P(div3) = 1 - P(nodiv3)\)
\end{itemize}

    \begin{Verbatim}[commandchars=\\\{\}]
Dado bueno:
Probabilidad de que NO salga uno: 0.8333333333333334
Probabilidad de que salga dos, tres, cuatro, cinco o seis 0.8333333333333333

Dado trucado:
Probabilidad de que NO salga uno: 0.9166666667
Probabilidad de que salga dos, tres, cuatro, cinco o seis 0.9166666667

    \end{Verbatim}

    Comparemos con probabilidades muestrales, obtenidas con lanzamientos
repetidos:

    \begin{Verbatim}[commandchars=\\\{\}]
Dado bueno:
Probabilidad de que NO salga uno (100 repeticiones): 0.85
suceso -> 1,2,3,4,5,6 =  [False  True  True  True  True  True]
Probabilidad de que salga dos, tres, cuatro, cinco o seis (100 repeticiones) 0.85

Dado trucado:
Probabilidad de que NO salga uno (100 repeticiones): 0.92
suceso -> 1,2,3,4,5,6 =  [False  True  True  True  True  True]
Probabilidad de que salga dos, tres, cuatro, cinco o seis (100 repeticiones) 0.92

    \end{Verbatim}

    \textbf{Sucesos complementarios y particiones}

    \begin{Verbatim}[commandchars=\\\{\}]
Probabilidades sucesos singulares:

Probabilidad dado bueno( [0 0 0 0 0 0] nulo ) -> 0.0
Probabilidad dado trucado( [0 0 0 0 0 0] nulo ) -> 0.0 

Probabilidad dado bueno( [0 1 0 1 0 1] par ) -> 0.5
Probabilidad dado trucado( [0 1 0 1 0 1] par ) -> 0.5000000001 

Probabilidad dado bueno( [1 0 1 0 1 0] impar ) -> 0.5
Probabilidad dado trucado( [1 0 1 0 1 0] impar ) -> 0.4999999999 

Probabilidad dado bueno( [0 0 1 0 0 1] div3 ) -> 0.333333333333333
Probabilidad dado trucado( [0 0 1 0 0 1] div3 ) -> 0.4166666668 

Probabilidad dado bueno( [1 1 0 1 1 0] nodiv3 ) -> 0.666666666666667
Probabilidad dado trucado( [1 1 0 1 1 0] nodiv3 ) -> 0.5833333332 

Probabilidad dado bueno( [1 1 1 1 1 1] seguro ) -> 1.0
Probabilidad dado trucado( [1 1 1 1 1 1] seguro ) -> 1.0 


    \end{Verbatim}

    \textbf{Probabilidades muestrales (sólo dado bueno)}

    \begin{Verbatim}[commandchars=\\\{\}]
Suceso par -> ['dos' 'cuatro' 'seis'] -> Prob. muestral (100 repeticiones): 0.58

    \end{Verbatim}

    \begin{Verbatim}[commandchars=\\\{\}]
Suceso impar -> ['uno' 'tres' 'cinco'] 
	 Prob. muestral (100 repeticiones): 0.42

    \end{Verbatim}

    \begin{Verbatim}[commandchars=\\\{\}]
Suceso divisible por tres -> ['tres' 'seis'] 
	 Prob. muestral (100 repeticiones): 0.34

    \end{Verbatim}

    \begin{Verbatim}[commandchars=\\\{\}]
Suceso no divisible por tres -> ['uno' 'dos' 'cuatro' 'cinco'] 
	 Prob. muestral (100 repeticiones): 0.66

    \end{Verbatim}

    \textbf{Sucesos seguro y nulo}

    \begin{Verbatim}[commandchars=\\\{\}]
suceso nulo INTERSECCIÓN suceso seguro: 	 [0 0 0 0 0 0]
COMPLEMENTARIO suceso nulo == suceso seguro: 	 True
suceso nulo UNIÓN suceso seguro: 		 [1 1 1 1 1 1]

    \end{Verbatim}

    \textbf{Partición resultado par y resultado impar}

    \begin{Verbatim}[commandchars=\\\{\}]
suceso par INTERSECCIÓN suceso impar: 		 [0 0 0 0 0 0]
COMPLEMENTARIO suceso par == suceso impar: 	 True
suceso par UNIÓN suceso impar: 			 [1 1 1 1 1 1]

    \end{Verbatim}

    \textbf{Partición resultado divisible por tres y resultado no divisible
por tres}

    \begin{Verbatim}[commandchars=\\\{\}]
suceso divisible por tres INTERSECCIÓN suceso no divisible por tres:	 [0 0 0 0 0 0]
COMPLEMENTARIO suceso divisible por tres == suceso no divisible por tres: 	 True
suceso divisible por tres UNIÓN suceso no divisible por tres: 		 [1 1 1 1 1 1]

    \end{Verbatim}

    \textbf{Unión e intersección de sucesos}

Sea el suceso \(A=\{3,4,5\}\). El suceso complementario es
\(\overline A=\{1.2.6\}\). Por tanto \(P(\overline A) = 1-P(A)\).

\begin{itemize}
\tightlist
\item
  \(P(A\cup par)=P(\{2,3,4,5,6\})=1-P(\{1\})\)
\item
  \(P(A\cap par)=P(\{4\})\)
\item
  \(P(A\cup impar)=P(\{1,3,4,5\})=P(\{1\})+P(\{3\})+P(\{4\})+P(\{5\})\)
\item
  \(P(A\cap impar)=P(\{3,5\})=P(\{3\})+P(\{5\})\)
\item
  \(P(A\cup div3)=P(\{3,4,5,6\})=P(\{3\})+P(\{4\})+P(\{5\})+P(\{6\})\)
\item
  \(P(A\cap div3)=P(\{3\})\)
\item
  \(P(A\cup nodiv3)=P(\{1,2,3,4,5\})=P(\{1\})+P(\{2\})+P(\{3\})+P(\{4\})+P(\{5\})\)
\item
  \(P(A\cap nodiv3)=P(\{4,5\})=P(\{4\})+P(\{5\})\)
\end{itemize}

    En lo que sigue se trabaja \textbf{solo} con el \textbf{dado bueno}...

    \begin{Verbatim}[commandchars=\\\{\}]
Probabilidad de que salga tres o cuatro o cinco:	 0.5
Probabilidad de que NO salga tres o cuatro o cinco:	 0.5
Probabilidad de que salga uno, dos o seis:		 0.5

    \end{Verbatim}

    \begin{Verbatim}[commandchars=\\\{\}]
Suceso tres, cuatro o cinco -> ['tres' 'cuatro' 'cinco'] 
	 Prob. muestral (100 repeticiones): 0.42

    \end{Verbatim}

    \begin{Verbatim}[commandchars=\\\{\}]
Suceso que salga par UNIÓN que salga tres o cuatro o cinco [0 1 1 1 1 1] 
	Probabilidad: 0.833333333333333

Suceso que salga par INTERSECCIÓN que salga tres o cuatro o cinco [0 0 0 1 0 0] 
	Probabilidad: 0.166666666666667

Probabilidad de que salga par ( 0.5 ) + 
Probabilidad de que salga tres o cuatro o cinco ( 0.5 ) - 
Probabilidad intersección ( 0.166666666666667 ) = 
	= 0.833333333333333

    \end{Verbatim}

    \begin{Verbatim}[commandchars=\\\{\}]
Suceso que salga impar UNIÓN que salga tres o cuatro o cinco [1 0 1 1 1 0] 
	Probabilidad: 0.6666666666666666

Suceso que salga impar INTERSECCIÓN que salga tres o cuatro o cinco [0 0 1 0 1 0] 
	Probabilidad: 0.3333333333333333

Probabilidad de que salga impar ( 0.5 ) + 
Probabilidad de que salga tres o cuatro o cinco ( 0.5 ) - 
Probabilidad intersección ( 0.3333333333333333 ) = 
	= 0.6666666666666667

    \end{Verbatim}

    \begin{Verbatim}[commandchars=\\\{\}]
Suceso que salga divisible por tres UNIÓN que salga tres o cuatro o cinco [0 0 1 1 1 1] 
	Probabilidad: 0.666666666666667

Suceso que salga divisible por 3 INTERSECCIÓN que salga tres o cuatro o cinco [0 0 1 0 0 0] 
	Probabilidad: 0.166666666666667

Probabilidad de que salga divisible por tres ( 0.333333333333333 ) + 
Probabilidad de que salga tres o cuatro o cinco ( 0.5 ) - 
Probabilidad intersección ( 0.166666666666667 ) = 
	= 0.6666666666666661

    \end{Verbatim}

    \begin{Verbatim}[commandchars=\\\{\}]
Suceso que salga no divisible por tres UNIÓN que salga tres o cuatro o cinco [1 1 1 1 1 0] 
	Probabilidad: 0.8333333333333333

Suceso que salga no divisible por 3 INTERSECCIÓN que salga tres o cuatro o cinco [0 0 0 1 1 0] 
	Probabilidad: 0.3333333333333333

Probabilidad de que salga no divisible por tres ( 0.666666666666667 ) + 
Probabilidad de que salga tres o cuatro o cinco ( 0.5 ) - 
Probabilidad intersección ( 0.3333333333333333 ) = 
	= 0.8333333333333337

    \end{Verbatim}


    % Add a bibliography block to the postdoc
    
    
    
    \end{document}
