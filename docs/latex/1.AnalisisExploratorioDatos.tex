
% Default to the notebook output style

    


% Inherit from the specified cell style.




    
\documentclass[11pt]{article}

    
    
    \usepackage[T1]{fontenc}
    % Nicer default font (+ math font) than Computer Modern for most use cases
    \usepackage{mathpazo}

    % Basic figure setup, for now with no caption control since it's done
    % automatically by Pandoc (which extracts ![](path) syntax from Markdown).
    \usepackage{graphicx}
    % We will generate all images so they have a width \maxwidth. This means
    % that they will get their normal width if they fit onto the page, but
    % are scaled down if they would overflow the margins.
    \makeatletter
    \def\maxwidth{\ifdim\Gin@nat@width>\linewidth\linewidth
    \else\Gin@nat@width\fi}
    \makeatother
    \let\Oldincludegraphics\includegraphics
    % Set max figure width to be 80% of text width, for now hardcoded.
    \renewcommand{\includegraphics}[1]{\Oldincludegraphics[width=.8\maxwidth]{#1}}
    % Ensure that by default, figures have no caption (until we provide a
    % proper Figure object with a Caption API and a way to capture that
    % in the conversion process - todo).
    \usepackage{caption}
    \DeclareCaptionLabelFormat{nolabel}{}
    \captionsetup{labelformat=nolabel}

    \usepackage{adjustbox} % Used to constrain images to a maximum size 
    \usepackage{xcolor} % Allow colors to be defined
    \usepackage{enumerate} % Needed for markdown enumerations to work
    \usepackage{geometry} % Used to adjust the document margins
    \usepackage{amsmath} % Equations
    \usepackage{amssymb} % Equations
    \usepackage{textcomp} % defines textquotesingle
    % Hack from http://tex.stackexchange.com/a/47451/13684:
    \AtBeginDocument{%
        \def\PYZsq{\textquotesingle}% Upright quotes in Pygmentized code
    }
    \usepackage{upquote} % Upright quotes for verbatim code
    \usepackage{eurosym} % defines \euro
    \usepackage[mathletters]{ucs} % Extended unicode (utf-8) support
    \usepackage[utf8x]{inputenc} % Allow utf-8 characters in the tex document
    \usepackage{fancyvrb} % verbatim replacement that allows latex
    \usepackage{grffile} % extends the file name processing of package graphics 
                         % to support a larger range 
    % The hyperref package gives us a pdf with properly built
    % internal navigation ('pdf bookmarks' for the table of contents,
    % internal cross-reference links, web links for URLs, etc.)
    \usepackage{hyperref}
    \usepackage{longtable} % longtable support required by pandoc >1.10
    \usepackage{booktabs}  % table support for pandoc > 1.12.2
    \usepackage[inline]{enumitem} % IRkernel/repr support (it uses the enumerate* environment)
    \usepackage[normalem]{ulem} % ulem is needed to support strikethroughs (\sout)
                                % normalem makes italics be italics, not underlines
    \usepackage{mathrsfs}
    

    
    
    % Colors for the hyperref package
    \definecolor{urlcolor}{rgb}{0,.145,.698}
    \definecolor{linkcolor}{rgb}{.71,0.21,0.01}
    \definecolor{citecolor}{rgb}{.12,.54,.11}

    % ANSI colors
    \definecolor{ansi-black}{HTML}{3E424D}
    \definecolor{ansi-black-intense}{HTML}{282C36}
    \definecolor{ansi-red}{HTML}{E75C58}
    \definecolor{ansi-red-intense}{HTML}{B22B31}
    \definecolor{ansi-green}{HTML}{00A250}
    \definecolor{ansi-green-intense}{HTML}{007427}
    \definecolor{ansi-yellow}{HTML}{DDB62B}
    \definecolor{ansi-yellow-intense}{HTML}{B27D12}
    \definecolor{ansi-blue}{HTML}{208FFB}
    \definecolor{ansi-blue-intense}{HTML}{0065CA}
    \definecolor{ansi-magenta}{HTML}{D160C4}
    \definecolor{ansi-magenta-intense}{HTML}{A03196}
    \definecolor{ansi-cyan}{HTML}{60C6C8}
    \definecolor{ansi-cyan-intense}{HTML}{258F8F}
    \definecolor{ansi-white}{HTML}{C5C1B4}
    \definecolor{ansi-white-intense}{HTML}{A1A6B2}
    \definecolor{ansi-default-inverse-fg}{HTML}{FFFFFF}
    \definecolor{ansi-default-inverse-bg}{HTML}{000000}

    % commands and environments needed by pandoc snippets
    % extracted from the output of `pandoc -s`
    \providecommand{\tightlist}{%
      \setlength{\itemsep}{0pt}\setlength{\parskip}{0pt}}
    \DefineVerbatimEnvironment{Highlighting}{Verbatim}{commandchars=\\\{\}}
    % Add ',fontsize=\small' for more characters per line
    \newenvironment{Shaded}{}{}
    \newcommand{\KeywordTok}[1]{\textcolor[rgb]{0.00,0.44,0.13}{\textbf{{#1}}}}
    \newcommand{\DataTypeTok}[1]{\textcolor[rgb]{0.56,0.13,0.00}{{#1}}}
    \newcommand{\DecValTok}[1]{\textcolor[rgb]{0.25,0.63,0.44}{{#1}}}
    \newcommand{\BaseNTok}[1]{\textcolor[rgb]{0.25,0.63,0.44}{{#1}}}
    \newcommand{\FloatTok}[1]{\textcolor[rgb]{0.25,0.63,0.44}{{#1}}}
    \newcommand{\CharTok}[1]{\textcolor[rgb]{0.25,0.44,0.63}{{#1}}}
    \newcommand{\StringTok}[1]{\textcolor[rgb]{0.25,0.44,0.63}{{#1}}}
    \newcommand{\CommentTok}[1]{\textcolor[rgb]{0.38,0.63,0.69}{\textit{{#1}}}}
    \newcommand{\OtherTok}[1]{\textcolor[rgb]{0.00,0.44,0.13}{{#1}}}
    \newcommand{\AlertTok}[1]{\textcolor[rgb]{1.00,0.00,0.00}{\textbf{{#1}}}}
    \newcommand{\FunctionTok}[1]{\textcolor[rgb]{0.02,0.16,0.49}{{#1}}}
    \newcommand{\RegionMarkerTok}[1]{{#1}}
    \newcommand{\ErrorTok}[1]{\textcolor[rgb]{1.00,0.00,0.00}{\textbf{{#1}}}}
    \newcommand{\NormalTok}[1]{{#1}}
    
    % Additional commands for more recent versions of Pandoc
    \newcommand{\ConstantTok}[1]{\textcolor[rgb]{0.53,0.00,0.00}{{#1}}}
    \newcommand{\SpecialCharTok}[1]{\textcolor[rgb]{0.25,0.44,0.63}{{#1}}}
    \newcommand{\VerbatimStringTok}[1]{\textcolor[rgb]{0.25,0.44,0.63}{{#1}}}
    \newcommand{\SpecialStringTok}[1]{\textcolor[rgb]{0.73,0.40,0.53}{{#1}}}
    \newcommand{\ImportTok}[1]{{#1}}
    \newcommand{\DocumentationTok}[1]{\textcolor[rgb]{0.73,0.13,0.13}{\textit{{#1}}}}
    \newcommand{\AnnotationTok}[1]{\textcolor[rgb]{0.38,0.63,0.69}{\textbf{\textit{{#1}}}}}
    \newcommand{\CommentVarTok}[1]{\textcolor[rgb]{0.38,0.63,0.69}{\textbf{\textit{{#1}}}}}
    \newcommand{\VariableTok}[1]{\textcolor[rgb]{0.10,0.09,0.49}{{#1}}}
    \newcommand{\ControlFlowTok}[1]{\textcolor[rgb]{0.00,0.44,0.13}{\textbf{{#1}}}}
    \newcommand{\OperatorTok}[1]{\textcolor[rgb]{0.40,0.40,0.40}{{#1}}}
    \newcommand{\BuiltInTok}[1]{{#1}}
    \newcommand{\ExtensionTok}[1]{{#1}}
    \newcommand{\PreprocessorTok}[1]{\textcolor[rgb]{0.74,0.48,0.00}{{#1}}}
    \newcommand{\AttributeTok}[1]{\textcolor[rgb]{0.49,0.56,0.16}{{#1}}}
    \newcommand{\InformationTok}[1]{\textcolor[rgb]{0.38,0.63,0.69}{\textbf{\textit{{#1}}}}}
    \newcommand{\WarningTok}[1]{\textcolor[rgb]{0.38,0.63,0.69}{\textbf{\textit{{#1}}}}}
    
    
    % Define a nice break command that doesn't care if a line doesn't already
    % exist.
    \def\br{\hspace*{\fill} \\* }
    % Math Jax compatibility definitions
    \def\gt{>}
    \def\lt{<}
    \let\Oldtex\TeX
    \let\Oldlatex\LaTeX
    \renewcommand{\TeX}{\textrm{\Oldtex}}
    \renewcommand{\LaTeX}{\textrm{\Oldlatex}}
    % Document parameters
    % Document title
    \title{1.Análisis Exploratorio de Datos}
    
    
    
    
    

    % Pygments definitions
    
\makeatletter
\def\PY@reset{\let\PY@it=\relax \let\PY@bf=\relax%
    \let\PY@ul=\relax \let\PY@tc=\relax%
    \let\PY@bc=\relax \let\PY@ff=\relax}
\def\PY@tok#1{\csname PY@tok@#1\endcsname}
\def\PY@toks#1+{\ifx\relax#1\empty\else%
    \PY@tok{#1}\expandafter\PY@toks\fi}
\def\PY@do#1{\PY@bc{\PY@tc{\PY@ul{%
    \PY@it{\PY@bf{\PY@ff{#1}}}}}}}
\def\PY#1#2{\PY@reset\PY@toks#1+\relax+\PY@do{#2}}

\expandafter\def\csname PY@tok@w\endcsname{\def\PY@tc##1{\textcolor[rgb]{0.73,0.73,0.73}{##1}}}
\expandafter\def\csname PY@tok@c\endcsname{\let\PY@it=\textit\def\PY@tc##1{\textcolor[rgb]{0.25,0.50,0.50}{##1}}}
\expandafter\def\csname PY@tok@cp\endcsname{\def\PY@tc##1{\textcolor[rgb]{0.74,0.48,0.00}{##1}}}
\expandafter\def\csname PY@tok@k\endcsname{\let\PY@bf=\textbf\def\PY@tc##1{\textcolor[rgb]{0.00,0.50,0.00}{##1}}}
\expandafter\def\csname PY@tok@kp\endcsname{\def\PY@tc##1{\textcolor[rgb]{0.00,0.50,0.00}{##1}}}
\expandafter\def\csname PY@tok@kt\endcsname{\def\PY@tc##1{\textcolor[rgb]{0.69,0.00,0.25}{##1}}}
\expandafter\def\csname PY@tok@o\endcsname{\def\PY@tc##1{\textcolor[rgb]{0.40,0.40,0.40}{##1}}}
\expandafter\def\csname PY@tok@ow\endcsname{\let\PY@bf=\textbf\def\PY@tc##1{\textcolor[rgb]{0.67,0.13,1.00}{##1}}}
\expandafter\def\csname PY@tok@nb\endcsname{\def\PY@tc##1{\textcolor[rgb]{0.00,0.50,0.00}{##1}}}
\expandafter\def\csname PY@tok@nf\endcsname{\def\PY@tc##1{\textcolor[rgb]{0.00,0.00,1.00}{##1}}}
\expandafter\def\csname PY@tok@nc\endcsname{\let\PY@bf=\textbf\def\PY@tc##1{\textcolor[rgb]{0.00,0.00,1.00}{##1}}}
\expandafter\def\csname PY@tok@nn\endcsname{\let\PY@bf=\textbf\def\PY@tc##1{\textcolor[rgb]{0.00,0.00,1.00}{##1}}}
\expandafter\def\csname PY@tok@ne\endcsname{\let\PY@bf=\textbf\def\PY@tc##1{\textcolor[rgb]{0.82,0.25,0.23}{##1}}}
\expandafter\def\csname PY@tok@nv\endcsname{\def\PY@tc##1{\textcolor[rgb]{0.10,0.09,0.49}{##1}}}
\expandafter\def\csname PY@tok@no\endcsname{\def\PY@tc##1{\textcolor[rgb]{0.53,0.00,0.00}{##1}}}
\expandafter\def\csname PY@tok@nl\endcsname{\def\PY@tc##1{\textcolor[rgb]{0.63,0.63,0.00}{##1}}}
\expandafter\def\csname PY@tok@ni\endcsname{\let\PY@bf=\textbf\def\PY@tc##1{\textcolor[rgb]{0.60,0.60,0.60}{##1}}}
\expandafter\def\csname PY@tok@na\endcsname{\def\PY@tc##1{\textcolor[rgb]{0.49,0.56,0.16}{##1}}}
\expandafter\def\csname PY@tok@nt\endcsname{\let\PY@bf=\textbf\def\PY@tc##1{\textcolor[rgb]{0.00,0.50,0.00}{##1}}}
\expandafter\def\csname PY@tok@nd\endcsname{\def\PY@tc##1{\textcolor[rgb]{0.67,0.13,1.00}{##1}}}
\expandafter\def\csname PY@tok@s\endcsname{\def\PY@tc##1{\textcolor[rgb]{0.73,0.13,0.13}{##1}}}
\expandafter\def\csname PY@tok@sd\endcsname{\let\PY@it=\textit\def\PY@tc##1{\textcolor[rgb]{0.73,0.13,0.13}{##1}}}
\expandafter\def\csname PY@tok@si\endcsname{\let\PY@bf=\textbf\def\PY@tc##1{\textcolor[rgb]{0.73,0.40,0.53}{##1}}}
\expandafter\def\csname PY@tok@se\endcsname{\let\PY@bf=\textbf\def\PY@tc##1{\textcolor[rgb]{0.73,0.40,0.13}{##1}}}
\expandafter\def\csname PY@tok@sr\endcsname{\def\PY@tc##1{\textcolor[rgb]{0.73,0.40,0.53}{##1}}}
\expandafter\def\csname PY@tok@ss\endcsname{\def\PY@tc##1{\textcolor[rgb]{0.10,0.09,0.49}{##1}}}
\expandafter\def\csname PY@tok@sx\endcsname{\def\PY@tc##1{\textcolor[rgb]{0.00,0.50,0.00}{##1}}}
\expandafter\def\csname PY@tok@m\endcsname{\def\PY@tc##1{\textcolor[rgb]{0.40,0.40,0.40}{##1}}}
\expandafter\def\csname PY@tok@gh\endcsname{\let\PY@bf=\textbf\def\PY@tc##1{\textcolor[rgb]{0.00,0.00,0.50}{##1}}}
\expandafter\def\csname PY@tok@gu\endcsname{\let\PY@bf=\textbf\def\PY@tc##1{\textcolor[rgb]{0.50,0.00,0.50}{##1}}}
\expandafter\def\csname PY@tok@gd\endcsname{\def\PY@tc##1{\textcolor[rgb]{0.63,0.00,0.00}{##1}}}
\expandafter\def\csname PY@tok@gi\endcsname{\def\PY@tc##1{\textcolor[rgb]{0.00,0.63,0.00}{##1}}}
\expandafter\def\csname PY@tok@gr\endcsname{\def\PY@tc##1{\textcolor[rgb]{1.00,0.00,0.00}{##1}}}
\expandafter\def\csname PY@tok@ge\endcsname{\let\PY@it=\textit}
\expandafter\def\csname PY@tok@gs\endcsname{\let\PY@bf=\textbf}
\expandafter\def\csname PY@tok@gp\endcsname{\let\PY@bf=\textbf\def\PY@tc##1{\textcolor[rgb]{0.00,0.00,0.50}{##1}}}
\expandafter\def\csname PY@tok@go\endcsname{\def\PY@tc##1{\textcolor[rgb]{0.53,0.53,0.53}{##1}}}
\expandafter\def\csname PY@tok@gt\endcsname{\def\PY@tc##1{\textcolor[rgb]{0.00,0.27,0.87}{##1}}}
\expandafter\def\csname PY@tok@err\endcsname{\def\PY@bc##1{\setlength{\fboxsep}{0pt}\fcolorbox[rgb]{1.00,0.00,0.00}{1,1,1}{\strut ##1}}}
\expandafter\def\csname PY@tok@kc\endcsname{\let\PY@bf=\textbf\def\PY@tc##1{\textcolor[rgb]{0.00,0.50,0.00}{##1}}}
\expandafter\def\csname PY@tok@kd\endcsname{\let\PY@bf=\textbf\def\PY@tc##1{\textcolor[rgb]{0.00,0.50,0.00}{##1}}}
\expandafter\def\csname PY@tok@kn\endcsname{\let\PY@bf=\textbf\def\PY@tc##1{\textcolor[rgb]{0.00,0.50,0.00}{##1}}}
\expandafter\def\csname PY@tok@kr\endcsname{\let\PY@bf=\textbf\def\PY@tc##1{\textcolor[rgb]{0.00,0.50,0.00}{##1}}}
\expandafter\def\csname PY@tok@bp\endcsname{\def\PY@tc##1{\textcolor[rgb]{0.00,0.50,0.00}{##1}}}
\expandafter\def\csname PY@tok@fm\endcsname{\def\PY@tc##1{\textcolor[rgb]{0.00,0.00,1.00}{##1}}}
\expandafter\def\csname PY@tok@vc\endcsname{\def\PY@tc##1{\textcolor[rgb]{0.10,0.09,0.49}{##1}}}
\expandafter\def\csname PY@tok@vg\endcsname{\def\PY@tc##1{\textcolor[rgb]{0.10,0.09,0.49}{##1}}}
\expandafter\def\csname PY@tok@vi\endcsname{\def\PY@tc##1{\textcolor[rgb]{0.10,0.09,0.49}{##1}}}
\expandafter\def\csname PY@tok@vm\endcsname{\def\PY@tc##1{\textcolor[rgb]{0.10,0.09,0.49}{##1}}}
\expandafter\def\csname PY@tok@sa\endcsname{\def\PY@tc##1{\textcolor[rgb]{0.73,0.13,0.13}{##1}}}
\expandafter\def\csname PY@tok@sb\endcsname{\def\PY@tc##1{\textcolor[rgb]{0.73,0.13,0.13}{##1}}}
\expandafter\def\csname PY@tok@sc\endcsname{\def\PY@tc##1{\textcolor[rgb]{0.73,0.13,0.13}{##1}}}
\expandafter\def\csname PY@tok@dl\endcsname{\def\PY@tc##1{\textcolor[rgb]{0.73,0.13,0.13}{##1}}}
\expandafter\def\csname PY@tok@s2\endcsname{\def\PY@tc##1{\textcolor[rgb]{0.73,0.13,0.13}{##1}}}
\expandafter\def\csname PY@tok@sh\endcsname{\def\PY@tc##1{\textcolor[rgb]{0.73,0.13,0.13}{##1}}}
\expandafter\def\csname PY@tok@s1\endcsname{\def\PY@tc##1{\textcolor[rgb]{0.73,0.13,0.13}{##1}}}
\expandafter\def\csname PY@tok@mb\endcsname{\def\PY@tc##1{\textcolor[rgb]{0.40,0.40,0.40}{##1}}}
\expandafter\def\csname PY@tok@mf\endcsname{\def\PY@tc##1{\textcolor[rgb]{0.40,0.40,0.40}{##1}}}
\expandafter\def\csname PY@tok@mh\endcsname{\def\PY@tc##1{\textcolor[rgb]{0.40,0.40,0.40}{##1}}}
\expandafter\def\csname PY@tok@mi\endcsname{\def\PY@tc##1{\textcolor[rgb]{0.40,0.40,0.40}{##1}}}
\expandafter\def\csname PY@tok@il\endcsname{\def\PY@tc##1{\textcolor[rgb]{0.40,0.40,0.40}{##1}}}
\expandafter\def\csname PY@tok@mo\endcsname{\def\PY@tc##1{\textcolor[rgb]{0.40,0.40,0.40}{##1}}}
\expandafter\def\csname PY@tok@ch\endcsname{\let\PY@it=\textit\def\PY@tc##1{\textcolor[rgb]{0.25,0.50,0.50}{##1}}}
\expandafter\def\csname PY@tok@cm\endcsname{\let\PY@it=\textit\def\PY@tc##1{\textcolor[rgb]{0.25,0.50,0.50}{##1}}}
\expandafter\def\csname PY@tok@cpf\endcsname{\let\PY@it=\textit\def\PY@tc##1{\textcolor[rgb]{0.25,0.50,0.50}{##1}}}
\expandafter\def\csname PY@tok@c1\endcsname{\let\PY@it=\textit\def\PY@tc##1{\textcolor[rgb]{0.25,0.50,0.50}{##1}}}
\expandafter\def\csname PY@tok@cs\endcsname{\let\PY@it=\textit\def\PY@tc##1{\textcolor[rgb]{0.25,0.50,0.50}{##1}}}

\def\PYZbs{\char`\\}
\def\PYZus{\char`\_}
\def\PYZob{\char`\{}
\def\PYZcb{\char`\}}
\def\PYZca{\char`\^}
\def\PYZam{\char`\&}
\def\PYZlt{\char`\<}
\def\PYZgt{\char`\>}
\def\PYZsh{\char`\#}
\def\PYZpc{\char`\%}
\def\PYZdl{\char`\$}
\def\PYZhy{\char`\-}
\def\PYZsq{\char`\'}
\def\PYZdq{\char`\"}
\def\PYZti{\char`\~}
% for compatibility with earlier versions
\def\PYZat{@}
\def\PYZlb{[}
\def\PYZrb{]}
\makeatother


    % Exact colors from NB
    \definecolor{incolor}{rgb}{0.0, 0.0, 0.5}
    \definecolor{outcolor}{rgb}{0.545, 0.0, 0.0}



    
    % Prevent overflowing lines due to hard-to-break entities
    \sloppy 
    % Setup hyperref package
    \hypersetup{
      breaklinks=true,  % so long urls are correctly broken across lines
      colorlinks=true,
      urlcolor=urlcolor,
      linkcolor=linkcolor,
      citecolor=citecolor,
      }
    % Slightly bigger margins than the latex defaults
    
    \geometry{verbose,tmargin=1in,bmargin=1in,lmargin=1in,rmargin=1in}
    
    

    \begin{document}
    
    
    \maketitle
    
    

    
    \section*{I.1 Estadística. Análisis Exploratorio de
Datos}\label{i.1-estaduxedstica.-anuxe1lisis-exploratorio-de-datos}

    \subsection*{?`Qué es la estadística?}\label{quuxe9-es-la-estaduxedstica}

La \textbf{estadística} es el campo de las matemáticas que estudia cómo
construir modelos, establecer hipótesis y tomar decisiones a partir de
\textbf{datos empíricos} provenientes de medidas, experimentos u
observaciones.

Los elementos objeto de estudio, de los que nos interesa conocer cierta
\textbf{información}, conforman la \textbf{población estadística}. Tales
poblaciones, así como la información que nos interesa extraer, pueden
tener naturalezas muy variadas. Como veremos, \emph{la información de
interés la representamos mediante variables}.

    Veamos algunos ejemplos: 
 \begin{itemize}   
\item Población: Todos los residentes de un país.
\begin{itemize}
\item Información: voto en próximas elecciones 
\item Información: relación entre género, altura y peso 
\end{itemize}
\item Población: Señales recibidas en un receptor, contaminadas por ruido 
\begin{itemize}
\item Información: señales originales, sin ruido 
\item Información: Símbolos envíados 
\end{itemize}
\item Población: Todos los países del mundo 
\begin{itemize}
\item Información: número de habitantes de cada país 
\item Información: producto interior bruto de cada uno 
\item Información: renta per cápita de cada país
\item Información: esperanza de vida en cada uno 
\item Información: relación entre renta per cápita y esperanza de vida
\end{itemize}
\end{itemize}

    \subsubsection*{Variables y observaciones}\label{variables-y-observaciones}

Una \textbf{variable estadística} o \textbf{variable muestral} es una
magnitud de interés, relativa a una población, que puede observarse. Por
ejemplo, en la población formada por todos los países del mundo, tenemos
magnitudes observables tales como el número de habitantes, el producto
interior bruto, la renta per cápita o la esperanza de vida. O, en una
población formada por una red de sensores, podemos considerar la
temperatura, la humedad y el nivel de CO2, por ejemplo.

El conjunto de valores que puede tomar una variable estadística se llama
\textbf{espacio muestral}.

    Las variables estadísticas pueden ser: 
 \begin{itemize}
 \item 
 \textbf{Variables continuas}:
toman \emph{valores numéricos reales}. Por ejemplo, la altura de los
individuos de una población, el producto interior bruto de los países
del mundo o la lectura de la señal medida por un receptor. 
\item
\textbf{Variables discretas}: toman valores \emph{numéricos enteros}.
Por ejemplo, el número de habitantes de cada uno de los países del
mundo, o nivel detectado por un receptor de una señal digital
multinivel. 
\item 
\textbf{Variables categóricas}: sus valores \emph{no son
numéricos}, sino \emph{pertenecientes a clases}, a veces llamadas
\emph{niveles}. Por ejemplo, la ciudadanía de un individuo, el grupo
sanguíneo o los símbolos recuperados por un receptor digital. 
\begin{itemize}
\item Los \textbf{intervalos} son variables categóricas que agrupan rangos de
valores numéricos.
\item A veces es posible establecer una \textbf{relación
de orden} en los valores de una variable categórica. Por ejemplo,
podemos ordenar personas por altura, aunque no sepamos el valor
numérico.
\end{itemize}
\end{itemize}

    Con frecuencia, sólo tendremos acceso a una subpoblación. Por ejemplo,
si la población es muy numerosa o infinita, todo lo más que podremos
hacer es recoger los datos de una \textbf{subpoblación muestral} elegida
de tal forma que pensemos que representa adecuadamente a la población
general. Estudiándola esperamos poder generalizar ciertas
características que sean aplicables a toda la población. Por ejemplo, 
\begin{itemize}
\item
\textbf{Observando} mediante una encuesta la intención de voto de unos
pocos miles de personas intentamos \textbf{inferir} qué resultado van a
tener las elecciones. 
\item 
Midiendo señales y ruido en
\textbf{experimentos} controlados esperamos \textbf{inferir} modelos de
comportamiento que nos permitan tratar nuevas señales recibidas.
\end{itemize}

En otros caso, podemos disponer de la totalidad de la población.
Piénsese, por ejemplo, en la población estadística formada por todos los
países del mundo, que son unos 200.

En cualquier caso, sea cual fuere la aplicación, la primera tarea es la
\textbf{recogida de datos}, sea por pura observación o mediante
experimentación.

    En nuestro caso, vamos a ilustrar los conceptos principales de la
\textbf{estadística descriptiva} mediante datos reales, provenientes del
Banco Mundial, correspondientes a ciertos indicadores de todos los
países del mundo así como de agregaciones de países en el periodo de
tiempo 1960 - 2017. En particular trabajaremos con los siguientes
indicadores:

\begin{itemize}
\tightlist
\item
  Número de habitantes, que nos permite analizar los países por el
  tamaño de sus poblaciones
\item
  Producto interior bruto, que nos permite hacerlo por el tamaño de sus
  economías
\item
  Renta per cápita, que nos permite analizar los países por su riqueza
\item
  Esperanza de vida, que nos proporciona una indicación de la salud y,
  en cierta manera, de la calidad de vida
\end{itemize}

También analizaremos relaciones estadísticas entre estas
\textbf{variables}. Por ejemplo, ¿el tamaño de los países tiene una
relación con la riqueza de los mismos? ¿y la riqueza con la esperanza de
vida?

    \subsubsection*{Etapas en el procesado estadístico de
datos}\label{etapas-en-el-procesado-estaduxedstico-de-datos}

De lo expuesto podemos intuir que el procesado de datos se estructura en
tres etapas sucesivas, íntimamente relacionadas, y con un cierto
solapamiento entre ellas:

\begin{enumerate}
\def\labelenumi{\arabic{enumi}.}
\tightlist
\item
  \textbf{Manipulación de datos}: los datos provienen de fuentes
  diversas, por lo que frecuentemente estan recogidos en formatos
  diferentes y, con casi total seguridad, de modo que directamente no
  podemos trabajar con ellos. Además, es habitual que algunos de ellos
  falten o estén contaminados por imprecisiones en su adquisición. En
  ingeniería la presencia de \textbf{ruido} es omnipresente, pero en
  otras disciplinas también nos encontramos con incorrecciones en los
  mismos.
\item
  \textbf{Análisis de datos}: corresponde al \textbf{anaĺisis
  exploratorio de datos} que nos permite representar y entender los
  datos disponibles, preparándolos para ulteriores análisis. En esta
  etapa podemos considerar que \emph{los datos se convierten en
  información}.
\item
  \textbf{Ciencia de datos}: se utilizan las herramientas de la
  estadística matemática y del aprendizaje artificial para extrar
  inferencias, estimaciones, clasificaciones y generalizaciones a partir
  de la información presente en los datos. Podemos considerar que, en
  esta etapa, \emph{la información se convierte en conocimiento}.
\end{enumerate}

    \subsubsection*{Secuencias, tablas y series
temporales}\label{secuencias-tablas-y-series-temporales}

Una \textbf{observación} es una captura del valor de una variable
correspondiente a una \textbf{muestra} de la población estadística.
Normalmente tendremos muchas observaciones para cada variable, conjunto
que denominamos \textbf{muestreo}. Por ejemplo, en la población de
países del mundo tendremos un \textbf{muestreo} consistente en una
observación o muestra correspondiente a cada país, de cada una de las
variables que estemos estudiando (número de habitantes, producto
interior bruto, etc). El \textbf{tamaño \(N\) de un muestreo} (a veces
se dice de una muestra) indica el \textbf{número de observaciones} o
muestras que se toman de la variable bajo estudio.

La \textbf{dimensionalidad} del conjunto de datos viene determinado por
el número de variables que se observan para cada miembro de la
subpoblación muestral. Por ejemplo, si para cada país sólo se considera
el número de habitantes, el conjunto de datos es \textbf{univariado}. Si
se considera el número de habitantes y el producto interior bruto, el
conjunto de datos es \textbf{bivariado}. Por lo general, se denomina
\textbf{multivariada} a la población muestral de la que se miden tres o
más características, en nuestro caso, por ejemplo, porque añadimos la
renta per cápita y la esperanza de vida.

    Cuando tenemos varias variables y estamos interesados en entender cómo
se relacionan, es importante distinguir entre \textbf{variables
independientes}, que reflejan causas, y \textbf{variables dependientes},
que reflejan efectos. Por ejemplo, podemos considerar que una renta per
cápita elevada es causa de una mayor esperanza de vida. Pero, ¿podría
ser al revés?, esto es, que una mayor esperanza de vida sea la causa de
una mayor renta per cápita.

Si las subpoblaciones muestrales son grandes y multivariadas no es
sencillo interpretar los datos disponibles. Es necesario disponer de
técnicas que lo hagan posible, como veremos en los próximos apartados.
Una cuestión crucial es la debida ordenación de los datos en series y
tablas, cuestión que se enmarca dentro de la etapa de manipulación de
datos:

\begin{itemize}
\tightlist
\item
  \textbf{Secuencias o series}: son tablas unidimensionales en las que
  se recogen las observaciones correspondientes a una única variable. Si
  las observaciones se realizan a lo largo del tiempo, estamos ante
  \textbf{series temporales}.
\item
  \textbf{Tablas}: en inglés denominadas \textbf{\emph{data frames}}.
  Son ordenaciones de muestreos multivariados, por lo general,
  \emph{asociando cada columna a una variable y cada fila a una
  observación}. Cuando se incluye una secuencia temporal de medidas, hay
  que reflexionar si cada instante de tiempo debe considerarse una
  variable o una observación.
\end{itemize}

    \subsubsection*{¿Qué es el análisis exploratorio de
datos?}\label{quuxe9-es-el-anuxe1lisis-exploratorio-de-datos}

Se trata de un conjunto de técnicas para sistematizar la debida
interpretación de los datos disponibles. Entre ellas, la
\textbf{estadística descriptiva} permite obtener parámetros que
describen la subpoblación muestral, tanto obteniendo \textbf{sumarios
numéricos} de los mismos, como recurriendo a \textbf{descripciones
gráficas}. Pero ello no es suficiente. Hay que aproximarse críticamente
a los datos disponibles, intentando entenderlos conforme sea el proceso
subyacente que los genera.

El análisis exploratorio de datos incluye la estadística descriptiva,
con sus sumarios numéricos y sus representaciones gráficas, y la
interpretación crítica que se haga de los datos para su debida
comprensión. Sólo entonces puede plantearse la inferencia de las
propiedades del conjunto de la población muestral. No lo olvidemos:
\textbf{la estadística suele trabajar con subpoblaciones muestrales, que
se describen y analizan, de forma previa a inferir una caracterización
del conjunto de la población}.

    \subsection*{Lectura de datos del Banco
Mundial}\label{lectura-de-datos-del-banco-mundial}

Nuestra \textbf{población estadística} son todos los países del mundo,
\textbf{que muestreamos en su totalidad}. Para ello vamos a trabajar con
datos obtenidos por el Banco Mundial y disponibles en su portal web, con
periodicidad anual desde 1960, hasta 2017. Para facilitar las cosas
vamos a utilizar datos previamente descargados en formato CSV
(\emph{Comma Separated Values}), aunque podríamos hacerlo también
"\emph{online}".

\begin{itemize}
\tightlist
\item
  \textbf{Variables estadísticas o muestrales}: población (número de
  habitantes, en millones de personas), producto interior bruto (en
  miles de millones de dólares), renta per cápita (en miles de dólares)
  y esperanza de vida (en años) para cada uno de los años entre 1960 y
  2017. Son las columnas de la tabla.
\item
  \textbf{Observaciones}: las mediciones de las variables estadísticas
  para cada uno de los países. Cuando no tengamos una medida,
  utilizaremos el símbolo \emph{NA} (Not Available) o el símbolo
  \emph{NaN} (Not a Number).
\end{itemize}

Como queremos inferir conclusiones para el conjunto de los paídes del
mundo, consideramos cada uno de ellos una muestra u observación, en vez
de considerar a cada año, que será una variable distinta. Por ejemplo,
una variable es la población en 1970, y otra variable es la esperanza de
vida en 1990. Las muestras u observaciones son las medidas para cada
país de cada una de las variables.


    \subsubsection*{Análisis preliminar de la población
mundial}\label{anuxe1lisis-preliminar-de-la-poblaciuxf3n-mundial}

Tenemos una tabla para cada indicador. En cada tabla tenemos 217 países
(filas) y 58 variables (columnas), correspondientes éstas a cada uno de
los años de los que se dispone de medidas. Para facilitar la
visualización, con frecuencia \emph{transpondremos} las tablas, de modo
que los años (variables) correspondan a las filas y los países
(observaciones o muestras) correspondan a las columnas. Esto no debe
confundirnos, y se hace sólo a efectos de visualización.

    \begin{Verbatim}[commandchars=\\\{\}]
<class 'pandas.core.frame.DataFrame'>
Index: 217 entries, ABW to ZWE
Columns: 58 entries, 1960 to 2017
Freq: A-DEC
dtypes: float64(58)
memory usage: 100.0+ KB

    \end{Verbatim}

    \paragraph{Países con poblaciones máximas y mínimas. Rango de la
población}\label{pauxedses-con-poblaciones-muxe1ximas-y-muxednimas.-rango-de-la-poblaciuxf3n}

Para cada variable (población en un año), nos interesa, en primer lugar,
saber si nos faltan valores, y cuáles son los valores máximos y mínimos.
Podemos visualizarlo fácilmente, poniendo en rojo las observaciones que
faltan (NaN) y resaltando en amarillo los valores máximos y mínimos.

El \textbf{valor máximo} y el \textbf{valor mínimo} de las muestras de
una variable estadística numérica ofrecen una interesante información
preliminar.

El \textbf{rango} es la diferencia entre los valores máximo y mínimo de
tal variable numérica.

Adviértase que China se ha mantenido desde 1960 como el país de mayor
población del mundo, y que los países más pequeños tienen una población
minúscula.

            
    \subparagraph{Identificación de países a partir de sus
códigos}\label{identificaciuxf3n-de-pauxedses-a-partir-de-sus-cuxf3digos}

\{'CHN': 'China',
 'MAF': 'St. Martin (French part)',
 'NRU': 'Nauru',
 'TUV': 'Tuvalu',
 'ERI': 'Eritrea',
 'PSE': 'West Bank and Gaza',
 'SRB': 'Serbia',
 'SXM': 'Sint Maarten (Dutch part)'\}
            
    \subparagraph{Rango de poblaciones}\label{rango-de-poblaciones}

    \begin{Verbatim}[commandchars=\\\{\}]
          1960    1970    1980     1990     2000     2010     2015     2017
idxmax     CHN     CHN     CHN      CHN      CHN      CHN      CHN      CHN
max    667.070 818.315 981.235 1135.185 1262.645 1337.705 1371.220 1386.395
idxmin     MAF     MAF     NRU      TUV      TUV      NRU      TUV      TUV
min      0.004   0.005   0.007    0.009    0.009    0.010    0.011    0.011

    \end{Verbatim}

            
    \subsubsection*{Ordenación de países por población. Estadísticos de
orden}\label{ordenaciuxf3n-de-pauxedses-por-poblaciuxf3n.-estaduxedsticos-de-orden}

El \textbf{estadístico de orden k} de una variable muestral de tamaño n
(con n muestras) corresponde al k-ésimo menor valor. Por tanto,
\textbf{el estadístico de orden 1 corresponde al mínimo}, mientras que
el \textbf{estadístico de orden n corresponde al máximo}.

Podemos también invertir el orden de los estadísticos, de forma que el
orden 1 corresponda al máximo y el orden n al mínimo.

A continuación prepararemos una tabla, en la que cada variable
corresponde al puesto por población, en orden descendente (máximo = 1),
ocupado por un país, en relación al total mundial en un año determinado.

VER TRANSPARENCIA ONLINE
            
    \paragraph{Evolución en el orden de población de una selección de
países}\label{evoluciuxf3n-en-el-orden-de-poblaciuxf3n-de-una-selecciuxf3n-de-pauxedses}

Country Code  CHN  IND  USA  RUS  NGA  JPN  DEU  GBR  FRA  ESP
Year                                                          
1960          1.0  2.0  3.0  4.0 13.0  5.0  7.0  9.0 12.0 18.0
1970          1.0  2.0  3.0  4.0 11.0  6.0  8.0 12.0 14.0 22.0
1980          1.0  2.0  3.0  5.0 11.0  7.0  9.0 14.0 15.0 24.0
1990          1.0  2.0  3.0  6.0 10.0  7.0 12.0 17.0 15.0 26.0
2000          1.0  2.0  3.0  6.0 10.0  9.0 12.0 21.0 20.0 28.0
2010          1.0  2.0  3.0  9.0  7.0 10.0 16.0 22.0 20.0 27.0
2015          1.0  2.0  3.0  9.0  7.0 10.0 16.0 22.0 21.0 30.0
2017          1.0  2.0  3.0  9.0  7.0 11.0 16.0 22.0 21.0 30.0
            
    \paragraph{Análisis de los países
mayores.}\label{anuxe1lisis-de-los-pauxedses-mayores.}

    \begin{Verbatim}[commandchars=\\\{\}]
['BGD', 'BRA', 'CHN', 'DEU', 'GBR', 'IDN', 'IND', 'ITA', 'JPN', 'MEX', 'NGA', 'PAK', 'RUS', 'USA']

    \end{Verbatim}

       1    2    3    4    5    6    7    8    9    10
Year                                                  
1960  CHN  IND  USA  RUS  JPN  IDN  DEU  BRA  GBR  ITA
1970  CHN  IND  USA  RUS  IDN  JPN  BRA  DEU  BGD  PAK
1980  CHN  IND  USA  IDN  RUS  BRA  JPN  BGD  DEU  PAK
1990  CHN  IND  USA  IDN  BRA  RUS  JPN  PAK  BGD  NGA
2000  CHN  IND  USA  IDN  BRA  RUS  PAK  BGD  JPN  NGA
2010  CHN  IND  USA  IDN  BRA  PAK  NGA  BGD  RUS  JPN
2015  CHN  IND  USA  IDN  BRA  PAK  NGA  BGD  RUS  JPN
2017  CHN  IND  USA  IDN  BRA  PAK  NGA  BGD  RUS  MEX
            
\{'BGD': 'Bangladesh',
 'BRA': 'Brazil',
 'CHN': 'China',
 'DEU': 'Germany',
 'GBR': 'United Kingdom',
 'IDN': 'Indonesia',
 'IND': 'India',
 'ITA': 'Italy',
 'JPN': 'Japan',
 'MEX': 'Mexico',
 'NGA': 'Nigeria',
 'PAK': 'Pakistan',
 'RUS': 'Russian Federation',
 'USA': 'United States'\}
            
    \paragraph{Comparación de la población agregada de los países mayores
con la
mundial}\label{comparaciuxf3n-de-la-poblaciuxf3n-agregada-de-los-pauxedses-mayores-con-la-mundial}

Year         1960   1970   1980   1990   2000   2010   2015   2017
LC Total   1845.0 2223.2 2667.5 3166.7 3650.6 4069.6 4268.4 4350.8
WLD Total  3014.9 3664.3 4414.3 5267.9 6099.5 6909.7 7329.3 7501.7
LC/WLD (\%)   61.2   60.7   60.4   60.1   59.9   58.9   58.2   58.0
            
    \subsubsection*{Algunas descripciones
gráficas}\label{algunas-descripciones-gruxe1ficas}

Representemos gráficamente las poblaciones de los países mayores del
mundo.

Adviértase que estos países pueden ser representados por una
\textbf{variable categórica} cuyos valores posibles son los "códigos de
los países".

Dado que cada país tiene una población, podemos considerar que \emph{la
variable categórica juega el papel de} \textbf{\emph{variable
independiente}}, mientras que la población, \emph{una variable numérica,
juega el papel de} \textbf{\emph{variable dependiente}}.

    \paragraph{Diagrama de barras o de columnas (bar plot, bar graph, bar
chart)}\label{diagrama-de-barras-o-de-columnas-bar-plot-bar-graph-bar-chart}

Un diagrama de barras o de columnas es un gráfico que representa datos
recogidos en una variable numérica (dependiente), por ejemplo
poblaciones, frente una variable categórica independiente, por ejemplo
países. Esto se hace mediante barras horizontales o verticales, cuyas
\emph{longitudes son proporcionales a los datos} (variables numéricas
dependientes) que representan, asignando una barra a cada variable
categórica independiente. Ello permite a los diagramas de barras
comparar datos numéricos asociados a categorías discretas.

    \begin{center}
    \adjustimage{max size={0.9\linewidth}{0.9\paperheight}}{1.AnalisisExploratorioDatos_files/1.AnalisisExploratorioDatos_35_0.png}
    \end{center}
    { \hspace*{\fill} \\}
    
    \subparagraph{Diagrama de barras
horizontales}\label{diagrama-de-barras-horizontales}

    \begin{center}
    \adjustimage{max size={0.9\linewidth}{0.9\paperheight}}{1.AnalisisExploratorioDatos_files/1.AnalisisExploratorioDatos_37_0.png}
    \end{center}
    { \hspace*{\fill} \\}
    
    \paragraph{Diagrama circular o de tarta
(pieplot)}\label{diagrama-circular-o-de-tarta-pieplot}

Un diagrama circular o de tarta es un gráfico que representa datos
recogidos en una variable numérica (dependiente), por ejemplo
poblaciones, frente una variable categórica independiente, por ejemplo
países. En este caso los datos (variables numéricas dependientes) se
representan mediante sectores circulares, cuyos ángulos son
proporcionales a los datos que representan, asignando un sector a cada
variable categórica independiente. Ello permite a los diagramas
circulares o de tarta, al igual que a los de barras, comparar datos
numéricos asociados a categorías discretas.

    \begin{center}
    \adjustimage{max size={0.9\linewidth}{0.9\paperheight}}{1.AnalisisExploratorioDatos_files/1.AnalisisExploratorioDatos_39_0.png}
    \end{center}
    { \hspace*{\fill} \\}
    
    Podemos también ver cuánto suponen los países más poblados en relación a
la población mundial. Para ello normalizamos las cantidades,
considerando que la totalidad de la población mundial ocupa un sector de
360º, esto es el círculo completo. Por tanto, en blanco queda la
población del mundo no incluida en los países mayores.

    \begin{center}
    \adjustimage{max size={0.9\linewidth}{0.9\paperheight}}{1.AnalisisExploratorioDatos_files/1.AnalisisExploratorioDatos_41_0.png}
    \end{center}
    { \hspace*{\fill} \\}
    
    \subsection*{Histogramas}\label{histogramas}

Consideremos que nuestra muestra consiste en \(N\) observaciones \(x_i\)
, \(i = 1 \ldots N\) de una variable muestral.

La \textbf{frecuencias absoluta} \(F_i\) indica el número de veces que
se repite el valor \(i\)-ésimo de la variable muestral.

Lógicamente, la suma de frecuencias absolutas de todos los valores
posibles de la variable coincide con el número de observaciones:
\(\sum_i F_i = N\).

    La \textbf{frecuencia relativa} \(f_i\) es la frecuencia absoluta
dividida por el número de muestras: \(f_i=F_i/N\).

La suma de las frecuencias relativas es la unidad: \(\sum_i f_i = 1\).

    \textbf{Podemos representar tanto las frecuencias absolutas como las
relativas mediante un diagrama de barras}.

    Este cálculo de las frecuencias absolutas y relativas tiene pleno
sentido

\begin{itemize}
\tightlist
\item
  para variables categóricas
\item
  para variables numéricas discretas, si el número de valores que toma
  es pequeño en relación al tamaño de la muestra
\end{itemize}

Por ejemplo, si la variable estadística es el color de los coches, los
valores que puede tomar será los colores, blanco, rozo, azul, amarillo,
negro,... y la frecuencia absoluta es el número de coches de cada color
que hemos observado. Si sumamos el número de coches de cada color que
hemos observado, obtenemos el total de coches observados.

    Sin embargo, si la variable estadística puede tomar valores sobre un
rango continuo, difícilmente coincidirán los valores de distintas
observaciones. Lo mismo paso aún pudiendo tomar un conjunto discreto de
valores, si los valores posibles son muchos en relación al número de
observaciones.

En tales casos las observaciones se agrupan en intervalos, mediante un
procedimiento conocido como \emph{binning} o \emph{bucketing}. Para
ello, se realiza una partición del espacio muestral ocupado por las
observaciones en \(K\) intervalos, cuyos límites son
\(l_0, l_1 \ldots l_N\).

El intervalo k-ésimo es \(I_k = (l_{k-1}, l_k]\), que se convierte en
una variable categórica que se observa tantas veces como muestras
\(x_i\) caigan dentro del mismo. Por tanto, la \textbf{frecuencia
absoluta \(F_k\) del intervalo \(I_k\)} es igual al número de
observaciones que caigan dentro de él, y su \textbf{frecuencia relativa}
\(f_k = F_k\)/N.

Los intervalos no tienen por qué tener igual longitud. \textbf{La
partición del espacio muestral se diseña según se distribuyan las
observaciones}.

    Asignemos las poblaciones de 2017 de los países del mundo a una
\textbf{partición de 10 intervalos idénticos}, y veamos las
\textbf{frecuencias absolutas de cada intervalo} (cuántos países tiene
cada uno). Los valores negativos se deben a que el software busca los
límites automáticamente, sin que se le haya especificado que no puede
haber valores de población negativos.

VER TRANSPARENCIA ONLINE


Country Code               ABW               AFG               AGO  \textbackslash{}
2017          (-1.375, 138.65]  (-1.375, 138.65]  (-1.375, 138.65]   

Country Code               ALB               AND               ARE  \textbackslash{}
2017          (-1.375, 138.65]  (-1.375, 138.65]  (-1.375, 138.65]   

Country Code               ARG               ARM               ASM  \textbackslash{}
2017          (-1.375, 138.65]  (-1.375, 138.65]  (-1.375, 138.65]   

Country Code               ATG               AUS               AUT  \textbackslash{}
2017          (-1.375, 138.65]  (-1.375, 138.65]  (-1.375, 138.65]   

Country Code               AZE               BDI               BEL  \textbackslash{}
2017          (-1.375, 138.65]  (-1.375, 138.65]  (-1.375, 138.65]   

Country Code               BEN               BFA                BGD  \textbackslash{}
2017          (-1.375, 138.65]  (-1.375, 138.65]  (138.65, 277.288]   

Country Code               BGR               BHR               BHS  \textbackslash{}
2017          (-1.375, 138.65]  (-1.375, 138.65]  (-1.375, 138.65]   

Country Code               BIH               BLR               BLZ  \textbackslash{}
2017          (-1.375, 138.65]  (-1.375, 138.65]  (-1.375, 138.65]   

Country Code               BMU               BOL                BRA  \textbackslash{}
2017          (-1.375, 138.65]  (-1.375, 138.65]  (138.65, 277.288]   

Country Code               BRB               BRN               BTN  \textbackslash{}
2017          (-1.375, 138.65]  (-1.375, 138.65]  (-1.375, 138.65]   

Country Code               BWA               CAF               CAN  \textbackslash{}
2017          (-1.375, 138.65]  (-1.375, 138.65]  (-1.375, 138.65]   

Country Code               CHE               CHI               CHL  \textbackslash{}
2017          (-1.375, 138.65]  (-1.375, 138.65]  (-1.375, 138.65]   

Country Code                   CHN               CIV               CMR  \textbackslash{}
2017          (1247.757, 1386.395]  (-1.375, 138.65]  (-1.375, 138.65]   

Country Code               COD               COG               COL  \textbackslash{}
2017          (-1.375, 138.65]  (-1.375, 138.65]  (-1.375, 138.65]   

Country Code               COM               CPV               CRI  \textbackslash{}
2017          (-1.375, 138.65]  (-1.375, 138.65]  (-1.375, 138.65]   

Country Code               CUB               CUW               CYM  \textbackslash{}
2017          (-1.375, 138.65]  (-1.375, 138.65]  (-1.375, 138.65]   

Country Code               CYP               CZE               DEU  \textbackslash{}
2017          (-1.375, 138.65]  (-1.375, 138.65]  (-1.375, 138.65]   

Country Code               DJI               DMA               DNK  \textbackslash{}
2017          (-1.375, 138.65]  (-1.375, 138.65]  (-1.375, 138.65]   

Country Code               DOM               DZA               ECU  \textbackslash{}
2017          (-1.375, 138.65]  (-1.375, 138.65]  (-1.375, 138.65]   

Country Code               EGY  ERI               ESP               EST  \textbackslash{}
2017          (-1.375, 138.65]  NaN  (-1.375, 138.65]  (-1.375, 138.65]   

Country Code               ETH               FIN               FJI  \textbackslash{}
2017          (-1.375, 138.65]  (-1.375, 138.65]  (-1.375, 138.65]   

Country Code               FRA               FRO               FSM  \textbackslash{}
2017          (-1.375, 138.65]  (-1.375, 138.65]  (-1.375, 138.65]   

Country Code               GAB               GBR               GEO  \textbackslash{}
2017          (-1.375, 138.65]  (-1.375, 138.65]  (-1.375, 138.65]   

Country Code               GHA               GIB               GIN  \textbackslash{}
2017          (-1.375, 138.65]  (-1.375, 138.65]  (-1.375, 138.65]   

Country Code               GMB               GNB               GNQ  \textbackslash{}
2017          (-1.375, 138.65]  (-1.375, 138.65]  (-1.375, 138.65]   

Country Code               GRC               GRD               GRL  \textbackslash{}
2017          (-1.375, 138.65]  (-1.375, 138.65]  (-1.375, 138.65]   

Country Code               GTM               GUM               GUY  \textbackslash{}
2017          (-1.375, 138.65]  (-1.375, 138.65]  (-1.375, 138.65]   

Country Code               HKG               HND               HRV  \textbackslash{}
2017          (-1.375, 138.65]  (-1.375, 138.65]  (-1.375, 138.65]   

Country Code               HTI               HUN                IDN  \textbackslash{}
2017          (-1.375, 138.65]  (-1.375, 138.65]  (138.65, 277.288]   

Country Code               IMN                   IND               IRL  \textbackslash{}
2017          (-1.375, 138.65]  (1247.757, 1386.395]  (-1.375, 138.65]   

Country Code               IRN               IRQ               ISL  \textbackslash{}
2017          (-1.375, 138.65]  (-1.375, 138.65]  (-1.375, 138.65]   

Country Code               ISR               ITA               JAM  \textbackslash{}
2017          (-1.375, 138.65]  (-1.375, 138.65]  (-1.375, 138.65]   

Country Code               JOR               JPN               KAZ  \textbackslash{}
2017          (-1.375, 138.65]  (-1.375, 138.65]  (-1.375, 138.65]   

Country Code               KEN               KGZ               KHM  \textbackslash{}
2017          (-1.375, 138.65]  (-1.375, 138.65]  (-1.375, 138.65]   

Country Code               KIR               KNA               KOR  \textbackslash{}
2017          (-1.375, 138.65]  (-1.375, 138.65]  (-1.375, 138.65]   

Country Code               KWT               LAO               LBN  \textbackslash{}
2017          (-1.375, 138.65]  (-1.375, 138.65]  (-1.375, 138.65]   

Country Code               LBR               LBY               LCA  \textbackslash{}
2017          (-1.375, 138.65]  (-1.375, 138.65]  (-1.375, 138.65]   

Country Code               LIE               LKA               LSO  \textbackslash{}
2017          (-1.375, 138.65]  (-1.375, 138.65]  (-1.375, 138.65]   

Country Code               LTU               LUX               LVA  \textbackslash{}
2017          (-1.375, 138.65]  (-1.375, 138.65]  (-1.375, 138.65]   

Country Code               MAC               MAF               MAR  \textbackslash{}
2017          (-1.375, 138.65]  (-1.375, 138.65]  (-1.375, 138.65]   

Country Code               MCO               MDA               MDG  \textbackslash{}
2017          (-1.375, 138.65]  (-1.375, 138.65]  (-1.375, 138.65]   

Country Code               MDV               MEX               MHL  \textbackslash{}
2017          (-1.375, 138.65]  (-1.375, 138.65]  (-1.375, 138.65]   

Country Code               MKD               MLI               MLT  \textbackslash{}
2017          (-1.375, 138.65]  (-1.375, 138.65]  (-1.375, 138.65]   

Country Code               MMR               MNE               MNG  \textbackslash{}
2017          (-1.375, 138.65]  (-1.375, 138.65]  (-1.375, 138.65]   

Country Code               MNP               MOZ               MRT  \textbackslash{}
2017          (-1.375, 138.65]  (-1.375, 138.65]  (-1.375, 138.65]   

Country Code               MUS               MWI               MYS  \textbackslash{}
2017          (-1.375, 138.65]  (-1.375, 138.65]  (-1.375, 138.65]   

Country Code               NAM               NCL               NER  \textbackslash{}
2017          (-1.375, 138.65]  (-1.375, 138.65]  (-1.375, 138.65]   

Country Code                NGA               NIC               NLD  \textbackslash{}
2017          (138.65, 277.288]  (-1.375, 138.65]  (-1.375, 138.65]   

Country Code               NOR               NPL               NRU  \textbackslash{}
2017          (-1.375, 138.65]  (-1.375, 138.65]  (-1.375, 138.65]   

Country Code               NZL               OMN                PAK  \textbackslash{}
2017          (-1.375, 138.65]  (-1.375, 138.65]  (138.65, 277.288]   

Country Code               PAN               PER               PHL  \textbackslash{}
2017          (-1.375, 138.65]  (-1.375, 138.65]  (-1.375, 138.65]   

Country Code               PLW               PNG               POL  \textbackslash{}
2017          (-1.375, 138.65]  (-1.375, 138.65]  (-1.375, 138.65]   

Country Code               PRI               PRK               PRT  \textbackslash{}
2017          (-1.375, 138.65]  (-1.375, 138.65]  (-1.375, 138.65]   

Country Code               PRY               PSE               PYF  \textbackslash{}
2017          (-1.375, 138.65]  (-1.375, 138.65]  (-1.375, 138.65]   

Country Code               QAT               ROU                RUS  \textbackslash{}
2017          (-1.375, 138.65]  (-1.375, 138.65]  (138.65, 277.288]   

Country Code               RWA               SAU               SDN  \textbackslash{}
2017          (-1.375, 138.65]  (-1.375, 138.65]  (-1.375, 138.65]   

Country Code               SEN               SGP               SLB  \textbackslash{}
2017          (-1.375, 138.65]  (-1.375, 138.65]  (-1.375, 138.65]   

Country Code               SLE               SLV               SMR  \textbackslash{}
2017          (-1.375, 138.65]  (-1.375, 138.65]  (-1.375, 138.65]   

Country Code               SOM               SRB               SSD  \textbackslash{}
2017          (-1.375, 138.65]  (-1.375, 138.65]  (-1.375, 138.65]   

Country Code               STP               SUR               SVK  \textbackslash{}
2017          (-1.375, 138.65]  (-1.375, 138.65]  (-1.375, 138.65]   

Country Code               SVN               SWE               SWZ  \textbackslash{}
2017          (-1.375, 138.65]  (-1.375, 138.65]  (-1.375, 138.65]   

Country Code               SXM               SYC               SYR  \textbackslash{}
2017          (-1.375, 138.65]  (-1.375, 138.65]  (-1.375, 138.65]   

Country Code               TCA               TCD               TGO  \textbackslash{}
2017          (-1.375, 138.65]  (-1.375, 138.65]  (-1.375, 138.65]   

Country Code               THA               TJK               TKM  \textbackslash{}
2017          (-1.375, 138.65]  (-1.375, 138.65]  (-1.375, 138.65]   

Country Code               TLS               TON               TTO  \textbackslash{}
2017          (-1.375, 138.65]  (-1.375, 138.65]  (-1.375, 138.65]   

Country Code               TUN               TUR               TUV  \textbackslash{}
2017          (-1.375, 138.65]  (-1.375, 138.65]  (-1.375, 138.65]   

Country Code               TZA               UGA               UKR  \textbackslash{}
2017          (-1.375, 138.65]  (-1.375, 138.65]  (-1.375, 138.65]   

Country Code               URY                 USA               UZB  \textbackslash{}
2017          (-1.375, 138.65]  (277.288, 415.926]  (-1.375, 138.65]   

Country Code               VCT               VEN               VGB  \textbackslash{}
2017          (-1.375, 138.65]  (-1.375, 138.65]  (-1.375, 138.65]   

Country Code               VIR               VNM               VUT  \textbackslash{}
2017          (-1.375, 138.65]  (-1.375, 138.65]  (-1.375, 138.65]   

Country Code               WSM               XKX               YEM  \textbackslash{}
2017          (-1.375, 138.65]  (-1.375, 138.65]  (-1.375, 138.65]   

Country Code               ZAF               ZMB               ZWE  
2017          (-1.375, 138.65]  (-1.375, 138.65]  (-1.375, 138.65]  
            
      (-1.375, 138.65]  (138.65, 277.288]  (277.288, 415.926]  \textbackslash{}
2017               207                  6                   1   

      (415.926, 554.565]  (554.565, 693.203]  (693.203, 831.841]  \textbackslash{}
2017                   0                   0                   0   

      (831.841, 970.48]  (970.48, 1109.118]  (1109.118, 1247.757]  \textbackslash{}
2017                  0                   0                     0   

      (1247.757, 1386.395]  
2017                     2  
            
    Veamos el \textbf{histograma} correspondiente

    \begin{center}
    \adjustimage{max size={0.9\linewidth}{0.9\paperheight}}{1.AnalisisExploratorioDatos_files/1.AnalisisExploratorioDatos_51_0.png}
    \end{center}
    { \hspace*{\fill} \\}
    
    Asignemos ahora las poblaciones a una partición de \textbf{100} y
\textbf{200 intervalos} idénticos y representemos los histogramas
respectivos.

    \begin{Verbatim}[commandchars=\\\{\}]
Partición en 100 intervalos

    \end{Verbatim}

VER TRANSPARENCIA ONLINE

      (-1.375, 13.875]  (13.875, 27.739]  (27.739, 41.603]  (41.603, 55.467]  \textbackslash{}
2017               143                23                17                 8   

      (55.467, 69.33]  (69.33, 83.194]  (83.194, 97.058]  (97.058, 110.922]  \textbackslash{}
2017                6                4                 1                  3   

      (110.922, 124.786]  (124.786, 138.65]  (138.65, 152.513]  \textbackslash{}
2017                   0                  2                  1   

      (152.513, 166.377]  (166.377, 180.241]  (180.241, 194.105]  \textbackslash{}
2017                   1                   0                   1   

      (194.105, 207.969]  (207.969, 221.833]  (221.833, 235.696]  \textbackslash{}
2017                   1                   1                   0   

      (235.696, 249.56]  (249.56, 263.424]  (263.424, 277.288]  \textbackslash{}
2017                  0                  0                   1   

      (277.288, 291.152]  (291.152, 305.016]  (305.016, 318.879]  \textbackslash{}
2017                   0                   0                   0   

      (318.879, 332.743]  (332.743, 346.607]  (346.607, 360.471]  \textbackslash{}
2017                   1                   0                   0   

      (360.471, 374.335]  (374.335, 388.199]  (388.199, 402.062]  \textbackslash{}
2017                   0                   0                   0   

      (402.062, 415.926]  (415.926, 429.79]  (429.79, 443.654]  \textbackslash{}
2017                   0                  0                  0   

      (443.654, 457.518]  (457.518, 471.382]  (471.382, 485.246]  \textbackslash{}
2017                   0                   0                   0   

      (485.246, 499.109]  (499.109, 512.973]  (512.973, 526.837]  \textbackslash{}
2017                   0                   0                   0   

      (526.837, 540.701]  (540.701, 554.565]  (554.565, 568.429]  \textbackslash{}
2017                   0                   0                   0   

      (568.429, 582.292]  (582.292, 596.156]  (596.156, 610.02]  \textbackslash{}
2017                   0                   0                  0   

      (610.02, 623.884]  (623.884, 637.748]  (637.748, 651.612]  \textbackslash{}
2017                  0                   0                   0   

      (651.612, 665.475]  (665.475, 679.339]  (679.339, 693.203]  \textbackslash{}
2017                   0                   0                   0   

      (693.203, 707.067]  (707.067, 720.931]  (720.931, 734.795]  \textbackslash{}
2017                   0                   0                   0   

      (734.795, 748.658]  (748.658, 762.522]  (762.522, 776.386]  \textbackslash{}
2017                   0                   0                   0   

      (776.386, 790.25]  (790.25, 804.114]  (804.114, 817.978]  \textbackslash{}
2017                  0                  0                   0   

      (817.978, 831.841]  (831.841, 845.705]  (845.705, 859.569]  \textbackslash{}
2017                   0                   0                   0   

      (859.569, 873.433]  (873.433, 887.297]  (887.297, 901.161]  \textbackslash{}
2017                   0                   0                   0   

      (901.161, 915.025]  (915.025, 928.888]  (928.888, 942.752]  \textbackslash{}
2017                   0                   0                   0   

      (942.752, 956.616]  (956.616, 970.48]  (970.48, 984.344]  \textbackslash{}
2017                   0                  0                  0   

      (984.344, 998.208]  (998.208, 1012.071]  (1012.071, 1025.935]  \textbackslash{}
2017                   0                    0                     0   

      (1025.935, 1039.799]  (1039.799, 1053.663]  (1053.663, 1067.527]  \textbackslash{}
2017                     0                     0                     0   

      (1067.527, 1081.391]  (1081.391, 1095.254]  (1095.254, 1109.118]  \textbackslash{}
2017                     0                     0                     0   

      (1109.118, 1122.982]  (1122.982, 1136.846]  (1136.846, 1150.71]  \textbackslash{}
2017                     0                     0                    0   

      (1150.71, 1164.574]  (1164.574, 1178.437]  (1178.437, 1192.301]  \textbackslash{}
2017                    0                     0                     0   

      (1192.301, 1206.165]  (1206.165, 1220.029]  (1220.029, 1233.893]  \textbackslash{}
2017                     0                     0                     0   

      (1233.893, 1247.757]  (1247.757, 1261.62]  (1261.62, 1275.484]  \textbackslash{}
2017                     0                    0                    0   

      (1275.484, 1289.348]  (1289.348, 1303.212]  (1303.212, 1317.076]  \textbackslash{}
2017                     0                     0                     0   

      (1317.076, 1330.94]  (1330.94, 1344.803]  (1344.803, 1358.667]  \textbackslash{}
2017                    0                    1                     0   

      (1358.667, 1372.531]  (1372.531, 1386.395]  
2017                     0                     1  
            
    \begin{Verbatim}[commandchars=\\\{\}]
Partición en 200 intervalos

    \end{Verbatim}

VER TRANSPARENCIA ONLINE

      (-1.375, 6.943]  (6.943, 13.875]  (13.875, 20.807]  (20.807, 27.739]  \textbackslash{}
2017              112               31                16                 7   

      (27.739, 34.671]  (34.671, 41.603]  (41.603, 48.535]  (48.535, 55.467]  \textbackslash{}
2017                10                 7                 4                 4   

      (55.467, 62.398]  (62.398, 69.33]  (69.33, 76.262]  (76.262, 83.194]  \textbackslash{}
2017                 3                3                0                 4   

      (83.194, 90.126]  (90.126, 97.058]  (97.058, 103.99]  (103.99, 110.922]  \textbackslash{}
2017                 0                 1                 1                  2   

      (110.922, 117.854]  (117.854, 124.786]  (124.786, 131.718]  \textbackslash{}
2017                   0                   0                   2   

      (131.718, 138.65]  (138.65, 145.581]  (145.581, 152.513]  \textbackslash{}
2017                  0                  1                   0   

      (152.513, 159.445]  (159.445, 166.377]  (166.377, 173.309]  \textbackslash{}
2017                   0                   1                   0   

      (173.309, 180.241]  (180.241, 187.173]  (187.173, 194.105]  \textbackslash{}
2017                   0                   0                   1   

      (194.105, 201.037]  (201.037, 207.969]  (207.969, 214.901]  \textbackslash{}
2017                   1                   0                   1   

      (214.901, 221.833]  (221.833, 228.765]  (228.765, 235.696]  \textbackslash{}
2017                   0                   0                   0   

      (235.696, 242.628]  (242.628, 249.56]  (249.56, 256.492]  \textbackslash{}
2017                   0                  0                  0   

      (256.492, 263.424]  (263.424, 270.356]  (270.356, 277.288]  \textbackslash{}
2017                   0                   1                   0   

      (277.288, 284.22]  (284.22, 291.152]  (291.152, 298.084]  \textbackslash{}
2017                  0                  0                   0   

      (298.084, 305.016]  (305.016, 311.948]  (311.948, 318.879]  \textbackslash{}
2017                   0                   0                   0   

      (318.879, 325.811]  (325.811, 332.743]  (332.743, 339.675]  \textbackslash{}
2017                   1                   0                   0   

      (339.675, 346.607]  (346.607, 353.539]  (353.539, 360.471]  \textbackslash{}
2017                   0                   0                   0   

      (360.471, 367.403]  (367.403, 374.335]  (374.335, 381.267]  \textbackslash{}
2017                   0                   0                   0   

      (381.267, 388.199]  (388.199, 395.131]  (395.131, 402.062]  \textbackslash{}
2017                   0                   0                   0   

      (402.062, 408.994]  (408.994, 415.926]  (415.926, 422.858]  \textbackslash{}
2017                   0                   0                   0   

      (422.858, 429.79]  (429.79, 436.722]  (436.722, 443.654]  \textbackslash{}
2017                  0                  0                   0   

      (443.654, 450.586]  (450.586, 457.518]  (457.518, 464.45]  \textbackslash{}
2017                   0                   0                  0   

      (464.45, 471.382]  (471.382, 478.314]  (478.314, 485.246]  \textbackslash{}
2017                  0                   0                   0   

      (485.246, 492.177]  (492.177, 499.109]  (499.109, 506.041]  \textbackslash{}
2017                   0                   0                   0   

      (506.041, 512.973]  (512.973, 519.905]  (519.905, 526.837]  \textbackslash{}
2017                   0                   0                   0   

      (526.837, 533.769]  (533.769, 540.701]  (540.701, 547.633]  \textbackslash{}
2017                   0                   0                   0   

      (547.633, 554.565]  (554.565, 561.497]  (561.497, 568.429]  \textbackslash{}
2017                   0                   0                   0   

      (568.429, 575.36]  (575.36, 582.292]  (582.292, 589.224]  \textbackslash{}
2017                  0                  0                   0   

      (589.224, 596.156]  (596.156, 603.088]  (603.088, 610.02]  \textbackslash{}
2017                   0                   0                  0   

      (610.02, 616.952]  (616.952, 623.884]  (623.884, 630.816]  \textbackslash{}
2017                  0                   0                   0   

      (630.816, 637.748]  (637.748, 644.68]  (644.68, 651.612]  \textbackslash{}
2017                   0                  0                  0   

      (651.612, 658.544]  (658.544, 665.475]  (665.475, 672.407]  \textbackslash{}
2017                   0                   0                   0   

      (672.407, 679.339]  (679.339, 686.271]  (686.271, 693.203]  \textbackslash{}
2017                   0                   0                   0   

      (693.203, 700.135]  (700.135, 707.067]  (707.067, 713.999]  \textbackslash{}
2017                   0                   0                   0   

      (713.999, 720.931]  (720.931, 727.863]  (727.863, 734.795]  \textbackslash{}
2017                   0                   0                   0   

      (734.795, 741.727]  (741.727, 748.658]  (748.658, 755.59]  \textbackslash{}
2017                   0                   0                  0   

      (755.59, 762.522]  (762.522, 769.454]  (769.454, 776.386]  \textbackslash{}
2017                  0                   0                   0   

      (776.386, 783.318]  (783.318, 790.25]  (790.25, 797.182]  \textbackslash{}
2017                   0                  0                  0   

      (797.182, 804.114]  (804.114, 811.046]  (811.046, 817.978]  \textbackslash{}
2017                   0                   0                   0   

      (817.978, 824.91]  (824.91, 831.841]  (831.841, 838.773]  \textbackslash{}
2017                  0                  0                   0   

      (838.773, 845.705]  (845.705, 852.637]  (852.637, 859.569]  \textbackslash{}
2017                   0                   0                   0   

      (859.569, 866.501]  (866.501, 873.433]  (873.433, 880.365]  \textbackslash{}
2017                   0                   0                   0   

      (880.365, 887.297]  (887.297, 894.229]  (894.229, 901.161]  \textbackslash{}
2017                   0                   0                   0   

      (901.161, 908.093]  (908.093, 915.025]  (915.025, 921.956]  \textbackslash{}
2017                   0                   0                   0   

      (921.956, 928.888]  (928.888, 935.82]  (935.82, 942.752]  \textbackslash{}
2017                   0                  0                  0   

      (942.752, 949.684]  (949.684, 956.616]  (956.616, 963.548]  \textbackslash{}
2017                   0                   0                   0   

      (963.548, 970.48]  (970.48, 977.412]  (977.412, 984.344]  \textbackslash{}
2017                  0                  0                   0   

      (984.344, 991.276]  (991.276, 998.208]  (998.208, 1005.139]  \textbackslash{}
2017                   0                   0                    0   

      (1005.139, 1012.071]  (1012.071, 1019.003]  (1019.003, 1025.935]  \textbackslash{}
2017                     0                     0                     0   

      (1025.935, 1032.867]  (1032.867, 1039.799]  (1039.799, 1046.731]  \textbackslash{}
2017                     0                     0                     0   

      (1046.731, 1053.663]  (1053.663, 1060.595]  (1060.595, 1067.527]  \textbackslash{}
2017                     0                     0                     0   

      (1067.527, 1074.459]  (1074.459, 1081.391]  (1081.391, 1088.322]  \textbackslash{}
2017                     0                     0                     0   

      (1088.322, 1095.254]  (1095.254, 1102.186]  (1102.186, 1109.118]  \textbackslash{}
2017                     0                     0                     0   

      (1109.118, 1116.05]  (1116.05, 1122.982]  (1122.982, 1129.914]  \textbackslash{}
2017                    0                    0                     0   

      (1129.914, 1136.846]  (1136.846, 1143.778]  (1143.778, 1150.71]  \textbackslash{}
2017                     0                     0                    0   

      (1150.71, 1157.642]  (1157.642, 1164.574]  (1164.574, 1171.506]  \textbackslash{}
2017                    0                     0                     0   

      (1171.506, 1178.437]  (1178.437, 1185.369]  (1185.369, 1192.301]  \textbackslash{}
2017                     0                     0                     0   

      (1192.301, 1199.233]  (1199.233, 1206.165]  (1206.165, 1213.097]  \textbackslash{}
2017                     0                     0                     0   

      (1213.097, 1220.029]  (1220.029, 1226.961]  (1226.961, 1233.893]  \textbackslash{}
2017                     0                     0                     0   

      (1233.893, 1240.825]  (1240.825, 1247.757]  (1247.757, 1254.689]  \textbackslash{}
2017                     0                     0                     0   

      (1254.689, 1261.62]  (1261.62, 1268.552]  (1268.552, 1275.484]  \textbackslash{}
2017                    0                    0                     0   

      (1275.484, 1282.416]  (1282.416, 1289.348]  (1289.348, 1296.28]  \textbackslash{}
2017                     0                     0                    0   

      (1296.28, 1303.212]  (1303.212, 1310.144]  (1310.144, 1317.076]  \textbackslash{}
2017                    0                     0                     0   

      (1317.076, 1324.008]  (1324.008, 1330.94]  (1330.94, 1337.872]  \textbackslash{}
2017                     0                    0                    0   

      (1337.872, 1344.803]  (1344.803, 1351.735]  (1351.735, 1358.667]  \textbackslash{}
2017                     1                     0                     0   

      (1358.667, 1365.599]  (1365.599, 1372.531]  (1372.531, 1379.463]  \textbackslash{}
2017                     0                     0                     0   

      (1379.463, 1386.395]  
2017                     1  
            
    \begin{center}
    \adjustimage{max size={0.9\linewidth}{0.9\paperheight}}{1.AnalisisExploratorioDatos_files/1.AnalisisExploratorioDatos_55_0.png}
    \end{center}
    { \hspace*{\fill} \\}
    
    Hagamos ahora una partición en 12 intervalos irregulares, cuyos puntos
de frontera son


\[\left\{0, 0.5, 1, 10, 20, 30, 40, 50, 100, 200, 400, 1300, 1400\right\}\]


VER TRANSPARENCIA ONLINE

      (0.0, 0.5]  (0.5, 1.0]  (1.0, 10.0]  (10.0, 20.0]  (20.0, 30.0]  \textbackslash{}
2017          47          11           70            31            12   

      (30.0, 40.0]  (40.0, 50.0]  (50.0, 100.0]  (100.0, 200.0]  \textbackslash{}
2017            10             8             14               8   

      (200.0, 400.0]  (400.0, 1300.0]  (1300.0, 1400.0]  
2017               3                0                 2  
            
    \begin{center}
    \adjustimage{max size={0.9\linewidth}{0.9\paperheight}}{1.AnalisisExploratorioDatos_files/1.AnalisisExploratorioDatos_58_0.png}
    \end{center}
    { \hspace*{\fill} \\}
    
    \subsubsection*{Histograma acumulativo}\label{histograma-acumulativo}

Las \textbf{frecuencias absolutas acumulativas} en el valor de cada
observación correesponde a la suma de frecuencias absolutas acumuladas
hasta dicha observación:

\[F_{cum}(x_n) = \sum_{i \leq n} F(x_i)\]

De forma semejante a como hemos hecho antes, podemos definir también:

\begin{itemize}
\tightlist
\item
  La \textbf{frecuencia cumulativa relativa} sin más que dividir por el
  número de observaciones \(N\).
\item
  El \textbf{histograma acumulativo}
\end{itemize}

VER TRANSPARENCIA ONLINE

                   F  Fcum
(0.0, 0.5]        47    47
(0.5, 1.0]        11    58
(1.0, 10.0]       70   128
(10.0, 20.0]      31   159
(20.0, 30.0]      12   171
(30.0, 40.0]      10   181
(40.0, 50.0]       8   189
(50.0, 100.0]     14   203
(100.0, 200.0]     8   211
(200.0, 400.0]     3   214
(400.0, 1300.0]    0   214
(1300.0, 1400.0]   2   216
            
    \begin{center}
    \adjustimage{max size={0.9\linewidth}{0.9\paperheight}}{1.AnalisisExploratorioDatos_files/1.AnalisisExploratorioDatos_61_0.png}
    \end{center}
    { \hspace*{\fill} \\}
    
    Como colofón, comparemos los histogramas de las poblaciones de los
países del mundo en 1960 y 2017, ambos calculados con 200 intervalos
iguales:

    \begin{center}
    \adjustimage{max size={0.9\linewidth}{0.9\paperheight}}{1.AnalisisExploratorioDatos_files/1.AnalisisExploratorioDatos_63_0.png}
    \end{center}
    { \hspace*{\fill} \\}
    
    Y las poblaciones acumuladas de todos los países del mundo a lo largo
del periodo 1960 a 2017:

    \begin{center}
    \adjustimage{max size={0.9\linewidth}{0.9\paperheight}}{1.AnalisisExploratorioDatos_files/1.AnalisisExploratorioDatos_65_0.png}
    \end{center}
    { \hspace*{\fill} \\}

VER TRANSPARENCIA ONLINE
    
Year   1960   1970   1980   1990   2000   2010   2015   2017
suma 3014.9 3664.3 4414.3 5267.9 6099.5 6909.7 7329.3 7501.7
WLD  3032.2 3685.8 4439.3 5288.1 6121.7 6932.9 7357.6 7530.4
            
    \subsection*{Estadísticos descriptivos: cuantiles, forma, tendencia
central,
dispersión}\label{estaduxedsticos-descriptivos-cuantiles-forma-tendencia-central-dispersiuxf3n}

Un \textbf{estadístico}, también llamado \textbf{estadístico muestral}
es una función matemática de las observaciones o muestras de una
\textbf{variable muestral}, que permite hacer una caracterización
parcial de las mismas. Utilizando simultáneamente varios estadísticos es
posible hacer una caractericación más completa de la muestra.

Dado que las observaciones o muestras estadísticas representan aspectos
de una población, los estadísticos calculados a partir de las muestras
también caracterizan parcialmente a la población.

    Consideremos que nuestra muestra consiste en \(N\) observaciones \(x_i\)
, \(i = 1 \ldots N\) de una variable muestral. Llamamos
\textbf{estadístico} tanto a una función \(f\) definida sobre tales
observaciones, como al valor que tomo la función al ser evaluada
\(f(x_1,\ldots , x_N)\). Salvo que indiquemos otra cosa, entenderemos
que estadístico se refiere a esta segunda acepción (el valor calculado
por la función a partir de las observaciones).

Adviértase que \textbf{un estadístico es aleatorio}, pues está calculado
a partir de observaciones que también lo son.

    Hay distintos tipos de estadístico, dependiendo del uso que se les
quiera dar. Entre ellos:

\begin{itemize}
\tightlist
\item
  \textbf{Estadísticos descriptivos}: se utilizan para realizar un
  resumen descriptivo de la muestra, y se asocian generalmente al
  anáĺisis exploratorio de datos y a la estadística descriptiva.
\item
  \textbf{Estimadores}: se utilizan para estimar parámetros de
  poblaciones estadísticas.
\item
  \textbf{Estadísticos de test}: se utilizan para hacer contrastes de
  hipótesis a partir de las observaciones.
\end{itemize}

Los estadísticos son \textbf{robustos} si se comportan bien en presencia
de observaciones atípicas (\emph{outliers}).

De momento nos centraremos en los estadísticos descriptivos más comunes,
que suelen denominarse \textbf{estadísticos descriptivos de resumen}
(\emph{summary statistics}), pues nos permiten obtener un rápido resumen
de las características de la muestra.

    \textbf{Estadísticos descriptivos} más habituales:

\begin{itemize}
\tightlist
\item
  \textbf{Estadísticos cuantiles}: entre ellos, la \textbf{mediana},
  \textbf{cuartiles}, \textbf{deciles} y \textbf{centiles}.
\item
  \textbf{Estadísticos de forma de la distribución}: son el
  \textbf{coeficiente de apuntamiento} (\emph{skewness}) y el
  \textbf{coeficiente de kurtosis}.
\item
  \textbf{Estadísticos de tendencia central}: son, entre otros, el
  \textbf{rango medio} (\emph{midrange}), el \textbf{rango medio
  intercuartil} (\emph{midhinge}), la \textbf{media}, la
  \textbf{mediana}, la \textbf{media ponderada}, las \textbf{medias
  recortadas} y la \textbf{moda}.
\item
  \textbf{Estadísticos de dispersión}: son, entre otros, la
  \textbf{desviación típica} y la \textbf{varianza} y los distintos
  \textbf{rangos}.
\item
  \textbf{Estadísticos de dependencia}: por ejemplo, el
  \textbf{coeficiente de correlación}.
\end{itemize}

Los estadísticos de orden, entre ellos el mínimo y el máximo, y las
frecuencias de las onservaciones e intervalos son también estadísticos.

    \subsubsection*{Cuantiles}\label{cuantiles}

Los \textbf{cuantiles} agrupan las observaciones de una variable
estadística \emph{ordenada} (numérica o categórica con relación de
orden) en intervalos con \emph{aproximadamente} igual número de
muestras. Estos intervalos se ordenan sucesivamente, tocándose en un
único punto de frontera, sin solaparse, y el conjunto de todos ellos
contiene todas las observaciones.

Los cuantiles son los puntos de frontera entre intervalos sucesivos. Si
\(q\) es el número de intervalos de la partición, tendremos \(q-1\)
puntos frontera que llamaremos \textbf{q-quantiles}. Los intervalos
contendrán idéntico número de muestras si el número total de
observaciones \(N\) es divisible por el número de intervalos \(q\).

    Algunos cuantiles tienen nombres especiales:

\begin{itemize}
\tightlist
\item
  Partición en 2 intervalos: hay un único 2-cuantil, que se llama
  \textbf{mediana}. Cada intervalo contiene el 50\% de las
  observaciones.
\item
  Particiones en 4 intervalos: hay tres 4-cuantiles, \(Q_1\), \(Q_2\) y
  \(Q_3\), que se llaman \textbf{cuartiles}. Cada intervalo contiene el
  25\% de las observaciones. El cuartil \(Q_2\) coincide con la mediana.
\item
  Particiones en 10 intervalos: hay nueve 10-cuantiles,
  \(D_1 \ldots D_9\), que se llaman \textbf{deciles}. Cada intervalo
  contiene el 10\% de las observaciones.
\item
  Particiones en 100 intervalos: hay noventa y nueve 100-cuantiles,
  \(C_1 \ldots C_{99}\), que se llaman \textbf{centiles}. Cada intervalo
  contiene el 1\% de las observaciones.
\end{itemize}

    Los cuantiles proporcionan una útil interpetación en términos de la
\textbf{frecuencia acumulada relativa}, esto es del \emph{porcentaje de
observaciones que quedan por debajo}.

\begin{itemize}
\tightlist
\item
  La mediana tiene \emph{aproximadamente} el 50\% de las observaciones
  por debajo de su valor.
\item
  En cuanto a los cuartiles:
\item
  El cuartil \(Q_1 \equiv Q_{25\%}\) tiene \emph{aproximadamente} el
  25\% de las muestras por debajo.
\item
  El cuartil \(Q_2 \equiv Q_{50\%}\) coincide con la mediana y tiene
  \emph{aproximadamente} el 50\% de las observaciones por debajo.
\item
  El cuartil \(Q_3 \equiv Q_{75\%}\) coincide con la mediana y tiene
  \emph{aproximadamente} el 75\% de las observaciones por debajo.
\item
  A veces se habla del cuartil \(Q_0\), valor mínimo sin ninguna
  observación por debajo, y del cuartil \(Q_4\), valor máximo con todas
  las observaciones por debajo.
\item
  El planteamiento es el mismo con otros cuatiles. Por ejemplo:
\item
  El decil \(D_3 \equiv D_{30\%}\) tiene el 30\% de las observaciones
  por debajo.
\item
  El centil \(C_{55} \equiv C_{55\%}\) tiene el 55\% de las
  observaciones por debajo.
\end{itemize}

    Veamos, por ejemplo, los \textbf{deciles} de las poblaciones de los
países del mundo en 2017:

VER TRANSPARENCIA ONLINE

(0.0102, 0.0901]      22
(0.0901, 0.395]       22
(0.395, 1.431]        21
(1.431, 3.717]        22
(3.717, 6.296]        21
(6.296, 10.294]       22
(10.294, 17.113]      21
(17.113, 31.977]      22
(31.977, 63.287]      21
(63.287, 1386.395]    22
Name: 2017, dtype: int64
            
    Igualmente, los \textbf{cuartiles} del año 2017:

(0.0102, 0.8]         54
(0.8, 6.296]          54
(6.296, 24.114]       54
(24.114, 1386.395]    54
Name: 2017, dtype: int64
            
    \subsubsection*{Diagrama de caja y bigote (box and whisker
plot)}\label{diagrama-de-caja-y-bigote-box-and-whisker-plot}

La distribución de las observaciones en el especio muestral puede
ilustrarse mediante un diagrama de cajas (\emph{boxplot}) y bigote
(\emph{whisker}):

\begin{itemize}
\tightlist
\item
  La \textbf{mediana} o segundo intercuantil es, como veremos, un
  \emph{indicador de la tendencia central o localización de la muestra}
  o población estadística.
\item
  La \textbf{caja} se forma con el primer cuartil \(Q1\), la mediana o
  segundo cuartil \(Q2\) y el tercer cuartil.
\item
  La cantidad \(IQR = Q_3-Q_1\) se llama \textbf{rango intercuantil}, y,
  como veremos, es un \emph{indicador de la dispersión o escala de la
  muestra} o población estadística.
\item
  El \textbf{bigote} se forma extendiendo los extremos de la caja un
  máximo de \(1.5\times IQR\), si bien el coeficiente 1.5 puede variarse
  dependiendo de la representación.
\item
  Las observaciones que quedan por debajo o por encima de los bigotes se
  consideran atípicas (\emph{ouliers}) y se representan con círculos.
\end{itemize}

    Representemos el diagramas de caja y bigote de las poblaciones de los
países del mundo en un conjunto de años seleccionados, eliminando todos
los que tengan población mayor de 100 millones para mejorar
visualización.

    \begin{center}
    \adjustimage{max size={0.9\linewidth}{0.9\paperheight}}{1.AnalisisExploratorioDatos_files/1.AnalisisExploratorioDatos_80_0.png}
    \end{center}
    { \hspace*{\fill} \\}
    
    Y ahora considerando todos los países, sin exclusión:

    \begin{center}
    \adjustimage{max size={0.9\linewidth}{0.9\paperheight}}{1.AnalisisExploratorioDatos_files/1.AnalisisExploratorioDatos_82_0.png}
    \end{center}
    { \hspace*{\fill} \\}
    
    \subsubsection*{Coeficientes de forma}\label{coeficientes-de-forma}

La distribución de observaciones se caracteriza con los diversos
estadísticos descriptivos, y se analiza gráficamente, entre otras
posibilidades, con los histogramas y diagramas de caja y bigotes.

\begin{itemize}
\tightlist
\item
  Localización o tendencia central: valor en torno al cual se
  distribuyen las observaciones. Por ejemplo, la mediana o segundio
  cuartil, el rango medio (\emph{midrange}) \(\frac{mín + máx}{2}\), el
  rango medio intercuartil (\emph{midhinge}) \(\frac{Q_1 + Q_3}{2}\) o
  la trimedia (\emph{trimean}) \(\frac{Q_1 + 2Q_2 + Q_3}{4}\).
\item
  Escala o dispersión: medida de cuánto se alejan las observaciones
  entre sí o de su localización central. Por ejemplo, el rango
  \(\left(máx - mín\right)\) o el rango intercuantil
  \(IQR = Q_3 - Q_1\).
\item
  Asimetría: si el histograma se extiende hacia la derecha o hacia la
  izquierda
\item
  Apuntamiento (\emph{kurtosis}): indica si las observaciones se
  distribuyen principalmente en torno a la localización central o, por
  el contrario, si el histograma muestra colas laterales gruesas.
\end{itemize}

    \paragraph{\texorpdfstring{Coeficiente de asimetría
(\emph{skewness})}{Coeficiente de asimetría (skewness)}}\label{coeficiente-de-asimetruxeda-skewness}

El coeficiente de asimetría es:

\begin{itemize}
\tightlist
\item
  \textbf{nulo} si las observaciones se distribuyen simétricamente en
  torno a la localización central. El histograma muestra un eje vertical
  de simetría.
\item
  \textbf{negativo} si las observaciones se extienden hacia la izquierda
  y están más concentradas hacia la derecha. El histograma muestra una
  cola más pronunciada hacia la izquierda.
\item
  \textbf{positivo} si las observaciones se extienden hacia la derecha y
  están más concentradas hacia la izquierda. El histograma muestra una
  cola más pronunciada hacia la derecha.
\end{itemize}

Hay varios estadísticos para hacer el cómputo del coeficiente de
asimetría, aunque no vamos a verlos aquí.

    \paragraph{Coeficiente de curtosis}\label{coeficiente-de-curtosis}

Vamos a trabajar con el \textbf{exceso de curtosis} que es el
coeficiente \emph{habitual} de curtosis, que por definición es siempre
positivo, minorado en 3 para que la distribución Gaussiana o normal
resulte con exceso de curtosis nulo:

\begin{itemize}
\tightlist
\item
  \textbf{nulo} si las observaciones siguen una \emph{distribución
  normal}. Digamos que es una situación intermedia. Estrictamente, la
  curtosis de la distribución normal es 3 pero, como se ha dicho, se
  suele trabajar con el exceso de curtosis, restando 3.
\item
  \textbf{positivo} si las observaciones se concentran en torno a la
  localización central, con colas gruesas y menos frecuencias de valores
  intermedios que en la distribución normal. Cuanto mayor sea el número
  mayor es el apuntamiento, esto es, mayor concentración en torno a la
  localización central y, al tiempo, colas más gruesas.
\item
  \textbf{negativo} si las observaciones se concentran menos en torno a
  la localización central, con colas estrechas o sin ellas y con más
  fecuencias de valores intermedios en que la distribución normal.
  Cuanto menor sea el número menor es el apuntamiento, esto es, menor
  concentración en torno a la localización central, colas más estrechas
  y más valores intermedios.
\end{itemize}

\textbf{Los valores negativos se deben a que se trata de un exceso de
curtosis}, pues estricamente la curtosis siempre es positiva.

VER TRANSPARENCIA ONLINE

Year      1960  1970  1980   1990   2000   2010   2015   2017
midrange 333.5 409.2 490.6  567.6  631.3  668.9  685.6  693.2
median     2.4   2.9   3.5    4.4    5.0    5.7    6.2    6.3
midhinge   4.1   4.9   5.6    6.6    8.5   10.6   11.8   12.5
trimean    3.3   3.9   4.5    5.5    6.8    8.2    9.0    9.4
range    667.1 818.3 981.2 1135.2 1262.6 1337.7 1371.2 1386.4
iqr        7.8   9.2  10.4   12.2   15.8   19.7   22.1   23.3
skew       9.1   9.2   9.2    9.2    9.1    9.0    8.9    8.9
kurt      92.4  94.4  93.8   91.9   89.2   86.3   84.8   84.5
            
        1960  1970  1980  1990  2000  2010  2015  2017
median   2.4   2.9   3.5   4.4   5.0   5.7   6.2   6.3
skew     9.1   9.2   9.2   9.2   9.1   9.0   8.9   8.9
kurt    92.4  94.4  93.8  91.9  89.2  86.3  84.8  84.5
            
    \subsubsection*{Estadísticos de tendencia
central}\label{estaduxedsticos-de-tendencia-central}

Calculan una localización o valor típico en torno al cual se distribuyen
las observaciones o muestras. Ya hemos visto algunes estadísticos de
tendencia central:

\begin{itemize}
\tightlist
\item
  El \textbf{rango medio} (\emph{midrange}) o valor medio de las
  observaciones mínima y máxima \(\frac{mín + máx}{2}\).
\item
  La \textbf{mediana} o segundo cuartil.
\item
  El \textbf{rango medio intercuartil} (\emph{midhinge}) o valor medio
  del primer y tercer cuartil \(\frac{Q_1+Q_3}{2}\).
\end{itemize}

    Veamos ahora:

\begin{itemize}
\tightlist
\item
  La \textbf{media}
\item
  Las \textbf{medias recortadas} o \textbf{truncadas} (\emph{trimmed /
  truncated means})
\item
  La \textbf{media ponderada} (\emph{weighted mean})
\item
  La \textbf{moda}
\item
  El \textbf{valor cuadrático medio} (\emph{Root Mean Square - RMS})
\end{itemize}

    Ilustraremos las definiciones con dos secuencias de observaciones:

\begin{itemize}
\tightlist
\item
  Número impar de observaciones: \(S_1 = (2, 1, 2, 6, 4)\)
\item
  Número par de observaciones: \(S_2 = (2, 1, 2, 6, 4, 99)\). Puede
  verse que esta secuencia es la misma que la anterior, con una
  observación más.
\end{itemize}

Advirtamos que el último valor es muy grande en relación al resto de
muestras. Se trata de un \textbf{valor atípico} (\emph{outlier}), que
debe manejarse con cuidado. ¿Habrá habido un error de medida o es
correcta la muestra? Como veremos, los valores atípicos pueden producir
efectos indeseados en el cómputo de algunos estadísticos.


    \paragraph{Rango medio}\label{rango-medio}

Como hemos visto, es el promedio de los valores mínimo y máximo.

Calculemos el rango medio de las dos secuencias de ejemplo:

\begin{itemize}
\tightlist
\item
  \(min(S_1) = 1 \; , \; max(S_1) = 6 \implies range(S_1) = \frac{1+6}{2} = 3.5\)
\item
  \(min(S_2) = 1 \; , \; max(S_2) = 99 \implies range(S_2) = \frac{1+99}{2} = 50\)
\end{itemize}


    \begin{Verbatim}[commandchars=\\\{\}]
range(S1): 3.5
range(S2): 50.0

    \end{Verbatim}

    \paragraph{Rango medio intercuartil}\label{rango-medio-intercuartil}

Como también hemos visto, se trata del promedio del primer y tercer
cuartiles.

Hagamos el cálculo con las dos secuencias de ejemplo, cuyos valores
debemos ordenar y asignar a cuatro intervalos.

\begin{itemize}
\tightlist
\item
  \(S_1:(1, \mathbf{2}, \mathbf{2}, 4, \mathbf{6}) \implies min = 1, Q_1 = 2, Q_2 = 2, Q_3 = 4, max = 6\)
\end{itemize}

\[midhinge(S_1)=\frac{2+4}{2}=3\]

\begin{itemize}
\tightlist
\item
  \(S_2: (1, \mathbf{2}, \mathbf{2}, 4, 6, \mathbf{99}) \implies mín = 1, Q_1 = 2, Q_2 = 3, Q_3 = 5.5, máx = 99\)
\end{itemize}

\[midhinge(S_2)=\frac{2+5.5}{2}=3.75\]

Como puede verse hay una ambigüedad para asignar observaciones a los
intervalos y calcular los cuartiles. Mostramos el cáculo obtenido en
\emph{Pandas} con interpolación lineal.

    \begin{Verbatim}[commandchars=\\\{\}]
{\color{incolor}In [{\color{incolor}39}]:} \PY{n}{pd}\PY{o}{.}\PY{n}{DataFrame}\PY{p}{(}\PY{p}{\PYZob{}}\PY{l+s+s1}{\PYZsq{}}\PY{l+s+s1}{S\PYZus{}1}\PY{l+s+s1}{\PYZsq{}}\PY{p}{:} \PY{n}{S\PYZus{}1}\PY{o}{.}\PY{n}{quantile}\PY{p}{(}\PY{p}{[}\PY{l+m+mi}{0}\PY{p}{,}\PY{o}{.}\PY{l+m+mi}{25}\PY{p}{,}\PY{o}{.}\PY{l+m+mi}{50}\PY{p}{,}\PY{o}{.}\PY{l+m+mi}{75}\PY{p}{,}\PY{l+m+mi}{1}\PY{p}{]}\PY{p}{)}\PY{p}{,}
                       \PY{l+s+s1}{\PYZsq{}}\PY{l+s+s1}{S\PYZus{}2}\PY{l+s+s1}{\PYZsq{}}\PY{p}{:} \PY{n}{S\PYZus{}2}\PY{o}{.}\PY{n}{quantile}\PY{p}{(}\PY{p}{[}\PY{l+m+mi}{0}\PY{p}{,}\PY{o}{.}\PY{l+m+mi}{25}\PY{p}{,}\PY{o}{.}\PY{l+m+mi}{50}\PY{p}{,}\PY{o}{.}\PY{l+m+mi}{75}\PY{p}{,}\PY{l+m+mi}{1}\PY{p}{]}\PY{p}{)}\PY{p}{\PYZcb{}}\PY{p}{)}
\end{Verbatim}

VER TRANSPARENCIA ONLINE

     S\_1  S\_2
0.0  1.0  1.0
0.2  2.0  2.0
0.5  2.0  3.0
0.8  4.0  5.5
1.0  6.0 99.0
            
    \paragraph{Mediana muestral}\label{mediana-muestral}

Es el valor intermedio de la variable, que tiene tantas observaciones
por encima como por debajo de ella. Para calcularla, debemos ordenar las
observaciones de menor a mayor.

Si el número de observaciones es impar, la mediana muestral se obtiene
sin ambigüedad, pues coincide con la observación que está justo en el
centro de la secuencia ordenada.

Si es el número de observaciones es par hay que obtener la mediana a
partir de las dos observaciones centrales, generalmente por
interpolación lineal.

    Veámoslo para las dos secuencias anteriores, cuyas muestras ahora hemos
de ordenar:

\begin{itemize}
\tightlist
\item
  \(S_1 = (2, 1, 2, 5, 3) \rightarrow (1, 2, \mathbf{2}, 4, 6)\): El
  valor que está en medio es el 2.
\item
  \(S_2 = (2, 1, 2, 5, 3, 99) \rightarrow (1, 2, \mathbf{2}, \mathbf{4}, 6, 99)\):
  Los valores que hay en medio son el 2 y el 4, que debemos promediar.
\end{itemize}

\[median(S_1) = 2\]

\[median(S_2) = \frac{2+4}{2} = 3\]

Adviértase que \textbf{la mediana muestral puede calcularse con
variables categóricas, siempre que sus valores admitan una relación de
orden}. Por ejemplo, si la variable categórica son personas, y puedo
ordenarlas por estatura, puedo determinar quien es la persona con
estatura mediana aunque no conozca los valores numéricos de las
estaturas.

    \paragraph{Media muestral}\label{media-muestral}

Es el \emph{promedio de las observaciones}, calculado como la suma de
todas las observaciones dividida por el número total de las mismas

\[\hat x = mean(x_1,\ldots, x_N) = \frac{1}{N}\sum_{1=1}^N x_i\]

    La media muestral de la primera secuencia, \(S_1\) es:

\[mean(S_1) = \frac{1}{5}(2+1+2+6+4)=15/5 = 3\]

Adviértase como varía la media muestral de la segunda secuencia,
\(S_2\). Esto es debido al \textbf{valor atípico} 114:

\[mean(S_2) = \frac{1}{6}(2+1+2+6+4+99)=114/6 = 19\]


    \paragraph{Medias muestrales recortadas o truncadas (trimmed / truncated
means)}\label{medias-muestrales-recortadas-o-truncadas-trimmed-truncated-means}

Se trata de un conjunto de técnicas que intentan proporcionar
\textbf{estadísticos robustos de localización central}, eliminando
valores atípicos conforme a algún criterio:

\begin{itemize}
\tightlist
\item
  \textbf{Media truncada al n\%}: se descartan el \(n\%\) de las
  observaciones menores y el \(n\%\) de las observaciones mayores,
  calculándose la media muestral de las observaciones restantes.
\item
  \textbf{Media modificada}: se descartan los valores menor y mayor de
  la muestra.
\item
  \textbf{Medias interdecil} e \textbf{intercuartil}: la media muestral
  se restringe a las observaciones que están, respectivamente, entre el
  primer y noveno decil, y entre el primer y tercer cuartil, descartando
  las restantes. Corresponden, respectivamente a las medias truncadas al
  10\% y al 25\%.
\item
  \textbf{Medias \emph{winsorizadas}}: en vez de descartar las
  observaciones, éstas se sustituyen con los valores extremos de las que
  se consideran para el cálculo.
\end{itemize}

    Veamos, por ejemplo, las \textbf{media truncada} al 17\% de la secuencia
\(S_2\). Esta secuencia tiene 6 muestras, de modo que el 17\% es
aproximadamente 1 muestra que habría que quitar de ambos extremos. Por
tanto, coincide con la \textbf{media modificada}:

\[mean_{17\%}(S_2) = \frac{1}{4}(2+2+6+4)=14/4 = 3.5\]

Podemos también calcular la versión \emph{winsorizada}:

\[winsormean_{17\%}(S_2) = \frac{1}{4}(2+2+2+6+4+6)=22/6 = 11/3 \approx 3.667\]

    \paragraph{\texorpdfstring{Media muestral
\emph{ponderada}}{Media muestral ponderada}}\label{media-muestral-ponderada}

Ponderamos cada observación con un coeficiente \(\alpha_i\), que expresa
la confianza que tenemos en cada una de ellas. Si confiamos igualmente
en todas las observaciones, todos los coeficientes son idénticos e
iguales a \(\alpha_i = 1/N\) y resulta la media muestral o promedio
anterior.

\[\hat x = mean_w(x_1,\ldots, x_N) = \sum_{1=1}^N \alpha_i x_i \qquad 0\leq\alpha_i \leq1 \qquad \sum_{i=1}^N \alpha_i = 1\]

    Supongamos que en las secuencias anteriores tenemos plena confianza en
nuestras primeras cinco muestras, pero que es nula en la sexta, pues nos
proporciona un valor atípico. Podemos plantear una ponderación como
sigue:

\[\alpha_1=\alpha_2=\alpha_3=\alpha_4 = \alpha_5 = 0.20 \quad \alpha_6=0 \qquad \alpha_1+\alpha_2+\alpha_3+\alpha_4+\alpha_5=1\]

\[mean_w(S_2) = 0.20 \times 2 + 0.20 \times 1 + 0.20 \times 2 + 0.20 \times 6+ 0.20 \times 4 + 0 \times 99= 3\]

Tal vez sí oterguemos cierta confianza a la última observación pero, en
todo caso, inferior a las anteriores. Por ejemplo:

\[\alpha_1 = \alpha_2 = \alpha_3 = \alpha_4 = \alpha_5 = 0.19 \quad \alpha_6=0.05 \qquad \alpha_1+\alpha_2+\alpha_3+\alpha_4+\alpha_5=1\]

\[mean_w(S_2) = 0.19 \times 2 + 0.18 \times 1 + 0.18 \times 2 + 0.18 \times 6+ 0.18 \times 4 + 0.10 \times 99= 7.8\]


    \paragraph{Moda muestral}\label{moda-muestral}

Es el valor de la variable con mayor frecuencia absoluta. En el ejemplo
anterior, sería el color con mayor número de coches.

Podemos también obtener la moda de nuestras secuencias numéricas de
ejemplo, por ejemplo de la \(S_1 = (2, 1, 2, 5, 3)\). Veamos las
frecuencias absolutas de cada valor:

\begin{itemize}
\tightlist
\item
  El \(1\) aparece 1 vez: \(F_1 = 1\)
\item
  El \(2\) aparece 2 veces: \(F_2 = 2\)
\item
  El \(3\) aparece 1 vez: \(F_3 = 1\)
\item
  El \(4\) no aparece: \(F_4 = 0\)
\item
  El \(5\) aparece 1 vez: \(F_5 = 1\)
\end{itemize}

Por tanto, la moda corresponde al valor \(2\). Sin embargo, si
tuviéramos la secuencia \(S_3 = (2, 1, 2, 5, 5)\) tendríamos dos modas,
el \(2\) y el \(5\), pues ambas aparecen dos veces


    \paragraph{Medias de valores absolutos y media
cuadrática}\label{medias-de-valores-absolutos-y-media-cuadruxe1tica}

Si las observaciones pueden ser positivas y negativas las medidas de
tendencia central compensan ambos tipos de valores, pudiéndose desplazar
hacia el cero.

Considérese, por ejemplo, la secuencia
\(S_4 =\left(-3, -1, 2, 4, 3, 1, -2, -4 \right)\), cuyo valor medio es:

\[mean(S_4) = \frac{(-3) + (-1) + 2 + 4 + 3 + 1 + (-2) + (-4)}{8}= 0\]

Una media muy pequeña, o nula, puede deberse tanto a que las
observaciones sean muy pequeñas, o nulas, o a que se compensen valores
positivos y negativos.

En ocasiones nops interesa distinguir ambas situaciones, para lo que las
observcaiones con valores negativos se convierten en otras con valores
positivos.

    Definimos la \textbf{media de valores absolutos} como:

\[mean_{abs}(x_1,\ldots, x_N) = \frac{1}{N}\sum_{1=1}^N \|x_i\|\].

    Y la \textbf{media cuadrática} (\emph{root mean square} ó \emph{rms})
como:

\[rms(x_1,\ldots, x_N) = \sqrt{\frac{1}{N}\sum_{1=1}^N x_i^2}\].

    \begin{Verbatim}[commandchars=\\\{\}]
mean(S4) = 0.0
mean\_abs(S4) = 2.5
rms(S4) = 2.7386127875258306

    \end{Verbatim}

    \subsubsection*{Estadísticos de
dispersión}\label{estaduxedsticos-de-dispersiuxf3n}

Ya hemos visto algunos estadísticos importantes de dispersión:

\begin{itemize}
\tightlist
\item
  El \textbf{rango} que es la diferecnia entre los valores máximo y
  mínimos observados.
\item
  El \textbf{rango intercuartil} \(IQR = Q_3-Q_1\). Podemos utilizar
  otros rangos intercuantiles. Por ejemplo, el \textbf{rango D1-D9} o el
  \textbf{rango C15-C85}.
\end{itemize}

Estos estadísticos cuantifican la dispersión fijándose directamente en
la extensión de los valores que toman las observaciones, o rango de las
mismas.

    Una forma alternativa de cuantificar la dispersión es considerar cuánto
se separan las observaciones de una medida de localización o tendencia
central \(<x>\) (media, mediana, moda,...) de las observaciones \(x_i\).

Para ello se contruye, a partir de la secuencia de observaciones y de la
medida de localización central elegida, una nueva \textbf{secuencia de
distancias absolutas}

\[ S_{AD}(x_i) = \|x_i - <x>\| \; , \; i = 1 \ldots N\]

\textbf{La dispersión se mide computando una medida de tendencia central
la secuencia de distancias absolutas}.

    Hay muchas posibles medidas de la dispersión:

\begin{itemize}
\tightlist
\item
  \textbf{Desviación absoluta media (\emph{mean absolute deviation} ó
  \emph{MAD})} alrededor de la media, de la mediana, de la moda o de
  cualquier otra medida de localización central. Se calcula con la
  \textbf{media} de tales distancias absolutas.
\item
  \textbf{Desviación absoluta mediana} alrededor de la media, de la
  mediana, de la moda o de cualquier otra medida de localización
  central. Se calcula con la \textbf{mediana} de tales distancias
  absolutas.
\item
  \textbf{Desviación absoluta máxima} alrededor de la media, de la
  mediana, de la moda o de cualquier otra medida de localización
  central. En este caso simplemente se coge la distancia absoluta mayor.
\end{itemize}

    \paragraph{\texorpdfstring{Desviación típica (\emph{standard deviation}
ó \emph{sdv}) y varianza
muestrales}{Desviación típica (standard deviation ó sdv) y varianza muestrales}}\label{desviaciuxf3n-tuxedpica-standard-deviation-uxf3-sdv-y-varianza-muestrales}

Primeramente se calcula el valor medio de las observaciones, y la
distancia de todas ellas al mismo. La \textbf{desviación típica
muestral} es simplemente la media cuadrática de las distancias de cada
observación al valor medio \(\hat x = \frac{1}{N}\sum x_i\).

\[ s = \sqrt{\frac{1}{N-1} \sum_{i=1}^N \left(x_i - \hat x\right)^2}\]

La división por \(N-1\) asegura un \textbf{estimador insesgado}, pues se
utiliza una media muestral computada con las mismas observaciones.

\textbf{La varianza muestral es el cuadrado de la desviación típica
muestral}.

    \paragraph{Coeficiente de variación}\label{coeficiente-de-variaciuxf3n}

La dispersión de las observaciones debe entenderse en relación a la
localización o tendencia central de las mismas. Para ello se definen
distintos coeficientes de variuación, como razón entre la medida de
dispersión y la de tendencia central. El más habitual utiliza la
desviación típica y la media

\[c_v = \frac{s}{\hat x}\]

Podemos utilizar una definición del coeficiente que sea \textbf{robusta}
(poca influencia de valores atípicos), por ejemplo:

\[c_v = \frac{IQR}{midhinge}\]

    Veamos cómo resulta la desviación típica y el coeficiente de variación
para las cuatro secuencias de ejemplo:

    \begin{Verbatim}[commandchars=\\\{\}]
S1:	mean = 3.0 	std = 2.00 	cv = 0.67
S2:	mean = 19.0 	std = 39.23 	cv = 2.06
S3:	mean = 3.0 	std = 1.87 	cv = 0.62
S4:	mean = 0.0 	std = 2.93 	cv = 0.98

    \end{Verbatim}

    \subsubsection*{Estadísticos de
resumen}\label{estaduxedsticos-de-resumen}

En la práctica, inicialmente se selecciona un número pequeño de
estadísticos de resumen que nos permita hacer una primera evaluación de
las observaciones, junto con el histograma.

Veamos los estadísticos de resumen para las poblaciones d elos países
del mundo:

VER TRANSPARENCIA ONLINE
            
    \subsection*{Series temporales}\label{series-temporales}

Estudiemos ahora la evolución temporal de la población de los países del
mundo. La población de cada año supone una variable estadística de la
que se toma una observación para cada país. Estamos interesados en
estudiar el comportamiento de la población muestral a lo largo del
tiempo.

Veamos primero la evolución de las observaciones correspondientes a una
selección de países:

    \begin{center}
    \adjustimage{max size={0.9\linewidth}{0.9\paperheight}}{1.AnalisisExploratorioDatos_files/1.AnalisisExploratorioDatos_122_0.png}
    \end{center}
    { \hspace*{\fill} \\}
    
    Dado que para cada año tenemos una variable estadística, podemos
calcular estadísticos para las observaciones de cada uno de ellos.

Veamos primero cómo evoluciona el valor medio y la desviación típica de
las poblaciones de los países del mundo:

    \begin{center}
    \adjustimage{max size={0.9\linewidth}{0.9\paperheight}}{1.AnalisisExploratorioDatos_files/1.AnalisisExploratorioDatos_124_0.png}
    \end{center}
    { \hspace*{\fill} \\}
    
    Y veamos ahora la evolución del coeficiente de variación, el coeficiente
de asimetría y el coeficiente de curtosis de las poblaciones de los
países del mundo:

    \begin{center}
    \adjustimage{max size={0.9\linewidth}{0.9\paperheight}}{1.AnalisisExploratorioDatos_files/1.AnalisisExploratorioDatos_126_0.png}
    \end{center}
    { \hspace*{\fill} \\}
    
    \subsubsection*{Estacionariedad}\label{estacionariedad}

Un conjunto de series temporales de variables estadísticas, que, por
tanto, para cada instante de tiempo proporciona una subpoblación
muestral de observaciones, se dice que es \textbf{estacionaria} si sus
estadísticos se mantienen constantes a lo largo del tiempo.

En sentido amplio, podemos referirnos a la estacionariedad en relación a
un estadístico. Por ejemplo, \textbf{estacionariedad en la media} o
\textbf{estacionariedad en la varianza}, si dichos estadísticos se
mantienen constantes.

En las representaciones anteriores vemos que la población de los países
del mundo no es estacionaria ni en la media ni en la varianza.

    Generalmente estamos interesados en trabajar con series estacionarias.
Por ello a veces se transforman las series originales para aproximar un
comportamiento estacionario. Una transformación que se usa habitualmente
es la primera diferencia, o incremento entre dos instantes de tiempo
sucesivos:

    \begin{center}
    \adjustimage{max size={0.9\linewidth}{0.9\paperheight}}{1.AnalisisExploratorioDatos_files/1.AnalisisExploratorioDatos_129_0.png}
    \end{center}
    { \hspace*{\fill} \\}
    
    Veamos cómo evolucionan la media y la desviación típica de la primera
diferencia \(x[n+1]-x[n]\):

    \begin{center}
    \adjustimage{max size={0.9\linewidth}{0.9\paperheight}}{1.AnalisisExploratorioDatos_files/1.AnalisisExploratorioDatos_131_0.png}
    \end{center}
    { \hspace*{\fill} \\}
    
    También suele resultar útil calcular la tasa de variación
\(\frac{x[n+1]-x[n]}{x[n]}\):

<pandas.io.formats.style.Styler at 0x7fef9d2104e0>
            
    \begin{center}
    \adjustimage{max size={0.9\linewidth}{0.9\paperheight}}{1.AnalisisExploratorioDatos_files/1.AnalisisExploratorioDatos_134_0.png}
    \end{center}
    { \hspace*{\fill} \\}
    
    Veamos cómo evolucionan la media y la desviación típica de la tasa de
variación:

    \begin{center}
    \adjustimage{max size={0.9\linewidth}{0.9\paperheight}}{1.AnalisisExploratorioDatos_files/1.AnalisisExploratorioDatos_136_0.png}
    \end{center}
    { \hspace*{\fill} \\}
    
    \subsection{Relación entre variables
estadísticas}\label{relaciuxf3n-entre-variables-estaduxedsticas}

Dos variables estadísticas pueden variar independientemente entre sí, o
tener una relación.

El \textbf{diagrama de dispersión} muestra observaciones
correspondientes de ambas variables sobre el plano, asignando la
coordonada \(x\) a una y la \(y\) a la otra. Permite obtener rápidamente
una intuación de si existe relación entre ambas variables y cómo es.

El \textbf{coeficiente de correlación} es un estadístico que muestra la
\textbf{relación lineal} entre ambas variables, con valores entre -1 y
1. Un valor de 0 indica que no hay relación lineal. Un valor de -1
indica una relación lineal perfecta negativa (cuando una variable crece
la otra decrece) y un valor de 1 una relación lineal perfecta positiva.

    Diagrama de dispersión y coeficiente de correlación entre las
observaciones de las poblaciones de todos los países del mundo en los
años 2015 y 2017:

Year  2015  2017
Year            
2015   1.0   1.0
2017   1.0   1.0
            
    \begin{center}
    \adjustimage{max size={0.9\linewidth}{0.9\paperheight}}{1.AnalisisExploratorioDatos_files/1.AnalisisExploratorioDatos_139_1.png}
    \end{center}
    { \hspace*{\fill} \\}
    
    Diagrama de dispersión y coeficiente de correlación entre las
observaciones de las poblaciones de todos los países del mundo en los
años 1965 y 2015:

Year  1965  2015
Year            
1965   1.0   1.0
2015   1.0   1.0
            
    \begin{center}
    \adjustimage{max size={0.9\linewidth}{0.9\paperheight}}{1.AnalisisExploratorioDatos_files/1.AnalisisExploratorioDatos_141_1.png}
    \end{center}
    { \hspace*{\fill} \\}
    
    \subsection*{Análisis del producto interior bruto (PIB)
mundial}\label{anuxe1lisis-del-producto-interior-bruto-pib-mundial}

<pandas.io.formats.style.Styler at 0x7fef9cfb1a90>
            
<pandas.io.formats.style.Styler at 0x7fef9ececac8>
            
    \begin{Verbatim}[commandchars=\\\{\}]
['AUS', 'BRA', 'CAN', 'CHN', 'DEU', 'ESP', 'FRA', 'GBR', 'IND', 'ITA', 'JPN', 'MEX', 'NLD', 'RUS', 'USA']

    \end{Verbatim}

       1    2    3    4    5    6    7    8    9    10
Year                                                  
1960  USA  JPN  GBR  FRA  ITA  CAN  BRA  ESP  AUS  NLD
1970  USA  JPN  DEU  FRA  GBR  ITA  CAN  ESP  BRA  AUS
1980  USA  JPN  DEU  FRA  ITA  GBR  BRA  CAN  ESP  MEX
1990  USA  JPN  DEU  FRA  ITA  GBR  RUS  BRA  CAN  ESP
2000  USA  JPN  DEU  FRA  CHN  GBR  ITA  BRA  CAN  ESP
2010  USA  CHN  JPN  DEU  FRA  GBR  BRA  ITA  IND  CAN
2015  USA  CHN  JPN  DEU  FRA  GBR  BRA  IND  ITA  CAN
2017  USA  CHN  JPN  DEU  FRA  GBR  IND  BRA  ITA  CAN
            
Country Code  CHN  IND  USA  RUS  NGA  JPN  DEU  GBR  FRA  ESP
Year                                                          
1960         14.0 12.0  1.0  nan 25.0  2.0  nan  3.0  4.0  8.0
1970         17.0 14.0  1.0  nan 27.0  2.0  3.0  5.0  4.0  8.0
1980         15.0 16.0  1.0  nan 29.0  2.0  3.0  6.0  4.0  9.0
1990         11.0 15.0  1.0  7.0 38.0  2.0  3.0  6.0  4.0 10.0
2000          5.0 14.0  1.0 11.0 40.0  2.0  3.0  6.0  4.0 10.0
2010          2.0  9.0  1.0 11.0 30.0  3.0  4.0  6.0  5.0 12.0
2015          2.0  8.0  1.0 11.0 26.0  3.0  4.0  6.0  5.0 12.0
2017          2.0  7.0  1.0 11.0 26.0  3.0  4.0  6.0  5.0 12.0
            
Year         1960    1970    1980    1990    2000    2010    2015    2017
LC Total   6927.5 13015.8 18670.8 26126.6 33952.3 42870.0 49271.2 52056.2
WLD Total  9025.8 17209.5 25612.0 37286.8 49611.5 65390.2 74628.2 78710.0
LC/WLD (\%)   76.8    75.6    72.9    70.1    68.4    65.6    66.0    66.1
            
    \begin{center}
    \adjustimage{max size={0.9\linewidth}{0.9\paperheight}}{1.AnalisisExploratorioDatos_files/1.AnalisisExploratorioDatos_148_0.png}
    \end{center}
    { \hspace*{\fill} \\}
    
    \begin{center}
    \adjustimage{max size={0.9\linewidth}{0.9\paperheight}}{1.AnalisisExploratorioDatos_files/1.AnalisisExploratorioDatos_149_0.png}
    \end{center}
    { \hspace*{\fill} \\}
    
        1960  1970  1980  1990  2000  2010  2015  2017
mean   100.3 153.7 188.3 223.3 259.7 322.1 382.7 423.2
median   8.9  10.2  11.7  12.0  15.3  20.3  26.3  32.2
skew     7.3   6.9   7.2   7.9   9.1   8.7   8.2   7.9
kurt    60.8  55.5  62.3  74.4  99.0  91.3  80.5  73.6
            
    \begin{center}
    \adjustimage{max size={0.9\linewidth}{0.9\paperheight}}{1.AnalisisExploratorioDatos_files/1.AnalisisExploratorioDatos_151_0.png}
    \end{center}
    { \hspace*{\fill} \\}
    
    \begin{center}
    \adjustimage{max size={0.9\linewidth}{0.9\paperheight}}{1.AnalisisExploratorioDatos_files/1.AnalisisExploratorioDatos_152_0.png}
    \end{center}
    { \hspace*{\fill} \\}
    
    \begin{center}
    \adjustimage{max size={0.9\linewidth}{0.9\paperheight}}{1.AnalisisExploratorioDatos_files/1.AnalisisExploratorioDatos_153_0.png}
    \end{center}
    { \hspace*{\fill} \\}
    
    \begin{center}
    \adjustimage{max size={0.9\linewidth}{0.9\paperheight}}{1.AnalisisExploratorioDatos_files/1.AnalisisExploratorioDatos_154_0.png}
    \end{center}
    { \hspace*{\fill} \\}
    
    \begin{center}
    \adjustimage{max size={0.9\linewidth}{0.9\paperheight}}{1.AnalisisExploratorioDatos_files/1.AnalisisExploratorioDatos_155_0.png}
    \end{center}
    { \hspace*{\fill} \\}
    
    \begin{center}
    \adjustimage{max size={0.9\linewidth}{0.9\paperheight}}{1.AnalisisExploratorioDatos_files/1.AnalisisExploratorioDatos_156_0.png}
    \end{center}
    { \hspace*{\fill} \\}
    
            Population  GDP
Population         1.0  0.5
GDP                0.5  1.0
            
    \subsection*{Análisis de la renta per cápita (PIBpc)
mundial}\label{anuxe1lisis-de-la-renta-per-cuxe1pita-pibpc-mundial}

<pandas.io.formats.style.Styler at 0x7fef9c9e12b0>
            
<pandas.io.formats.style.Styler at 0x7fef9c71bb70>
            
    \begin{Verbatim}[commandchars=\\\{\}]
['AND', 'ARE', 'AUS', 'BHS', 'BMU', 'BRN', 'CAN', 'CHE', 'DNK', 'IMN', 'IRL', 'JPN', 'KWT', 'LIE', 'LUX', 'MAC', 'MCO', 'NLD', 'NOR', 'QAT', 'SAU', 'SGP', 'SMR', 'SWE', 'USA']

    \end{Verbatim}

       1    2    3    4    5    6    7    8    9    10
Year                                                  
1960  BMU  LUX  NOR  DNK  AUS  SWE  BHS  CAN  USA  NLD
1970  AND  BMU  LUX  NOR  DNK  SWE  AUS  BHS  CAN  NLD
1980  ARE  BMU  BRN  CHE  NOR  LUX  AND  SAU  DNK  CAN
1990  ARE  LUX  CHE  BMU  NOR  DNK  JPN  SWE  BRN  CAN
2000  LUX  NOR  BMU  SMR  CHE  ARE  QAT  DNK  NLD  USA
2010  MCO  LIE  LUX  BMU  NOR  CHE  IMN  QAT  SMR  DNK
2015  LUX  NOR  IMN  CHE  QAT  IRL  DNK  SWE  AUS  MAC
2017  LUX  NOR  CHE  IRL  QAT  DNK  SWE  MAC  AUS  SGP
            
Year          1960  1970  1980  1990  2000  2010  2015  2017
Country Code                                                
CHN           89.0 111.0 131.0 138.0 129.0 110.0  93.0  81.0
IND           83.0 103.0 129.0 145.0 161.0 157.0 149.0 140.0
USA            9.0  11.0  14.0  11.0  10.0  16.0  13.0  12.0
RUS            nan   nan   nan  45.0  72.0  72.0  66.0  61.0
NGA           52.0  68.0  88.0 117.0 143.0 145.0 141.0 134.0
JPN           19.0  18.0  21.0   7.0  16.0  22.0  17.0  17.0
DEU            nan  16.0  20.0  19.0  22.0  26.0  20.0  19.0
GBR           11.0  20.0  25.0  22.0  27.0  30.0  25.0  23.0
FRA           14.0  15.0  18.0  18.0  21.0  27.0  24.0  22.0
ESP           20.0  25.0  29.0  27.0  32.0  38.0  34.0  30.0
            
    \begin{center}
    \adjustimage{max size={0.9\linewidth}{0.9\paperheight}}{1.AnalisisExploratorioDatos_files/1.AnalisisExploratorioDatos_163_0.png}
    \end{center}
    { \hspace*{\fill} \\}
    
    \begin{center}
    \adjustimage{max size={0.9\linewidth}{0.9\paperheight}}{1.AnalisisExploratorioDatos_files/1.AnalisisExploratorioDatos_164_0.png}
    \end{center}
    { \hspace*{\fill} \\}
    
        1960  1970  1980  1990  2000  2010  2015  2017
mean     4.9   7.2  10.1   9.8  12.0  15.5  14.4  14.2
median   1.5   2.2   3.1   3.1   3.5   5.4   6.1   5.8
skew     1.8   1.7   3.1   2.1   2.1   2.7   2.0   2.1
kurt     2.5   2.4  14.0   4.4   4.5   9.5   4.4   4.5
            
    \begin{Verbatim}[commandchars=\\\{\}]
{\color{incolor}In [{\color{incolor}113}]:} \PY{n}{gdp\PYZus{}pc\PYZus{}countries}\PY{o}{.}\PY{n}{T}\PY{p}{[}\PY{n}{selected\PYZus{}countries}\PY{p}{]}\PY{o}{.}\PY{n}{plot}\PY{p}{(}\PY{n}{figsize}\PY{o}{=}\PY{p}{(}\PY{l+m+mi}{12}\PY{p}{,}\PY{l+m+mi}{4}\PY{p}{)}\PY{p}{)}
          \PY{n}{plt}\PY{o}{.}\PY{n}{ylabel}\PY{p}{(}\PY{l+s+s1}{\PYZsq{}}\PY{l+s+s1}{GDPpc (thousandss)}\PY{l+s+s1}{\PYZsq{}}\PY{p}{)}\PY{p}{,} \PY{n}{plt}\PY{o}{.}\PY{n}{title}\PY{p}{(}\PY{l+s+s1}{\PYZsq{}}\PY{l+s+s1}{GDPpc evolution 1960\PYZhy{}2017 (selected countries)}\PY{l+s+s1}{\PYZsq{}}\PY{p}{)}\PY{p}{;}
\end{Verbatim}

    \begin{center}
    \adjustimage{max size={0.9\linewidth}{0.9\paperheight}}{1.AnalisisExploratorioDatos_files/1.AnalisisExploratorioDatos_166_0.png}
    \end{center}
    { \hspace*{\fill} \\}
    
    \begin{center}
    \adjustimage{max size={0.9\linewidth}{0.9\paperheight}}{1.AnalisisExploratorioDatos_files/1.AnalisisExploratorioDatos_167_0.png}
    \end{center}
    { \hspace*{\fill} \\}
    
    \begin{center}
    \adjustimage{max size={0.9\linewidth}{0.9\paperheight}}{1.AnalisisExploratorioDatos_files/1.AnalisisExploratorioDatos_168_0.png}
    \end{center}
    { \hspace*{\fill} \\}
    
    \begin{center}
    \adjustimage{max size={0.9\linewidth}{0.9\paperheight}}{1.AnalisisExploratorioDatos_files/1.AnalisisExploratorioDatos_169_0.png}
    \end{center}
    { \hspace*{\fill} \\}
    
            Population  GDPpc
Population         1.0   -0.0
GDPpc             -0.0    1.0
            
    \subsection*{Análisis de la esperanza de vida de los países del
mundo}\label{anuxe1lisis-de-la-esperanza-de-vida-de-los-pauxedses-del-mundo}

<pandas.io.formats.style.Styler at 0x7fefa38fb358>
            
<pandas.io.formats.style.Styler at 0x7fef9ebbbb00>
            
    \begin{Verbatim}[commandchars=\\\{\}]
['AUS', 'BGR', 'CAN', 'CHE', 'CHI', 'CYM', 'CYP', 'DNK', 'ESP', 'FRA', 'FRO', 'GBR', 'GRC', 'HKG', 'ISL', 'ISR', 'ITA', 'JPN', 'LIE', 'LTU', 'MAC', 'NLD', 'NOR', 'NZL', 'SGP', 'SMR', 'SWE']

    \end{Verbatim}

       1    2    3    4    5    6    7    8    9    10
Year                                                  
1960  NOR  ISL  NLD  SWE  DNK  CHE  NZL  CAN  GBR  AUS
1970  SWE  NOR  ISL  NLD  DNK  CHE  CAN  CYP  ESP  GBR
1980  ISL  JPN  NLD  SWE  NOR  CHE  ESP  CAN  CYP  HKG
1990  JPN  ISL  SWE  HKG  CAN  MAC  CHE  AUS  ITA  GRC
2000  JPN  HKG  MAC  ITA  CHE  ISL  SWE  CAN  AUS  FRA
2010  HKG  JPN  MAC  CHE  CYM  ITA  ISL  LIE  AUS  FRA
2015  HKG  JPN  MAC  CHE  ESP  SGP  LIE  ITA  ISL  AUS
            
Year          1960  1970  1980  1990  2000  2010  2015
Country Code                                          
CHN          141.0 104.0  83.0  86.0  78.0  65.0  66.0
IND          157.0 149.0 147.0 150.0 145.0 143.0 142.0
USA           18.0  25.0  22.0  27.0  35.0  39.0  43.0
RUS           41.0  52.0  80.0  91.0 134.0 131.0 124.0
NGA          171.0 173.0 178.0 187.0 197.0 195.0 195.0
JPN           38.0  11.0   2.0   1.0   1.0   2.0   2.0
DEU           21.0  27.0  28.0  26.0  21.0  29.0  33.0
GBR            9.0  10.0  20.0  20.0  25.0  24.0  30.0
FRA           14.0  13.0  13.0  14.0  10.0  10.0  13.0
ESP           24.0   9.0   7.0  12.0  11.0  11.0   5.0
            
    \begin{center}
    \adjustimage{max size={0.9\linewidth}{0.9\paperheight}}{1.AnalisisExploratorioDatos_files/1.AnalisisExploratorioDatos_176_0.png}
    \end{center}
    { \hspace*{\fill} \\}
    
    \begin{center}
    \adjustimage{max size={0.9\linewidth}{0.9\paperheight}}{1.AnalisisExploratorioDatos_files/1.AnalisisExploratorioDatos_177_0.png}
    \end{center}
    { \hspace*{\fill} \\}
    
        1960  1970  1980  1990  2000  2010  2015
mean    53.9  58.2  61.9  64.9  67.1  70.5  72.0
median  54.7  60.2  64.6  68.1  70.3  72.8  73.9
skew    -0.2  -0.4  -0.7  -0.8  -0.8  -0.7  -0.6
kurt    -1.2  -1.1  -0.4  -0.2  -0.4  -0.4  -0.4
            
    \begin{center}
    \adjustimage{max size={0.9\linewidth}{0.9\paperheight}}{1.AnalisisExploratorioDatos_files/1.AnalisisExploratorioDatos_179_0.png}
    \end{center}
    { \hspace*{\fill} \\}
    
    \begin{center}
    \adjustimage{max size={0.9\linewidth}{0.9\paperheight}}{1.AnalisisExploratorioDatos_files/1.AnalisisExploratorioDatos_180_0.png}
    \end{center}
    { \hspace*{\fill} \\}
    
    \begin{center}
    \adjustimage{max size={0.9\linewidth}{0.9\paperheight}}{1.AnalisisExploratorioDatos_files/1.AnalisisExploratorioDatos_181_0.png}
    \end{center}
    { \hspace*{\fill} \\}
    
                  Life Expectation  GDP per Capita
Life Expectation               1.0             0.6
GDP per Capita                 0.6             1.0
            

    % Add a bibliography block to the postdoc
    
    
    
    \end{document}
