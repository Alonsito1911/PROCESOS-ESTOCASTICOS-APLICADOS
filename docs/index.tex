
% Default to the notebook output style

    


% Inherit from the specified cell style.




    
\documentclass[11pt]{article}

    
    
    \usepackage[T1]{fontenc}
    % Nicer default font (+ math font) than Computer Modern for most use cases
    \usepackage{mathpazo}

    % Basic figure setup, for now with no caption control since it's done
    % automatically by Pandoc (which extracts ![](path) syntax from Markdown).
    \usepackage{graphicx}
    % We will generate all images so they have a width \maxwidth. This means
    % that they will get their normal width if they fit onto the page, but
    % are scaled down if they would overflow the margins.
    \makeatletter
    \def\maxwidth{\ifdim\Gin@nat@width>\linewidth\linewidth
    \else\Gin@nat@width\fi}
    \makeatother
    \let\Oldincludegraphics\includegraphics
    % Set max figure width to be 80% of text width, for now hardcoded.
    \renewcommand{\includegraphics}[1]{\Oldincludegraphics[width=.8\maxwidth]{#1}}
    % Ensure that by default, figures have no caption (until we provide a
    % proper Figure object with a Caption API and a way to capture that
    % in the conversion process - todo).
    \usepackage{caption}
    \DeclareCaptionLabelFormat{nolabel}{}
    \captionsetup{labelformat=nolabel}

    \usepackage{adjustbox} % Used to constrain images to a maximum size 
    \usepackage{xcolor} % Allow colors to be defined
    \usepackage{enumerate} % Needed for markdown enumerations to work
    \usepackage{geometry} % Used to adjust the document margins
    \usepackage{amsmath} % Equations
    \usepackage{amssymb} % Equations
    \usepackage{textcomp} % defines textquotesingle
    % Hack from http://tex.stackexchange.com/a/47451/13684:
    \AtBeginDocument{%
        \def\PYZsq{\textquotesingle}% Upright quotes in Pygmentized code
    }
    \usepackage{upquote} % Upright quotes for verbatim code
    \usepackage{eurosym} % defines \euro
    \usepackage[mathletters]{ucs} % Extended unicode (utf-8) support
    \usepackage[utf8x]{inputenc} % Allow utf-8 characters in the tex document
    \usepackage{fancyvrb} % verbatim replacement that allows latex
    \usepackage{grffile} % extends the file name processing of package graphics 
                         % to support a larger range 
    % The hyperref package gives us a pdf with properly built
    % internal navigation ('pdf bookmarks' for the table of contents,
    % internal cross-reference links, web links for URLs, etc.)
    \usepackage{hyperref}
    \usepackage{longtable} % longtable support required by pandoc >1.10
    \usepackage{booktabs}  % table support for pandoc > 1.12.2
    \usepackage[inline]{enumitem} % IRkernel/repr support (it uses the enumerate* environment)
    \usepackage[normalem]{ulem} % ulem is needed to support strikethroughs (\sout)
                                % normalem makes italics be italics, not underlines
    

    
    
    % Colors for the hyperref package
    \definecolor{urlcolor}{rgb}{0,.145,.698}
    \definecolor{linkcolor}{rgb}{.71,0.21,0.01}
    \definecolor{citecolor}{rgb}{.12,.54,.11}

    % ANSI colors
    \definecolor{ansi-black}{HTML}{3E424D}
    \definecolor{ansi-black-intense}{HTML}{282C36}
    \definecolor{ansi-red}{HTML}{E75C58}
    \definecolor{ansi-red-intense}{HTML}{B22B31}
    \definecolor{ansi-green}{HTML}{00A250}
    \definecolor{ansi-green-intense}{HTML}{007427}
    \definecolor{ansi-yellow}{HTML}{DDB62B}
    \definecolor{ansi-yellow-intense}{HTML}{B27D12}
    \definecolor{ansi-blue}{HTML}{208FFB}
    \definecolor{ansi-blue-intense}{HTML}{0065CA}
    \definecolor{ansi-magenta}{HTML}{D160C4}
    \definecolor{ansi-magenta-intense}{HTML}{A03196}
    \definecolor{ansi-cyan}{HTML}{60C6C8}
    \definecolor{ansi-cyan-intense}{HTML}{258F8F}
    \definecolor{ansi-white}{HTML}{C5C1B4}
    \definecolor{ansi-white-intense}{HTML}{A1A6B2}

    % commands and environments needed by pandoc snippets
    % extracted from the output of `pandoc -s`
    \providecommand{\tightlist}{%
      \setlength{\itemsep}{0pt}\setlength{\parskip}{0pt}}
    \DefineVerbatimEnvironment{Highlighting}{Verbatim}{commandchars=\\\{\}}
    % Add ',fontsize=\small' for more characters per line
    \newenvironment{Shaded}{}{}
    \newcommand{\KeywordTok}[1]{\textcolor[rgb]{0.00,0.44,0.13}{\textbf{{#1}}}}
    \newcommand{\DataTypeTok}[1]{\textcolor[rgb]{0.56,0.13,0.00}{{#1}}}
    \newcommand{\DecValTok}[1]{\textcolor[rgb]{0.25,0.63,0.44}{{#1}}}
    \newcommand{\BaseNTok}[1]{\textcolor[rgb]{0.25,0.63,0.44}{{#1}}}
    \newcommand{\FloatTok}[1]{\textcolor[rgb]{0.25,0.63,0.44}{{#1}}}
    \newcommand{\CharTok}[1]{\textcolor[rgb]{0.25,0.44,0.63}{{#1}}}
    \newcommand{\StringTok}[1]{\textcolor[rgb]{0.25,0.44,0.63}{{#1}}}
    \newcommand{\CommentTok}[1]{\textcolor[rgb]{0.38,0.63,0.69}{\textit{{#1}}}}
    \newcommand{\OtherTok}[1]{\textcolor[rgb]{0.00,0.44,0.13}{{#1}}}
    \newcommand{\AlertTok}[1]{\textcolor[rgb]{1.00,0.00,0.00}{\textbf{{#1}}}}
    \newcommand{\FunctionTok}[1]{\textcolor[rgb]{0.02,0.16,0.49}{{#1}}}
    \newcommand{\RegionMarkerTok}[1]{{#1}}
    \newcommand{\ErrorTok}[1]{\textcolor[rgb]{1.00,0.00,0.00}{\textbf{{#1}}}}
    \newcommand{\NormalTok}[1]{{#1}}
    
    % Additional commands for more recent versions of Pandoc
    \newcommand{\ConstantTok}[1]{\textcolor[rgb]{0.53,0.00,0.00}{{#1}}}
    \newcommand{\SpecialCharTok}[1]{\textcolor[rgb]{0.25,0.44,0.63}{{#1}}}
    \newcommand{\VerbatimStringTok}[1]{\textcolor[rgb]{0.25,0.44,0.63}{{#1}}}
    \newcommand{\SpecialStringTok}[1]{\textcolor[rgb]{0.73,0.40,0.53}{{#1}}}
    \newcommand{\ImportTok}[1]{{#1}}
    \newcommand{\DocumentationTok}[1]{\textcolor[rgb]{0.73,0.13,0.13}{\textit{{#1}}}}
    \newcommand{\AnnotationTok}[1]{\textcolor[rgb]{0.38,0.63,0.69}{\textbf{\textit{{#1}}}}}
    \newcommand{\CommentVarTok}[1]{\textcolor[rgb]{0.38,0.63,0.69}{\textbf{\textit{{#1}}}}}
    \newcommand{\VariableTok}[1]{\textcolor[rgb]{0.10,0.09,0.49}{{#1}}}
    \newcommand{\ControlFlowTok}[1]{\textcolor[rgb]{0.00,0.44,0.13}{\textbf{{#1}}}}
    \newcommand{\OperatorTok}[1]{\textcolor[rgb]{0.40,0.40,0.40}{{#1}}}
    \newcommand{\BuiltInTok}[1]{{#1}}
    \newcommand{\ExtensionTok}[1]{{#1}}
    \newcommand{\PreprocessorTok}[1]{\textcolor[rgb]{0.74,0.48,0.00}{{#1}}}
    \newcommand{\AttributeTok}[1]{\textcolor[rgb]{0.49,0.56,0.16}{{#1}}}
    \newcommand{\InformationTok}[1]{\textcolor[rgb]{0.38,0.63,0.69}{\textbf{\textit{{#1}}}}}
    \newcommand{\WarningTok}[1]{\textcolor[rgb]{0.38,0.63,0.69}{\textbf{\textit{{#1}}}}}
    
    
    % Define a nice break command that doesn't care if a line doesn't already
    % exist.
    \def\br{\hspace*{\fill} \\* }
    % Math Jax compatability definitions
    \def\gt{>}
    \def\lt{<}
    % Document parameters
    \title{Estadística y Procesos Estocásticos (EPE)}
    \author{Juan Ruiz Alzola}
   

    % Pygments definitions
    
\makeatletter
\def\PY@reset{\let\PY@it=\relax \let\PY@bf=\relax%
    \let\PY@ul=\relax \let\PY@tc=\relax%
    \let\PY@bc=\relax \let\PY@ff=\relax}
\def\PY@tok#1{\csname PY@tok@#1\endcsname}
\def\PY@toks#1+{\ifx\relax#1\empty\else%
    \PY@tok{#1}\expandafter\PY@toks\fi}
\def\PY@do#1{\PY@bc{\PY@tc{\PY@ul{%
    \PY@it{\PY@bf{\PY@ff{#1}}}}}}}
\def\PY#1#2{\PY@reset\PY@toks#1+\relax+\PY@do{#2}}

\expandafter\def\csname PY@tok@w\endcsname{\def\PY@tc##1{\textcolor[rgb]{0.73,0.73,0.73}{##1}}}
\expandafter\def\csname PY@tok@c\endcsname{\let\PY@it=\textit\def\PY@tc##1{\textcolor[rgb]{0.25,0.50,0.50}{##1}}}
\expandafter\def\csname PY@tok@cp\endcsname{\def\PY@tc##1{\textcolor[rgb]{0.74,0.48,0.00}{##1}}}
\expandafter\def\csname PY@tok@k\endcsname{\let\PY@bf=\textbf\def\PY@tc##1{\textcolor[rgb]{0.00,0.50,0.00}{##1}}}
\expandafter\def\csname PY@tok@kp\endcsname{\def\PY@tc##1{\textcolor[rgb]{0.00,0.50,0.00}{##1}}}
\expandafter\def\csname PY@tok@kt\endcsname{\def\PY@tc##1{\textcolor[rgb]{0.69,0.00,0.25}{##1}}}
\expandafter\def\csname PY@tok@o\endcsname{\def\PY@tc##1{\textcolor[rgb]{0.40,0.40,0.40}{##1}}}
\expandafter\def\csname PY@tok@ow\endcsname{\let\PY@bf=\textbf\def\PY@tc##1{\textcolor[rgb]{0.67,0.13,1.00}{##1}}}
\expandafter\def\csname PY@tok@nb\endcsname{\def\PY@tc##1{\textcolor[rgb]{0.00,0.50,0.00}{##1}}}
\expandafter\def\csname PY@tok@nf\endcsname{\def\PY@tc##1{\textcolor[rgb]{0.00,0.00,1.00}{##1}}}
\expandafter\def\csname PY@tok@nc\endcsname{\let\PY@bf=\textbf\def\PY@tc##1{\textcolor[rgb]{0.00,0.00,1.00}{##1}}}
\expandafter\def\csname PY@tok@nn\endcsname{\let\PY@bf=\textbf\def\PY@tc##1{\textcolor[rgb]{0.00,0.00,1.00}{##1}}}
\expandafter\def\csname PY@tok@ne\endcsname{\let\PY@bf=\textbf\def\PY@tc##1{\textcolor[rgb]{0.82,0.25,0.23}{##1}}}
\expandafter\def\csname PY@tok@nv\endcsname{\def\PY@tc##1{\textcolor[rgb]{0.10,0.09,0.49}{##1}}}
\expandafter\def\csname PY@tok@no\endcsname{\def\PY@tc##1{\textcolor[rgb]{0.53,0.00,0.00}{##1}}}
\expandafter\def\csname PY@tok@nl\endcsname{\def\PY@tc##1{\textcolor[rgb]{0.63,0.63,0.00}{##1}}}
\expandafter\def\csname PY@tok@ni\endcsname{\let\PY@bf=\textbf\def\PY@tc##1{\textcolor[rgb]{0.60,0.60,0.60}{##1}}}
\expandafter\def\csname PY@tok@na\endcsname{\def\PY@tc##1{\textcolor[rgb]{0.49,0.56,0.16}{##1}}}
\expandafter\def\csname PY@tok@nt\endcsname{\let\PY@bf=\textbf\def\PY@tc##1{\textcolor[rgb]{0.00,0.50,0.00}{##1}}}
\expandafter\def\csname PY@tok@nd\endcsname{\def\PY@tc##1{\textcolor[rgb]{0.67,0.13,1.00}{##1}}}
\expandafter\def\csname PY@tok@s\endcsname{\def\PY@tc##1{\textcolor[rgb]{0.73,0.13,0.13}{##1}}}
\expandafter\def\csname PY@tok@sd\endcsname{\let\PY@it=\textit\def\PY@tc##1{\textcolor[rgb]{0.73,0.13,0.13}{##1}}}
\expandafter\def\csname PY@tok@si\endcsname{\let\PY@bf=\textbf\def\PY@tc##1{\textcolor[rgb]{0.73,0.40,0.53}{##1}}}
\expandafter\def\csname PY@tok@se\endcsname{\let\PY@bf=\textbf\def\PY@tc##1{\textcolor[rgb]{0.73,0.40,0.13}{##1}}}
\expandafter\def\csname PY@tok@sr\endcsname{\def\PY@tc##1{\textcolor[rgb]{0.73,0.40,0.53}{##1}}}
\expandafter\def\csname PY@tok@ss\endcsname{\def\PY@tc##1{\textcolor[rgb]{0.10,0.09,0.49}{##1}}}
\expandafter\def\csname PY@tok@sx\endcsname{\def\PY@tc##1{\textcolor[rgb]{0.00,0.50,0.00}{##1}}}
\expandafter\def\csname PY@tok@m\endcsname{\def\PY@tc##1{\textcolor[rgb]{0.40,0.40,0.40}{##1}}}
\expandafter\def\csname PY@tok@gh\endcsname{\let\PY@bf=\textbf\def\PY@tc##1{\textcolor[rgb]{0.00,0.00,0.50}{##1}}}
\expandafter\def\csname PY@tok@gu\endcsname{\let\PY@bf=\textbf\def\PY@tc##1{\textcolor[rgb]{0.50,0.00,0.50}{##1}}}
\expandafter\def\csname PY@tok@gd\endcsname{\def\PY@tc##1{\textcolor[rgb]{0.63,0.00,0.00}{##1}}}
\expandafter\def\csname PY@tok@gi\endcsname{\def\PY@tc##1{\textcolor[rgb]{0.00,0.63,0.00}{##1}}}
\expandafter\def\csname PY@tok@gr\endcsname{\def\PY@tc##1{\textcolor[rgb]{1.00,0.00,0.00}{##1}}}
\expandafter\def\csname PY@tok@ge\endcsname{\let\PY@it=\textit}
\expandafter\def\csname PY@tok@gs\endcsname{\let\PY@bf=\textbf}
\expandafter\def\csname PY@tok@gp\endcsname{\let\PY@bf=\textbf\def\PY@tc##1{\textcolor[rgb]{0.00,0.00,0.50}{##1}}}
\expandafter\def\csname PY@tok@go\endcsname{\def\PY@tc##1{\textcolor[rgb]{0.53,0.53,0.53}{##1}}}
\expandafter\def\csname PY@tok@gt\endcsname{\def\PY@tc##1{\textcolor[rgb]{0.00,0.27,0.87}{##1}}}
\expandafter\def\csname PY@tok@err\endcsname{\def\PY@bc##1{\setlength{\fboxsep}{0pt}\fcolorbox[rgb]{1.00,0.00,0.00}{1,1,1}{\strut ##1}}}
\expandafter\def\csname PY@tok@kc\endcsname{\let\PY@bf=\textbf\def\PY@tc##1{\textcolor[rgb]{0.00,0.50,0.00}{##1}}}
\expandafter\def\csname PY@tok@kd\endcsname{\let\PY@bf=\textbf\def\PY@tc##1{\textcolor[rgb]{0.00,0.50,0.00}{##1}}}
\expandafter\def\csname PY@tok@kn\endcsname{\let\PY@bf=\textbf\def\PY@tc##1{\textcolor[rgb]{0.00,0.50,0.00}{##1}}}
\expandafter\def\csname PY@tok@kr\endcsname{\let\PY@bf=\textbf\def\PY@tc##1{\textcolor[rgb]{0.00,0.50,0.00}{##1}}}
\expandafter\def\csname PY@tok@bp\endcsname{\def\PY@tc##1{\textcolor[rgb]{0.00,0.50,0.00}{##1}}}
\expandafter\def\csname PY@tok@fm\endcsname{\def\PY@tc##1{\textcolor[rgb]{0.00,0.00,1.00}{##1}}}
\expandafter\def\csname PY@tok@vc\endcsname{\def\PY@tc##1{\textcolor[rgb]{0.10,0.09,0.49}{##1}}}
\expandafter\def\csname PY@tok@vg\endcsname{\def\PY@tc##1{\textcolor[rgb]{0.10,0.09,0.49}{##1}}}
\expandafter\def\csname PY@tok@vi\endcsname{\def\PY@tc##1{\textcolor[rgb]{0.10,0.09,0.49}{##1}}}
\expandafter\def\csname PY@tok@vm\endcsname{\def\PY@tc##1{\textcolor[rgb]{0.10,0.09,0.49}{##1}}}
\expandafter\def\csname PY@tok@sa\endcsname{\def\PY@tc##1{\textcolor[rgb]{0.73,0.13,0.13}{##1}}}
\expandafter\def\csname PY@tok@sb\endcsname{\def\PY@tc##1{\textcolor[rgb]{0.73,0.13,0.13}{##1}}}
\expandafter\def\csname PY@tok@sc\endcsname{\def\PY@tc##1{\textcolor[rgb]{0.73,0.13,0.13}{##1}}}
\expandafter\def\csname PY@tok@dl\endcsname{\def\PY@tc##1{\textcolor[rgb]{0.73,0.13,0.13}{##1}}}
\expandafter\def\csname PY@tok@s2\endcsname{\def\PY@tc##1{\textcolor[rgb]{0.73,0.13,0.13}{##1}}}
\expandafter\def\csname PY@tok@sh\endcsname{\def\PY@tc##1{\textcolor[rgb]{0.73,0.13,0.13}{##1}}}
\expandafter\def\csname PY@tok@s1\endcsname{\def\PY@tc##1{\textcolor[rgb]{0.73,0.13,0.13}{##1}}}
\expandafter\def\csname PY@tok@mb\endcsname{\def\PY@tc##1{\textcolor[rgb]{0.40,0.40,0.40}{##1}}}
\expandafter\def\csname PY@tok@mf\endcsname{\def\PY@tc##1{\textcolor[rgb]{0.40,0.40,0.40}{##1}}}
\expandafter\def\csname PY@tok@mh\endcsname{\def\PY@tc##1{\textcolor[rgb]{0.40,0.40,0.40}{##1}}}
\expandafter\def\csname PY@tok@mi\endcsname{\def\PY@tc##1{\textcolor[rgb]{0.40,0.40,0.40}{##1}}}
\expandafter\def\csname PY@tok@il\endcsname{\def\PY@tc##1{\textcolor[rgb]{0.40,0.40,0.40}{##1}}}
\expandafter\def\csname PY@tok@mo\endcsname{\def\PY@tc##1{\textcolor[rgb]{0.40,0.40,0.40}{##1}}}
\expandafter\def\csname PY@tok@ch\endcsname{\let\PY@it=\textit\def\PY@tc##1{\textcolor[rgb]{0.25,0.50,0.50}{##1}}}
\expandafter\def\csname PY@tok@cm\endcsname{\let\PY@it=\textit\def\PY@tc##1{\textcolor[rgb]{0.25,0.50,0.50}{##1}}}
\expandafter\def\csname PY@tok@cpf\endcsname{\let\PY@it=\textit\def\PY@tc##1{\textcolor[rgb]{0.25,0.50,0.50}{##1}}}
\expandafter\def\csname PY@tok@c1\endcsname{\let\PY@it=\textit\def\PY@tc##1{\textcolor[rgb]{0.25,0.50,0.50}{##1}}}
\expandafter\def\csname PY@tok@cs\endcsname{\let\PY@it=\textit\def\PY@tc##1{\textcolor[rgb]{0.25,0.50,0.50}{##1}}}

\def\PYZbs{\char`\\}
\def\PYZus{\char`\_}
\def\PYZob{\char`\{}
\def\PYZcb{\char`\}}
\def\PYZca{\char`\^}
\def\PYZam{\char`\&}
\def\PYZlt{\char`\<}
\def\PYZgt{\char`\>}
\def\PYZsh{\char`\#}
\def\PYZpc{\char`\%}
\def\PYZdl{\char`\$}
\def\PYZhy{\char`\-}
\def\PYZsq{\char`\'}
\def\PYZdq{\char`\"}
\def\PYZti{\char`\~}
% for compatibility with earlier versions
\def\PYZat{@}
\def\PYZlb{[}
\def\PYZrb{]}
\makeatother


    % Exact colors from NB
    \definecolor{incolor}{rgb}{0.0, 0.0, 0.5}
    \definecolor{outcolor}{rgb}{0.545, 0.0, 0.0}



    
    % Prevent overflowing lines due to hard-to-break entities
    \sloppy 
    % Setup hyperref package
    \hypersetup{
      breaklinks=true,  % so long urls are correctly broken across lines
      colorlinks=true,
      urlcolor=urlcolor,
      linkcolor=linkcolor,
      citecolor=citecolor,
      }
    % Slightly bigger margins than the latex defaults
    
    \geometry{verbose,tmargin=1in,bmargin=1in,lmargin=1in,rmargin=1in}
    
    

    \begin{document}
    
    
    \maketitle
    
    

    
%    \hypertarget{estaduxedstica-y-procesos-estocuxe1sticos-epe}{%
%\section{Estadística y Procesos Estocásticos
%(EPE)}\label{estaduxedstica-y-procesos-estocuxe1sticos-epe}}

    Estadística y Procesos Estocásticos (EPE) es una asignatura de formación
matemática básica, que se estudia en el \textbf{segundo semestre del
primer curso} del \emph{Grado de Ingeniera en Tecnologías de la
Telecomunicación}, en la Escuela de Ingeniería de Telecomunicación y
Electrónica (EITE) de la Universidad de Las Palmas de Gran Canaria
(ULPGC). Esta asignatura también se cursa en el \textbf{primer semestre
del segundo curso} del Doble Grado en Dirección y Administración de
Empresas y de Ingeniería en Tecnologías de la Telecomunicación, en la
misma Escuela. El presente sitio web almacena un repositorio con los
contenidos informáticos necesarios para seguir la asignatura.

    En esta asignatura se explora el concepto de \emph{variabilidad
aleatoria} y la forma en que puede modelarse en los distintos contextos
que surgen en el ámbito de las telecomunicaciones.

    \hypertarget{contenidos}{%
\section{Contenidos}\label{contenidos}}

\begin{itemize}
\tightlist
\item
  Espacios de Probabilidad.
\item
  Variables aleatorias. Simulación de variables aleatorias.
\item
  Vectores aleatorios. Simulación de vectores aleatorios.
\item
  Introducción a los procesos estocásticos.
\item
  Recorridos aleatorios.
\item
  Simulación de procesos.
\item
  Procesos de Markov.
\item
  Cadenas homogéneas de Markov.
\item
  Simulación de cadenas de Markov.
\item
  Sistemas de colas.
\item
  Procesos de nacimiento y muerte.
\item
  Simulaciones de sistemas M/M/m.
\item
  Procesos estacionarios. Simulación de procesos estacionarios. Análisis
  espectral. Problemas de filtrado.
\item
  Estimación de procesos estacionarios. Ideas sobre la consistencia de
  la estimación.
\end{itemize}

    Los contenidos se exponen en clases teóricas y se demuestran en clases
prácticas en el aula, y con proyectos de laboratorio. Los estudiantes
deben contribuir activamente tanto en las clases prácticas en el aula
como en las sesiones de laboratorio. Se hace un uso intensivo de
software informático basado en Python, mediante el entorno Anaconda /
Jupyter notebooks, que facilita un acceso cómodo e intuitivo al conjunto
de paquetes de computación científica SciPy, cuyo conocimiento resultará
de suma utilidad al alumnado, no solo en esta asignatura sino también en
el futuro por su prestigio en la ciencia de datos actual en todo el
mundo. Los estudiantes han de completar tanto mediante su trabajo no
presencial como en las clases de laboratorio pequeños proyectos
demostrativos sobre tal entorno.



    \hypertarget{temario}{%
\section{Temario}\label{temario}}

Los contenidos de la asignatura se estructuran en tres bloques de cinco
semanas de duración, conforme al siguiente temario y programa. Las
claves que se utilizan son las siguientes:

\begin{itemize}
\tightlist
\item
  CLT: horas Clase Teoría,
\item
  CPA: horas Clase Problemas Aula,
\item
  LAB: horas Laboratorio,
\item
  Tut: horas de clase tutorizada,
\item
  Eva: horas Evaluación Continua
\end{itemize}

\textbf{Trabajo total presencial: 60 horas = 30 CLT + 14 CPA + 8 LAB + 2
Tut + 6 Eva}

    \textbf{Bloque temático I: Espacio de Probabilidad} (semanas 1 a cinco)

\begin{enumerate}
\def\labelenumi{\arabic{enumi}.}
\tightlist
\item
  Estadística. Introducción al Análisis Exploratorio de Datos. (4 CLT, 1
  CPA, 1 LAB)
\item
  Probabilidad y Álgebra de Sucesos.(2 CLT, 1 CPA)
\item
  Probabilidad condicionada. Independencia de sucesos. Verosimilitud (1
  CLT, 0.5 CPA, 0.25 LAB)
\item
  Teorema de la Probabilidad Total. Teorema de Bayes y probabilidad a
  posteriori (1 CLT, 0.5 CPA, 0.25 LAB)
\item
  Decisores de máxima verosimilitud (ML). Decisores máximo a posteriori
  (ML) (1 CLT, 0.5 CPA, 0.25 LAB)
\item
  Experimentos compuestos. Pruebas de Bernoulli (1 CLT, 0.5 CPA, 0.25
  LAB)
\end{enumerate}

    \begin{itemize}
\tightlist
\item
  La segunda semana habrá una clase tutorizada de introducción a Python.
\item
  La quinta semana finaliza con una evaluación de dos horas, dentro de
  la evaluacion continua
\item
  Totales trabajo presencial: 20 horas = 10 CLT + 4 CPA + 2 LAB + 2 Tut
  + 2 Eva
\end{itemize}

    \begin{longtable}[]{@{}lllllll@{}}
\toprule
\textbf{Semanas} & \textbf{CLT} & \textbf{CPA} & \textbf{LAB} &
\textbf{Tut} & \textbf{Eva} & \textbf{TOT}\tabularnewline
\midrule
\endhead
Semana 1 & 4 & 0 & 0 & 0 & 0 & \textbf{4}\tabularnewline
Semana 2 & 2 & 0 & 0 & 2 & 0 & \textbf{4}\tabularnewline
Semana 3 & 2 & 2 & 0 & 0 & 0 & \textbf{4}\tabularnewline
Semana 4 & 2 & 1 & 1 & 0 & 0 & \textbf{4}\tabularnewline
Semana 5 & 0 & 1 & 1 & 0 & 2 & \textbf{4}\tabularnewline
\textbf{TOT hrs} & \textbf{10} & \textbf{4} & \textbf{2} & \textbf{2} &
\textbf{2} & \textbf{20}\tabularnewline
\bottomrule
\end{longtable}

    

    \textbf{Bloque temático II: Variable aleatoria} (semanas 6 a diez)

\begin{enumerate}
\def\labelenumi{\arabic{enumi}.}
\tightlist
\item
  Variable aleatoria discreta. Funciones de masa de probabilidad.
  Caracterización (1 CLT)
\item
  Distribuciones de probabilidad habituales (1 CLT, 1 LAB)
\item
  Variable aleatoria continua. Funciones de densidad y distribución de
  probabilidad. Caracterización (1.5 CLT, 1 CPA)
\item
  Variable aleatoria Gaussiana. Otras variables aleatorias continuas
  habituales. (1.5 CLT, 1 CPA, 1 LAB)
\item
  Transformación de variables aleatorias (1 CLT, 1 CPA)
\item
  Variable aleatoria bidimensional. Caracterización conjunta de vectores
  aleatorios bidimensionales (1.5 CLT, 1 CPA)
\item
  Distribución Gaussiana bidimensional. Mezcla de Gaussianas.
  Clasificación y Estimación (1.5 CLT, 1 CPA, 1 LAB)
\item
  Variable aleatoria multidimensional. Caracterización conjunta de
  vectores aleatorios n-dimensionales (0.5 CLT)
\item
  Distribución Gaussiana multidimensional. Mezcla de Gaussianas.
  Clasificación y Estimación (0.5 CLT)
\end{enumerate}

    \begin{itemize}
\tightlist
\item
  La décima semana finaliza con una evaluación de dos horas, dentro de
  la evaluacion continua
\item
  Totales trabajo presencial: 20 horas = 10 CLT + 5 CPA + 3 LAB + 2 Eva
\end{itemize}

    \begin{longtable}[]{@{}lllllll@{}}
\toprule
\textbf{Semanas} & \textbf{CLT} & \textbf{CPA} & \textbf{LAB} &
\textbf{Tut} & \textbf{Eva} & \textbf{TOT}\tabularnewline
\midrule
\endhead
Semana 6 & 4 & 0 & 0 & 0 & 0 & \textbf{4}\tabularnewline
Semana 7 & 2 & 2 & 0 & 0 & 0 & \textbf{4}\tabularnewline
Semana 8 & 2 & 1 & 1 & 0 & 0 & \textbf{4}\tabularnewline
Semana 9 & 2 & 1 & 1 & 0 & 0 & \textbf{4}\tabularnewline
Semana 10 & 0 & 1 & 1 & 0 & 2 & \textbf{4}\tabularnewline
\textbf{TOT hrs} & \textbf{10} & \textbf{5} & \textbf{3} & \textbf{0} &
\textbf{2} & \textbf{20}\tabularnewline
\bottomrule
\end{longtable}

    

    \textbf{Bloque temático III: Secuencias y procesos aleatorios} (semanas
11 a 15)

\begin{enumerate}
\def\labelenumi{\arabic{enumi}.}
\tightlist
\item
  Secuencias y procesos estocásticos. Caracterización de segundo orden.
  Secuencias y procesos Gaussianos. (2 CLT, 1 CPA)
\item
  Estacionariedad. Ergodicidad. Estimación de parámetros. (1 CLT)
\item
  Señal en ruido. Filtrado temporal. Regresión y estimación.
  Clasificación y detección. (3 CLT, 2 CPA, 2 LAB)
\item
  Introducción a la caracterización espectral de secuencias y procesos
  estocásticos. Filtros frecuenciales. (1 CLT).
\item
  Condición de Markov. Cadenas de Markov. Procesos puntuales. El proceso
  homogéneo de Poisson. (1 CLT, 1 CPA)
\item
  Procesos de nacimiento y muerte. Sistemas de colas: Sistemas M/M/1 y
  M/M/m. (2 CLT, 1 CPA, 1 LAB)
\end{enumerate}

    \begin{itemize}
\tightlist
\item
  La décimo quinta semana finaliza con una evaluación de dos horas,
  dentro de la evaluacion continua
\item
  Totales trabajo presencial: 20 horas = 10 CLT + 5 CPA + 3 LAB + 2 Eva
\end{itemize}

    \begin{longtable}[]{@{}lllllll@{}}
\toprule
\textbf{Semanas} & \textbf{CLT} & \textbf{CPA} & \textbf{LAB} &
\textbf{Tut} & \textbf{Eva} & \textbf{TOT}\tabularnewline
\midrule
\endhead
Semana 11 & 4 & 0 & 0 & 0 & 0 & \textbf{4}\tabularnewline
Semana 12 & 2 & 2 & 0 & 0 & 0 & \textbf{4}\tabularnewline
Semana 13 & 2 & 1 & 1 & 0 & 0 & \textbf{4}\tabularnewline
Semana 15 & 2 & 1 & 1 & 0 & 0 & \textbf{4}\tabularnewline
Semana 15 & 0 & 1 & 1 & 0 & 2 & \textbf{4}\tabularnewline
\textbf{TOT hrs} & \textbf{10} & \textbf{5} & \textbf{3} & \textbf{0} &
\textbf{2} & \textbf{20}\tabularnewline
\bottomrule
\end{longtable}

    


    % Add a bibliography block to the postdoc
    
    
    
    \end{document}
